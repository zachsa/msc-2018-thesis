\subsection{Overview}
Many software packages exist to facilitate data processing; spreadsheet programs such as Microsoft Excel (or their online competition Google Sheets), databases such as Microsoft Access, MySQL are designed for this time of work.

But these tools fail when looking at scale of storage that Sakai's event data requires either in terms of ability (Microsoft Excel can work with a theoretical maximum of around 1 million lines, Access databases can be maximum 2GB), complexity or cost.

This thesis looks at creating a cost-effective 'analysis-engine' capable of scaling to many times the amount of data that a single machine can hold in a way that is still effective to analyze (distributed computing) and not too mentally taxing to implement.

specifically this project analyzed the viability of a software package called CouchDB, a NoSQL DBMS as a replacement for conventional RDMSs when analyzing data that has conventionally been housed RDBMSs (usually with an Oracle, Microsoft or SAP price tag associated with it).

Coming from a relational database environment there are a plethora of tools avaialable that facilitate transfer of CSV dato to a DBMS. These tools are available at a variety of different levels of extraction depending on a users technical skillset, time constraints and requirements. xxx: list some of these tools.

In a SQL Server environment SSDT (formerly SSIS) is considered the de facto standard for extracting/transforming and loading data between different data sources. xxx: find a graph on the usage of SSDT/SSIS in companies.

In fact it's likely that the availability of of SSDT/SSIS has influenced the uptake of SQL Server in operations that require dealing with large amount of data. It's fair to say that a barrier to using open source software such as CouchDB is the LACK of such software. Bespoke scripts are currently the only viable way of interfacing with CouchDB in a way that is comparable to SQL Server and SSDT. But with high-level languages such as node.js maturing, and the proliferation of small, focused libraries in these languages that abstract much of the unpleasant and gnarly aspects of bespoke scripting (xxx examples), bespoke data-scripting is nowhere near as difficult as it would be within the Microsoft environment (C\# or VB).

In line with the requirement of transferring large amounts of CSV data from a CSV source to CouchDB, and taking into account the comparatively low entry barrier to bespoke data-transformation scripts, a component of this MSc is an exploration of a possible alternative to SSDT for an environment other than Microsoft's SQL Server. This MSc project actually has several requirements that fall within the ETL spectrum that such a framework could easily be adapted to handle in a generic way. The framework has been published as an npm library and is available at ..., with source code available at github somewhere xxx.