% \bibliography{../bibliography/msc_citations}

\subsection{Motivation \& Aim}
Unstructured datasets such as popularized by NoSQL databases such as CouchDB, Couchbase, Mongo and other document stores provide several advantages over traditional RDMSs as highlighted by the Couchbase project \cite{couchbaseWhitePaper}, where is mentioned that much of that data being produced on a day-to-day basis is semi-structured or unstructured due to the diversity of the kind of devices and users that are using data-collecting devices. RDBMSs seem ungainly in this scenario with their strictly defined data models making handling unstructured data cumbersome and expensive in terms of implementation time, and complex in terms of architecture and design. Additionally, systems like Oracle, DB2, SQL Server, MySQL and others scale more easily vertically than they do horizontally \cite{couchbaseWhitePaper}, which is comparatively limiting and expensive. Unlike RDMSs, NoSQL databases with their unstructured data models allow for easy expansion beyond single servers to many, many servers fairly easily and allow for a more agile approach to data modeling since data does not have to be statically modeled, and a model can changed very easily as a system evolves.

CouchDB is a scalable JSON storage that allows for database sharding (clustering features were added to CouchDB 2.0 released in 2016 with the merge of IBM's Cloudant code \cite{couchdb2.0}) across multiple commodity servers very easily. Theoretically, CouchDB as a data store is suitable for storing an unlimited amount of unstructured data at affordable infrastructure costs. It is also substantially cheaper to license than many RDBMSs since it is open source and available for free. This project looks to assess CouchDB in terms of being a viable alternative to working with data where SQL operations (JOINS in particular) are a common business requirement.

As a side note, CouchDB also supports an HTTP (RESTful) API that allows for interacting with the data-layer from any HTTP(S) clients such as browsers. Combined with CouchDB's handling of file data (attachments), and a couple other features (show functions, list functions, URL rewrites and a vHost engine to name a few) CouchDB can double as a web server that serves HTML - what the CouchDB community has traditionally referred to as "CouchApps". These tools allow for designing bespoke database-management tools very quickly and are definitely a CouchDB draw-card for many as seen in this open email exchange \cite{googleCon2017} by Apache Foundation members. However, this same exchange shows (unfortunately) that the "CouchApp" feature of CouchDB is unlikely to survive future releases due to lack of interest in contributing to this feature.