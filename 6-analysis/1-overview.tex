\label{chapter-analysis}

The nETL and CouchDB applications are incorporated into an analysis workflow as represented in Figure \ref{analysis}. CSVs are parsed by the nETL application and loaded into a CouchDB database via the CouchDB server application. An index is created from the CouchDB database, and data is retrieved directly from the index file using a List function. The index is primarily used as a means of sorting the data from the main CouchDB file; both the 2-way join and 3-way join are performed directly on the sorted index rather than in the MapReduce function. Although the MapReduce function fills the important role of aggregating entities prior to joining them (as discussed in 3-way join section), it is technically possible to join documents directly on retrieval from the main database file since both the main database file and derived indexes are structured as B+trees and can be sorted. However there are three problems with retrieving data directly from the main database file and skipping index creation:

\begin{itemize}
    \item CouchDB database files are sorted by the ``\_id'' field, which when unspecified on document insert is initialized as a UUID. Using UUIDs as unique document identifiers allow for dispersed systems, and so even though not required by this project, best practices in this case are followed. A B+tree sorted by a UUID is not useful for document retrieval, and as such, views are required of the underlying datastore for any kind of sorted querying
    \item Since List functions are invoked via an HTTP GET request with the requirement of specifying a view within the URI, without using a view in this project an alternative means of data retrieval to using List functions would be required. Although List functions are effectively deprecated and alternative means of database interaction are recommended, they exist in the current release of CouchDB 2.1.1. They are convenient and present a strong case for usage in terms of this project
    \item When aggregations of specific entities are required, retrieving data directly from these indexes is vastly easier than having to aggregate during data retrieval. As aggregation logic becomes more complicated the difficulty of such direct data retrieval increases and the benefit of using indexes increases as a result
\end{itemize}

In terms of defining MapReduce tasks, the map function is always user-defined, whereas only built-in reduce functions are used. The built-in reduce functions are implemented within the main Erlang process, which according to the documentation offers a performance boost since the IO transfer cost between the Erlang process and the view engine (couchjs.exe by default) is negated. Working on a Windows machine the IO cost is apparently exaggerated (see the slack correspondence with Jan Lehnardt in appendix \ref{slack-1-nov}) due to the difference between Unix-based and Windows kernel implementations.

During analysis, runtime results of the different components of the system are recorded and are shown in Table \ref{performance-analysis}. The metrics include running time of nETL tasks, a summary of the data processed by nETL, CouchDB indexing times, and database/index storage footprints.

\section{MapReduce Approach}
Initial attempts at joining the three entities - Grades with Benchmarks, and then Grades with Benchmarks with Events were attempted directly via MapReduce (the custom Map and \_stats reduce function). This approach involved specifying the map function to always output a value of the same format for all three entities. This format is a tuple of 11 indexes - [0, 0, 0, 0, 0, 0, 0, 0, 0, 0, 0]; each of the 11 indexes correlate to a specific output attribute. In other words, i = 0 (indexes in JavaScript start at 0) correlates to Grade \%, i = 1,2,3,4,etc. correlated to the percents for Benchmark data, and the last 2 indexes (i = 9, i = 10) indicate an event occurrence (from the Event data) in either the first semester (i[9] = 1, i[10] = 0) or the second semester (i[9] = 0, i[10] = 1).

During index calculation, CouchDB then iterates through every database document, passing each document to the Map function. This Map function instantiates the output tuple ([0, 0, 0, 0, 0, 0, 0, 0, 0, 0, 0]) on every execution, and then depending on the value of the `type\_' attribute of the document it's processing (this attribute is added to every document prior to insertion to CouchDB), the map function alters the tuple at certain indexes before returning and adding the key:value set to the view index currently being built.

To further explain, if the document being processed by the map function is a ‘grade’ document, then the map function adjusts the value tuple to emit the grade percent – i.e. [percent, 0, 0, 0, 0, 0, 0, 0, 0, 0, 0]. Or if the document is of type ‘demographic’, the map function changes the values at indexes 2 through 9 and emits the value: [0, x, x, x, x, x, x, x, x, 0, 0]. If the document is of type ‘event’, then the map function alters the value at indexes 9 and 10. i.e. [0, 0, 0, 0, 0, 0, 0, 0, s1Event, s2Event] (s1Event is 1 for first semester event, or 0 for second semester, etc.).

The reduce function (whether that is the \_stats function a used in this study or any other function) then receives a tuple of tuples (a list of the tuples output by the map function executions), as the output per key, and can perform calculations across corresponding indexes; i.e. a single key references a list of 6 tuples: 1 Grade document output, 1 Benchmark document output, and 4 Event document outputs. By performing an aggregation across these 6 tuples at corresponding indexes, the Grades percentage (i=1) is an aggregation of the grade as output from the Grade entity along with five 0s, the Benchmark percentages are each aggregations of the Benchmark percent each along with five 0s, and the Events data is an aggregation of four event counts and two 0s. As such, aggregation across the output tuple effectively only takes into account relevant values (since all other values are 0) and a join is achieved.

However, for this approach to work, the reduce function needs to be able to group by common keys. To perform a grouping on the compound key [Student ID, Course, Course Year], all the entities need to output data in this format; but the Events data doesn't include Course, and the the Benchmarks data doesn't include Course or Course Year. As such, to allow for Grade data to be joined to Benchmark data on Course and Course Year in addition to Student ID, each Benchmark document needs to be output for every possible combination of Course and Course Year per student. That is the same with the Events documents - each Event needs to be output for every possible course that a student took in a given year.

In terms of performance this approach is disastrous. To analyze 40 courses taken over 3 years, each Benchmark document needs to be emitted for a student $40 x 3 = 120$ times so that the key of the Grade document [Student ID, Course, Year] can always be joined to Benchmark document. Likewise, Each Event data (which has year but not course information) needs to be emitted 40 times - once for each course a join could potentially be performed on. Considering that there are several million event documents this is impracticable.

Instead, CouchDB's usage of B+trees as a means of sorting view indexes by keys is utilized to allow for joining the 3 documents in the final dataset. Figure \ref{mapreduce-key-output} shows the key format for each type of document the Map function processes (Grade, Event, Benchmark). Keys are defined so that the index produced by the Map function is always ordered per particular student: i.e. for a particular Student ID, and scanning the B+tree, the first document found will be the Benchmark document, followed by Event document, and followed by Grade documents. The reduce function is then used to create an aggregation of all the event documents so that retrieval from the reduced view-index, for any student ID, iteratively produces first a Benchmark document, then a single (aggregated) Event document, then a single Grade document for each course that student enrolled in. This results in a much more efficient way of aggregating different types of documents for a given set of keys than was first attempted.

With ordered output, and Event data already aggregated via the reduce function, data retrieval involves iterating over view indexes, processing a single ID at a time. This is very efficient in terms of memory usage since only documents relating to a single ID need to be held in memory at a time.

Part of the MapReduce logic includes a normalization of percentage fields, which may occur either as number or symbol (A, B, C, D, etc) values. Normalization is performed by a sub-routine defined within the Map function body. Table \ref{tbl-grades-normalize} shows the logic used to translate grade symbols to percentages, and Table \ref{tbl-benchmarks-normalize} shows the same logic for the Benchmarks data.

TODO: activity diagram showing the sub-routing logic.