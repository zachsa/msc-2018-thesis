\section{Overview}
Producing a dataset suitable for analysis involves the workflow specified in \ref{analysis-workflow}. Effectively this is a 3 stage process, where a user configures and runs the \textit{nETL} application to load the data into CouchDB, and then configures a design document with instructions for CouchDB on how to produce an index and emit the contained data in CSV form. Once a CSV of the joined data has been obtained, the final stage of analysis is to load the data into Excel and produce results.

\begin{figure}[ht]
    \centering
    \begin{mdframed}
        \centering
        \includegraphics[scale=0.35]{./resources/figures/analysis-workflow.png}
    \end{mdframed}
    \caption[Analysis Workflow]{\textbf{Figure \ref{analysis-workflow}: Workflow to perform an analysis.}1) User creates a configuration file (JSON) that is loaded into the running \textit{\_nETL} service. This configuration includes instructions on which CSVs to load, which modules should be loaded to process the CSVs, and configurations for the modules. 2) Modules are loaded from a library of available modules into the \textit{\_nETL} service. 3) CSVs are loaded into the service, and transformations are applied to the CSV data as specified by the configuration in (1). 4) Data from the CSVs is loaded into CouchDB; this is also achieved via a module specified in (1). 5) A user creates a CouchDB design document, specifying the MapReduce functions, and a List function. 6) The user asks CouchDB to produce the view index as specified by the design document in (5). 7) The user retrieves the data from the view index using the list function specified in (5). 8) The user then loads the resultant CSV into Excel to produce useful metrics.}
    \label{analysis-workflow}
\end{figure}

Analyses are conducted in an iterative fashion; with each iteration an increasing volume of data is handled so as to gain insight into the viability of handling varying amounts of data in CouchDB. Data volume is controlled by the number of courses analyses (more courses taken into account results in higher volumes of data), And the number of entities joined together (grades \& FU data vs grades, FU data, \& Sakai usage). Results are discussed in terms of the insights into business domain (EDM) as well as the effectiveness of the data-mining approach.

\section{Result 1}
This initial analysis is a comparison of student benchmarks (FU data) with CSC1015F course grades.






To maintain the resolution that exists within the Grade data, that is; \textit{results of a particular student in a particular year for a particular course}, joins are performed using a compound key of the tuple [studentID, courseCode, year], i.e. that map functions should emit a key of this tuple.







Configurations used for \textit{nETL} for all the analyses - see \ref{netl-run1-config}). Runtime results of \textit{nETL}, CouchDB indexing times, database/index storage footprints are shown in Table \ref{performance-analysis}.



\section{Join of Grades/Demographics}
Creating datasets comprising grade and FU data involved filtering CSVs and loading that data into CouchDB using \textit{nETL}

For a join on the Grades and Demographics entities, the map function is configured to output key-value pairs of the form: \textless studentID, courseCode, year \textgreater : <9 element list>. A description of the 11-element value list is shown in \ref{grades-join-demographics-output}. Configuration for the \textit{nETL} task is shown in the appendix (see \ref{netl-config-grades-join-demographics}), as is the Map function and list function (see\ref{msc-design-doc}).

The list shown in \ref{grades-join-demographics-output} shows ALL the values output by a map function on the join between all three entities. For the first two runs, event information is NOT emitted.

\begin{figure}[ht]
    \centering
    \begin{minted}{javascript}
 [
    // Grade Entitty output
    "CSC1015F %",

    // Demographic Entity output
    "Gr12 Eng %", 
    "Gr12 Sci %",
    "Gr12 Mth %",
    "Gr12 Mth Lit %",
    "Gr12 Mth Adv %",
    "NBT AL %",
    "NBT QL %",
    "NBT Mth %",

    // Event entity output
    "eventCount S1", // Only included for Run 3/4
    "eventCount S2" // Only included for Run 3/4
 ]    
    \end{minted}
    \caption[2-way-join map output]{\textbf{Figure \ref{grades-join-demographics-output}: Output of map function for grades joined with demographics.} This list, shown as a JavaScript array, is the key to the map function output. In other words, the map function outputs a list of values that correspond to this list. When retrieving the view output, headers can be given back to the columns retrieved using this key. View output can be achieved via a List function (as has been done in this project), or via bespoke JavaScript code.}
    \label{grades-join-demographics-output}
\end{figure}

% Runs
\subsection{Run 1: CS1015F}
To create the dataset to show correlation between student benchmarking and achieved grades in the CS1015F course for undergraduates, \textit{nETL} was configured to filter Grade data on the field 'Course' to only allow the value 'CS1015F' in addition to the filtering and transforming \textit{nETL} is configured to on all Grade rows as described in the data overview. Additional filtering on the Demographic entity is performed on the StudentID field (anonIdnew) to load demographic data of the students who took the CSC2015F course. This list of students was prepared by Excel filtering, but could just as easily have been performed in CouchDB or any other database if the size of the file did not allow for opening in Excel. Since the view index is small, the list function returns the CSV output instantly. The CSV output size is 67Kb

\subsection{Run 2: Multiple Courses}
With the ease at which the \textit{nETL} software handled loading data in Run 1, and the ease at which CouchDB was able to handle indexing, a second run was created with multiple courses. For this run, 40 courses were selected: ECO1010F, ECO1011S, ACC1006F, STA1000S, ECO2003F, BUS1036F, ECO2004S, CML1001F, MAM1010F, PSY1004F, FTX2024S, ECO2007S, ACC2011S, CSC1015F, PHI2043S, ACC3023S, INF2004F, PSY1005S, STA2020F, CML2010S, CML2001F, SOC1001F, ACC2018S, SOC1005S, BUS2010F, ACC2012W, AXL1100S, ACC3022H, ACC3004H, PHY1012F, MAM1020F, PHI2043F, FTX3045S, ACC3009W, MAM1012S, FTX3044F, MAM1000W, POL1004F, CSC1016S, ACC4000H. These courses were selected simply because they have a high level of student enrollment. \textit{nETL} filtering on the courseCode field was configured to allow all these codes. Filtering on student IDs in the demographic was removed, since over 10 000 student numbers would need to be included a list for such a filter.

Compared to Run1, there is a significant increase in the size footprint of the view index (from \textless 1MB to 143MB) do to the requirement of duplicating demographic output in the map function. Performance of the list function degraded considerably, with streaming of the 2MB CSV taking several minutes. To efficiently retrieve view output of larger indexes would require implementing data retrieval outside of CouchDB. List functions require iterating through every view output individually, whereas working with the view directly would allow for retrieving in batches which is much more efficient. \textit{nETL} could be configured to do this, but hasn't been.

\subsection{Correlation analysis}
The results for \textit{Run 1} and \textit{Run 2} have been summarized in the graphs in Figure \ref{run1-chart1}; several courses were picked at random to show correlation between course grades and student benchmarks. The results show that in general, higher benchmarking scores are indicative of higher course results overall. Some of the benchmarks show stronger correlation to grade results than other, as seen by steeper trendline gradient.

\begin{figure}[H]
    \centering
    \begin{mdframed}
        \centering
        \includegraphics[scale=0.55]{./resources/figures/run1-chart1.png}
    \end{mdframed}
    \caption[CSC1015 grade vs benchmark correlation]{\textbf{Figure \ref{run1-chart1}: Correlation between student benchmarks and CSC1015F results.} This graph shows benchmark scores plotted against the final CSC1015F grade. Datasets focus on a single benchmark, so a trendline shows the correlation between a single benchmark and final grades.}
    \label{run1-chart1}
\end{figure}

\section{join of Grades/Demographics/Events}

% Runs
\subsubsection{Run 3: CS1015F}
Building on the results from \textit{Run 1}, a join on events data was added. To achieve this, n\textit{nETL} was reconfigured to filter grades for only 2016 (since event data is only available for 2016), and an additional \textit{nETL} task was configured to load event data. Filtering on the event data was performed on the (anonymized) uct\_id field to include only events from students who took the CSC1015F course in 2016. This list was derived from the Grade data using Excel.


\subsubsection{Run 4: Multiple Courses}


\subsubsection{Correlation analysis}
