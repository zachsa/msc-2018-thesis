\begin{figure}[ht]
    \centering
    \begin{minted}{javascript}
 [
    // Grade Entitty output
    "CSC1015F %",

    // Demographic Entity output
    "Gr12 Eng %", 
    "Gr12 Sci %",
    "Gr12 Mth %",
    "Gr12 Mth Lit %",
    "Gr12 Mth Adv %",
    "NBT AL %",
    "NBT QL %",
    "NBT Mth %",

    // Event entity output
    "eventCount S1", // Only included for Result 3/4
    "eventCount S2" // Only included for Result 3/4
 ]    
    \end{minted}
    \caption[2-way-join map output]{\textbf{Figure \ref{grades-join-demographics-output}: Output of map function for grades joined with demographics.} This list, shown as a JavaScript array, is the key to the map function output. In other words, the map function outputs a list of values that correspond to this list. When retrieving the view output, headers can be given back to the columns retrieved using this key. View output can be achieved via a List function (as has been done in this project), or via bespoke JavaScript code.}
    \label{grades-join-demographics-output}
\end{figure}


\begin{table}[h]
    \begin{threeparttable}
        \textbf{Table \ref{performance-analysis}}\par\medskip\par\medskip
        \caption[Software performance analysis]{Running time analysis of \textit{nETL} tasks and CouchDB MapReduce indexing}
        \label{performance-analysis}
        \begin{tabularx}{\textwidth}{>{\hsize=1\hsize}Y>{\hsize=1\hsize}X>{\hsize=1\hsize}X>{\hsize=1\hsize}X>{\hsize=1\hsize}X}
            \toprule
            \mC{c}{}                                               & \mC{c}{Run 1} & \mC{c}{Run 2} & \mC{c}{Run 3} & \mC{c}{Run 4} \\
            \midrule
            Demographic lines extracted                            & 12 219        & 12 219        & 12 219        &               \\
            Demographic lines loaded                               & 1 381         & 9 874         & 595           &               \\
            Demographic task time (sec)\tnote{\textsuperscript{1}} & 2.488         & 3.114         & 6.755         &               \\
            Grade lines extracted                                  & 513 872       & 513 872       & 513 872       &               \\
            Grade lines loaded                                     & 1 891         & 79 849        & 738           &               \\
            Grade task time (sec)\tnote{\textsuperscript{1}}       & 37.684        & 42.001        & 97.221        &               \\
            Events lines extracted                                 & -             & -             & 44 420 508    &               \\
            Events lines loaded                                    & -             & -             & 661 555       &               \\
            Events task time (sec)\tnote{\textsuperscript{1}}      & -             & -             & 2 225.44      &               \\
            CouchDB footprint (MB)\tnote{\textsuperscript{2}}      & 0.9           & 23.2          & 172.1         &               \\
            View calculation time (sec)\tnote{\textsuperscript{3}} & 0.685         & 49.042        & 340.413       &               \\
            View size (MB)                                         & 0.813         & 143           & 521           &               \\
            \bottomrule
        \end{tabularx}
        \scriptsize
        \begin{tablenotes}
            \item[\textsuperscript{1}]Tasks are run asynchronously, so time taken includes processing of other tasks in this run. Task run times are printed out to the log
            \item[\textsuperscript{2}]This is representative of the amount of data processed by \textit{nETL}
            \item[\textsuperscript{3}]CouchDB views are calculated per shard. By default a database contains 8 shards (even in single node mode). The log file shows start and end times of view calculations for each shard, the time is taken as time the first shard starts indexing, to the time the last shard stops indexing.
        \end{tablenotes}
    \end{threeparttable}
\end{table}

