\subsubsection{Run 1: CS1015F}
To create the dataset to show correlation between student benchmarking and achieved grades in the CS1015F course for undergraduates, \textit{nETL} was configured to filter Grade data on the field 'Course' to only allow the value 'CS1015F' in addition to the filtering and transforming \textit{nETL} is configured to on all Grade rows as described in the data overview. Additional filtering on the Demographic entity is performed on the StudentID field (anonIdnew) to load demographic data of the students who took the CSC2015F course. This list of students was prepared by Excel filtering, but could just as easily have been performed in CouchDB or any other database if the size of the file did not allow for opening in Excel. Since the view index is small, the list function returns the CSV output instantly. The CSV output size is 67Kb

\subsubsection{Run 2: Multiple Courses}
With the ease at which the \textit{nETL} software handled loading data in Run 1, and the ease at which CouchDB was able to handle indexing, a second run was created with multiple courses. For this run, 40 courses were selected: ECO1010F, ECO1011S, ACC1006F, STA1000S, ECO2003F, BUS1036F, ECO2004S, CML1001F, MAM1010F, PSY1004F, FTX2024S, ECO2007S, ACC2011S, CSC1015F, PHI2043S, ACC3023S, INF2004F, PSY1005S, STA2020F, CML2010S, CML2001F, SOC1001F, ACC2018S, SOC1005S, BUS2010F, ACC2012W, AXL1100S, ACC3022H, ACC3004H, PHY1012F, MAM1020F, PHI2043F, FTX3045S, ACC3009W, MAM1012S, FTX3044F, MAM1000W, POL1004F, CSC1016S, ACC4000H. These courses were selected simply because they have a high level of student enrollment. \textit{nETL} filtering on the courseCode field was configured to allow all these codes. Filtering on student IDs in the demographic was removed, since over 10 000 student numbers would need to be included a list for such a filter.

Compared to Run1, there is a significant increase in the size footprint of the view index (from \textless 1MB to 143MB) do to the requirement of duplicating demographic output in the map function. Performance of the list function degraded considerably, with streaming of the 2MB CSV taking several minutes. To efficiently retrieve view output of larger indexes would require implementing data retrieval outside of CouchDB. List functions require iterating through every view output individually, whereas working with the view directly would allow for retrieving in batches which is much more efficient. \textit{nETL} could be configured to do this, but hasn't been.

\subsubsection{Correlation analysis}
TODO: graphs from the CSV analysis