\section{Aggregation Example}
\label{aggregation}
Admissions data can be used to profile students based on their Grade 12 results, NBT scores, or some combination of the two. In other words, several benchmarks datasets can be derived from the admissions data. The variance and standard deviation of each of these datasets is calculated below to provide an example of aggregation using CouchDB.

\subsection{ETL}
Rows are extracted from the admissions data in batches of 5 000. Each row is converted into an object, and then the objects for undergraduate students who have citizenship or permanent residency in South Africa and who have a grade for the CSC1015F course are selected. A \textit{type\_} attribute with the value 'admission' is added to each row (now an object). Batches of objects (there are at most 5 000 objects per batch, but usually far fewer due to the filtering) are loaded into a CouchDB database. An example of a row from the admissions data serialized to a JSON string and loaded into CouchDB is shown in Figure \ref{fig-json-admission}.

\begin{figure}[H]
    \centering
    \begin{mdframed}
        \centering
        \begin{minted}{text}
{
    "_id": "6587aa5b36bba2a70aeba96d06f05d2b",
    "_rev": "1-8c7d395e046e452442907e3388c74b41",
    "anonIDnew": 3103212,
    "Eng Grd12 Fin Rslt": 58,
    "Math Grd12 Fin Rslt": 73,
    "Mth Lit Grd12 Fin Rslt": "",
    "Adv Mth Grd12 Fin Rslt": "",
    "Phy Sci Grd12 Fin Rslt": 67,
    "NBT AL Score": 53,
    "NBT QL Score": 43,
    "NBT Math Score": 60,
    "RegAcadYear": 2016,
    "type_": "demographic"
}         
        \end{minted}
    \end{mdframed}
    \caption[Serialized admissions document]{\textbf{Figure \ref{fig-json-admission}: Serialized admissions document}}
    \label{fig-json-admission}
\end{figure}

\subsection{Indexing}
All the admissions documents are loaded into the CouchDB database and mapped to an index consisting of \textit{anonIDnew:[benchmarks]} key-value pairs. Derivative benchmarks are calculated during map function execution before being incorporated into the index – each value added to the index is a tuple of benchmarks categorized according to the following indices:

\begin{minted}[firstnumber=0]{text}
Gr12 Eng % 
Gr12 Sci % 
Gr12 Mth % 
NBT AL % 
NBT QL % 
NBT Mth % 
Avg Gr12 % 
Avg Gr12 % (Dbl Mth)
Avg Gr12 % (Dbl Mth \& Sci)
Avg NBT % 
Avg NBT % (Dbl AL)
Avg NBT % (Dbl QL)
Avg NBT % (Dbl Mth)
Avg NBT % (Dbl AL/QL)
Avg NBT % (Dbl AL/Mth)
Avg NBT % (Dbl QL/Mth)
Avg Gr12 & NBT 
Avg Gr12 & NBT (Dbl Gr12 Mth)
Avg Gr12 & NBT (Dbl Gr12 Mth & Sci)
\end{minted}

The built-in \_stats function is used to aggregate values in the view-index by benchmark type, which when used with \mintinline{text}{group=false\footnote{index keys are ignored and treated as \mintinline{text}{null}, as a result all index values are grouped together}} results in the index output of a single stats-object, as shown in Figure \ref{fig-stats-output}\footnote{Values are rounded for better display} (indices of each object in the value output are indicated on the right-hand side of the figure). The objects produced by the \_stats function (one object per benchmark dataset) contain fields for:

\begin{itemize}
    \item The sum of all the items in the dataset
    \item The count of all the items in the dataset
    \item The smallest number in the dataset
    \item The largest number in the dataset
    \item The sum of each item squared in the dataset
\end{itemize}

\begin{figure}[H]
  \centering
  \begin{mdframed}
    \centering
    \begin{minted}{text}
{"rows": [{
"key": null,    
"value": [
  {"sum": 71751,"count": 908,"min": 50,"max": 97,"sumsqr": 5720599},  // 0
  {"sum": 74174,"count": 908,"min": 48,"max": 100,"sumsqr": 6151326}, // 1
  {"sum": 78802,"count": 908,"min": 63,"max": 100,"sumsqr": 6900682}, // 2
  {"sum": 67207,"count": 908,"min": 33,"max": 94,"sumsqr": 5057191},  // 3
  {"sum": 69713,"count": 908,"min": 27,"max": 98,"sumsqr": 5512393},  // 4
  {"sum": 69136,"count": 908,"min": 29,"max": 98,"sumsqr": 5439872},  // 5
  {"sum": 74909,"count": 908,"min": 61,"max": 98,"sumsqr": 6227518},  // 6
  {"sum": 75882,"count": 908,"min": 62,"max": 98,"sumsqr": 6389866},  // 7
  {"sum": 75541,"count": 908,"min": 61,"max": 98,"sumsqr": 6338061},  // 8
  {"sum": 68685,"count": 908,"min": 39,"max": 94,"sumsqr": 5288410},  // 9
  {"sum": 68316,"count": 908,"min": 40,"max": 94,"sumsqr": 5221243},  // 10
  {"sum": 68942,"count": 908,"min": 36,"max": 94,"sumsqr": 5337861},  // 11
  {"sum": 68798,"count": 908,"min": 40,"max": 95,"sumsqr": 5315192},  // 12
  {"sum": 68595,"count": 908,"min": 39,"max": 94,"sumsqr": 5272344},  // 13
  {"sum": 68412,"count": 908,"min": 40,"max": 94,"sumsqr": 5238199},  // 14
  {"sum": 68981,"count": 908,"min": 37,"max": 95,"sumsqr": 5346677},  // 15
  {"sum": 71797,"count": 908,"min": 52,"max": 94,"sumsqr": 5730957},  // 16
  {"sum": 74132,"count": 908,"min": 56,"max": 96,"sumsqr": 6102577},  // 17
  {"sum": 74142,"count": 908,"min": 56,"max": 97,"sumsqr": 6107676}   // 18
]
}]}      
        \end{minted}
  \end{mdframed}
  \caption[Aggregated output from \textit{\_stats}-function]{\textbf{Figure \ref{fig-stats-output}: Aggregated output from \textit{\_stats}-function}}
  \label{fig-stats-output}
\end{figure}


\subsection{Presentation}
With reference to the reduced (aggregated benchmarks) output shown in Figure \ref{fig-stats-output}, variance and standard deviation are worked out according to Equation \ref{eq:variance-with-fields}, with standard deviation being the square root of variance. The computational variation of the sample variance formula is used.
\begin{align}
    (\sigma_{\overline{x}})^{2} =  \frac{sumsqr - \frac{sum^2}{count}}{count - 1}\label{eq:variance-with-fields}
\end{align}
Variance and standard deviation are calculated directly from the reduced index output during data retrieval using a CouchDB list function, with the results output in tabular form, as shown in Table \ref{tbl-variance-benchmarks} (page \pageref{tbl-variance-benchmarks}).