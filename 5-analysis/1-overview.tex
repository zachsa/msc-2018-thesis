\label{chapter-analysis}
CouchDB's MapReduce implementation is limiting in terms of performing \textit{joins} and \textit{selections} across multiple entities since index key:values pairs are derived from single documents. Each document is processed by MapReduce in isolation, the output of which is added to the index and no shared state between documents is available. In other words, map function execution is idempotent by design. Shared state between documents and other documents/databases during MapReduce calculation would violate this principle as well as pose a significant security risk. The MapReduce engine is pure JavaScript by design - meaning no IO to either a file system or to a network is possible. Although many JavaScript implementation provide APIs that allow IO, such features are restricted for the couchjs engine \cite{slack28Feb}.

Selection in terms of filtering documents can be performed during map function execution but requires hard-coded filter logic into the map function; selection cannot be dynamic in that documents cannot be filtered based on values in other documents. For the same reason, joins cannot be performed during map function execution. Although from a logical perspective joins can be accomplished during reduce function execution, they should not be used for this (performance will deteriorate extremely as database size increases). Practically speaking, neither joins nor selection can be achieved when indexing CouchDB databases. Since indices are the means by which data is retrieved from CouchDB it is fair to say that neither joins nor selections can be performed in CouchDB at all.

With CouchDB's means of data retrieval seemingly crippled compared to SQL, it is necessary to keep in mind CouchDB's benefits (as discussed previously) compared to such databases to stay in good humor! In conjunction with external processes such as ETL and index extraction \textit{selections} and \textit{joins} can be implemented with relative ease.

Working with relational data in CouchDB therefore requires implementation across the entire software stack. That is, ETL processes, indexing and data retrieval processes should all be geared towards working with relational data if that is the requirement. As such, in terms of retrieving processed datasets useful for analysis, for each analysis the nETL and CouchDB applications are incorporated into an workflow represented by the following steps:

\begin{itemize}
    \item CSVs are parsed, rows filtered (selection) and data loaded into CouchDB by the nETL application
    \item An index is created from the CouchDB database via MapReduce
    \item Data is retrieved directly from the index file using a List function, during which time any required joining and statistical calculations are performed
\end{itemize}

In terms of the MapReduce tasks, only built-in reduce functions are used; these functions (\_sum, \_count and \_stats) are implemented within the main Erlang process. This offers a performance boost over custom reduce functions since the IO transfer cost between the Erlang process and the view engine (couchjs.exe by default) is avoided. Working on a Windows machine, the IO cost is apparently exaggerated due to the difference between Unix-based and Windows kernel implementations \cite{slack1Nov}.

Metrics of the different components of the system are recorded and are shown in Table \ref{tbl-metadata}. The metrics include running time of nETL tasks, a summary of the data processed by nETL, CouchDB indexing times, and database/index storage footprints.

\begin{table}[H]
    \begin{threeparttable}
        \textbf{Table \ref{tbl-metadata}}\par\medskip\par\medskip
        \caption[Software performance analysis]{Running time analysis of \textit{nETL} tasks and CouchDB MapReduce indexing}
        \label{tbl-metadata}
        \begin{tabularx}{\textwidth}{>{\hsize=1.6\hsize}X>{\hsize=0.8\hsize}Y>{\hsize=0.8\hsize}Y>{\hsize=0.8\hsize}Y}
            \toprule
            \mC{c}{}                                                & \mC{c}{Variance}                 & \mC{c}{Grades}                    & \mC{c}{Events}                    \\
            \midrule
            Admissions lines extracted                              & 12 219                           & 12 219                            & 12 219                            \\
            Admissions lines loaded                                 & 1 381                            & 1 381                             & 595                               \\
            Admissions task time (sec)                              & 1.588\tnote{\textsuperscript{5}} & 2.712\tnote{\textsuperscript{6}}  & 4.881\tnote{\textsuperscript{8}}  \\
            \midrule
            Grades lines extracted                                  &                                  & 513 872                           & 513 872                           \\
            Grades lines loaded                                     &                                  & 1 891                             & 738                               \\
            Grades task time (sec)\tnote{\textsuperscript{1}}       &                                  & 50.993\tnote{\textsuperscript{7}} & 95.796\tnote{\textsuperscript{9}} \\
            \midrule
            Events lines extracted                                  &                                  &                                   & 44 420 508                        \\
            Events lines loaded                                     &                                  &                                   & -                                 \\
            Events task time (sec)\tnote{\textsuperscript{1}}       &                                  &                                   & -                                 \\
            \midrule
            CouchDB footprint (MB)                                  & 0.5                              & 1                                 & -                                 \\
            Index size (MB)                                         & -                                & -                                 & -                                 \\
            Index calculation time (sec)\tnote{\textsuperscript{3}} & -                                & -                                 & -                                 \\
            \bottomrule
        \end{tabularx}
        \scriptsize
        \begin{tablenotes}
            \item[\textsuperscript{3}]CouchDB views are calculated per shard. By default a database contains 8 shards (even in single node mode). The log file shows start and end times of view calculations for each shard, the time is taken as time the first shard starts indexing, to the time the last shard stops indexing
            \item[\textsuperscript{5}]$(1.502, 1.745, 1.518)$
            \item[\textsuperscript{6}]$(3.045, 2.787, 2.305)$
            \item[\textsuperscript{7}]$(52.899, 51.399, 48.681)$
            \item[\textsuperscript{8}]$(5.313, 3.548, 5.782)$
            \item[\textsuperscript{9}]$(94.353, 92.902, 100.134)$
        \end{tablenotes}
    \end{threeparttable}
\end{table}

\section{Joining Data}
The three datasets used in this project require joining on the StudentID field. This field appears once for every student in the admissions data, several times for each student in the grades data and up to thousands of times per student in the events data. Both the grade and event data contain a field for year - i.e. the year in which a grade was obtained or the year in which an event was registered; these two datasets require joining on both the StudentID and Year fields. Admissions data is only collected once per student enrollment, so is associated with grade and event data only by StudentID. To describe the join in terms of SQL, an inner-join is performed on the grades/events data, and the resultant dataset is outer-joined to admissions data: \begin{spreadlines}{15pt}
    \begin{gather*}
        (grades \bowtie events) \leftouterjoin admissions
    \end{gather*}
\end{spreadlines}

An initial attempt was to perform a join during index creation via MapReduce; this can be achieved via configuring the map function to output identical keys across the three entities for rows that require joining, and emit a tuple of 11 values for each key; index position in the tuple indicates the value-output of specific entities. During map function execution all 11 values are instantiated as 0 (a falsy value). Each index (or group of indices) of the output tuple is reserved for a specific entity type's output and adjusted appropriately during map function execution depending on the type of document the map function is applied to (the map function executes once for every document in the database):

\begin{minted}[linenos=false]{text}
[
   0,                          # i = 0: a course % grade or 0
   0, 0, 0, 0, 0, 0, 0, 0,     # 0 < i < 8: admission grade %s
   0, 0                        # 8 < i > 11: event count for semester 1/2
]
\end{minted}

Depending on the value of the `type\_' attribute of the document being processed by the map function, values at relevant indices of the tuple for that particular document type are altered to represent the info in the document that should be output. If the document being processed by the map function is of \textit{type\_} ‘grade’, then the map function emits a tuple with a value at \mintinline{text}{i = 0} and 0s for all other indices: \[[\%, 0, 0, 0, 0, 0, 0, 0, 0, 0, 0]\]

If the document is of \textit{type\_} `admission', the map function emits a tuple with values at \mintinline{text}{0 < i < 8}: \[[0, \%, \%, \%, \%, \%, \%, \%, \%, 0, 0]\]

If the document is of \textit{type\_} ‘event’, the map function emits a tuple with values at \mintinline{text}{8 < i > 11}: \[[0, 0, 0, 0, 0, 0, 0, 0, s1EventCount, s2EventCount]\]

The reduce function is passed key:value pairs, where unique keys are associated with a list of the tuples; together, these tuples represent information of all three entities, and since they are all passed to a single reduce function a join is achievable (the reduce function is effectively able to process information from all three entities for a single key). Structuring tuple output by reserving indices for specific data representations allows for preserving entity information in the resultant index in addition to document specific information. This information is preserved even across reduction when only values (no keys) of the index are preserved. The key in this case is compound, and has the form:

\[[StudentId,CourseCode,Year]\]

But since the events data doesn't include a course code, and the admissions data doesn't include course code or a course year, to allow for grades data to be joined to admissions data on course code and course year (in addition to student number), each admissions document needs to be output for every possible combination of course code and course year per student. That is the same with the events documents - a tuple needs to be output for each events document for every possible course that a student took in a given year.

In terms of performance this approach is disastrous. To analyze 40 courses taken over 3 years, each admissions document needs to be emitted for a student $40 x 3 = 120$ times so that the key of the Grade document [StudentID, Course, Year] can always be joined to admissions document. Likewise, Each events document (which has year but not course information) needs to be emitted 40 times - once for each course a join could potentially be performed on. Considering the large number of events documents, this is impracticable.

Instead, it is necessary to utilize CouchDB's data-structures (B+trees) as a means of retrieving sorted data - because joins can be performed on sorted data quite easily. Figure \ref{fig-sorted-index} (A) shows key output for all three entities in the form: [StudentID, Course, Year]. Documents that don't have properties for these fields emit 0 instead; resulting in a predictable key format of for each entity that is ordered.

Although many events documents are outputted by the map function per student, using reduction results in grouping all these documents by common key. In other words, only a single document that is an aggregation of all events documents will be stored as reduced output in the view. So, for any student number, scanning the index iteratively produces first an admissions document, then a single (aggregated) events document, then a single grade document for each course that student enrolled in - this is shown in \ref{fig-sorted-index} (B). Retrieving reduced index output requires querying the index and specifying that reduce is true. In addition, it is necessary to specify that grouping is true. This results in reduction being performed on grouped keys instead of retrieving an aggregation of the entire index (which useful, for example, when calculating variance).

\begin{figure}[H]
    \centering
    \begin{mdframed}
        \centering
        \begin{minted}[linenos=false]{text}
/* (A) reduce = false */
[<ID>, ‘0’, 1]: [0, b1, b2, b3, b4, b5, b6, b7, b8, 0, 0]
[<ID>, ‘0’, <Year>]: [0, 0, 0, 0, 0, 0, 0, 0, 0, 1, 0]
[<ID>, ‘0’, <Year>]: [0, 0, 0, 0, 0, 0, 0, 0, 0, 1, 0]
[<ID>, ‘0’, <Year>]: [0, 0, 0, 0, 0, 0, 0, 0, 0, 0, 1]
[<ID>, ‘0’, <Year>]: [0, 0, 0, 0, 0, 0, 0, 0, 0, 0, 1]
[<ID>, ‘0’, <Year>]: [0, 0, 0, 0, 0, 0, 0, 0, 0, 0, 1]
[<ID>, ‘0’, <Year>]: [0, 0, 0, 0, 0, 0, 0, 0, 0, 1, 0]
[<ID>, ‘CSC1015F’, <Year>]: [98, 0, 0, 0, 0, 0, 0, 0, 0, 0, 0]
[<ID>, ‘MAM100F, <Year>]: [94, 0, 0, 0, 0, 0, 0, 0, 0, 0, 0]

/* (B) reduce = true & group = true */
[<ID>, ‘0’, 1]: [0, b1, b2, b3, b4, b5, b6, b7, b8, 0, 0]
[<ID>, ‘0’, <Year>]: [0, 0, 0, 0, 0, 0, 0, 0, 0, 3, 3]
[<ID>, ‘CSC1015F’, <Year>]: [98, 0, 0, 0, 0, 0, 0, 0, 0, 0, 0]
[<ID>, ‘MAM100F, <Year>]: [94, 0, 0, 0, 0, 0, 0, 0, 0, 0, 0]
        \end{minted}
    \end{mdframed}
    \caption[Index structure]{\textbf{Figure \ref{fig-sorted-index}: Index structure}}
    \label{fig-sorted-index}
\end{figure}


Because output is systematically processed one student at a time, with grades, events and admissions data iteratively process in a predictable order per student, joins can be achieved by holding documents for a particular StudentID in memory and processing grades, events, admissions rows for that StudentID. When a new StudentID is encountered, a joined row is output, memory is flushed and a new joined row for the new studentID is started. This results in a much more efficient way of aggregating/joining different types of documents for a given set of keys.

With this approach,MapReduce fills the important role of aggregating entities prior to joining them (as discussed in 3-way join section), and then outputting a sorted index - but the join is possible because the index is sorted, not because MapReduce is performed. In fact, a CouchDB database is also structured as a B+tree and so is also sorted (according to the \_id field). So it is possible to join documents directly on retrieval from the main database rather than creating an index first. However there are three problems with retrieving data directly from the main database file and skipping index creation:

\begin{itemize}
    \item CouchDB database files are sorted by the ``\_id'' field, which when unspecified on document insert is initialized as a UUID. Using UUIDs as unique document identifiers allow for distributed systems in which cluster nodes can operate independently of each other without the possibility of documents being created in separate nodes with conflicting IDs. Even though not required by this project, such best practices are followed. A B+tree sorted by a UUID is not useful for document retrieval, and as such, views are required of the underlying data store for any kind of index-based querying
    \item List functions are invoked via an HTTP GET request with the requirement of specifying a view within the URI. List functions are convenient for usage in this project as they facilitate ordered, iterative, range-based access (meaning that datum can be accessed sequentially, in isolation and in reliable order)
    \item When aggregations of specific entities are required, retrieving data directly from these indexes is vastly easier than having to aggregate during data retrieval. Without a reduce function, additional logic would be required during retrieval to perform such aggregations. As aggregation logic becomes more complicated the difficulty of such direct data retrieval increases and the benefit of using indexes increases as a result. Also, performing aggregation during the indexing phase instead of data retrieval greatly improves performance of data retrieval
\end{itemize}

\subsection{Document joins using nETL}


\section{Selecting Data}
Performing joins via iterating over ordered indices is quite efficient in terms of memory usage. No matter the size of datasets being processed, memory usage will always be fairly low. An increase in size of datasets will simply result in more processing time. Selections performed during the MapReduce stage are therefore completely feasible regardless of the size of the dataset being queried. However, in terms of time, selections are expensive when performed as part of the MapReduce process; the events data as over 40 million records of which a very small number are selected for this analysis. To perform a selection on the events data during MapReduce requires iterating through through all events records - which is expensive in terms of time.

Additionally, it is impossible to select records by a field in which values are specified dynamically. This means that selecting data during MapReduce requires an additional step in data-analysis in which a list of values to use for filtering are determined either via using an additional CouchDB view, Microsoft Excel, a SQL database, etc. CouchDB's MapReduce engine doesn't allow any database, file, etc. interactions by design, and although it's possible to swap out the default MapReduce engine with one that does allow this, this is not recommended. An experiment to implement the couchjs engine using Google's V8 engine (i.e. using node.js) is available on online \cite{v8couchjs}, which is misleading since that is the JavaScript implementation that \textit{node.js} uses. Replacing the default couchjs engine with this one, it would seem as if network requests from map function execution should be possible (making dynamic filtering possible). However, this experiment was abandoned in 2013 precisely because it is too difficult to disable the numerous APIs that the V8 engine makes available to runtime JavaScript code. The CouchDB specification is that the MapReduce execution environment is required to be isolated in terms of interactions with the database or other instances of MapReduce function execution for security and to maintain integrity of the data layer \cite{slack28Feb}.

In essence, dynamic filtering is actually a join - in that a join is required to obtain the values on which filtering is done. But working within the constraints of CouchDB, such a join is treated as a selection. Such an approach is conceptually similar to how joins are performed on distributed datasets, where a list of keys on which data joins are required is acquired prior to performing database operations so as to minimize network transport costs \cite{sonia2018}. As such, for the purposes of this project, dynamic filtering is treated as a selection operation and not as a join.

To allow for dynamic filtering, and to improved index-calculation efficiency, selection is implemented during the ETL phase of the analysis in nETL. A dynamic filter is used as part of the transformations pipeline, in which a URI to a CouchDB view is specified where a list of values to perform dynamic filtering on is made available. The result of this view is used to filter (i.e. perform selection) on data extracted from the CSVs, with only relevant rows from the CSV added to the database. This approach requires configuring an additional index prior to loading data into the database; a CouchDB database \textit{grades} is created and indexed using just a map function, in which a list of objects is output. Each output contains a \textit{value} attribute which specified fields that should dynamic filtering should be performed on, as well as the values that should be whitelisted by this dynamic filtering process. An example of index output that allows for dynamic filtering on the \textit{anonIDnew} field for students who attended the CSC1015F course in 2016 is shown in Figure \ref{fig-dynamic-filter-example} on an HTTP request to the CouchDB endpoint: \url{http://localhost:5984/grades/_design/grades/_view/anonIDnew-2016}.

\begin{figure}[H]
    \centering
    \begin{mdframed}
        \centering
        \begin{minted}[linenos=false]{text}
{"total_rows":738,"offset":0,"rows":[
    {
        "id":"0eaa693397d0f30826ad5554e40896e9",
        "key":"0eaa693397d0f30826ad5554e40896e9",
        "value":{"anonIDnew":2784248}
    },
    {
        "id":"0eaa693397d0f30826ad5554e4926c04",
        "key":"0eaa693397d0f30826ad5554e4926c04",
        "value":{"anonIDnew":2800748}
    },
    {
        "id":"0eaa693397d0f30826ad5554e4a72f38",
        "key":"0eaa693397d0f30826ad5554e4a72f38",
        "value":{"anonIDnew":2802150}
    },
    {
        "id":"1e2ce41c2c142cce762589a1b2023c9e",
        "key":"1e2ce41c2c142cce762589a1b2023c9e",
        "value":{"anonIDnew":3084806}
    },
    {
        "id":"1e2ce41c2c142cce762589a1b20278f7",
        "key":"1e2ce41c2c142cce762589a1b20278f7",
        "value":{"anonIDnew":3084814}
    },
    {
        "id":"1e2ce41c2c142cce762589a1b202cae8",
        "key":"1e2ce41c2c142cce762589a1b202cae8",
        "value":{"anonIDnew":3084816}
    },
    {
        "id":"1e2ce41c2c142cce762589a1b2045c8b",
        "key":"1e2ce41c2c142cce762589a1b2045c8b",
        "value":{"anonIDnew":3084890}
    },
    etc...
]}        
        \end{minted}
    \end{mdframed}
    \caption[Dynamic Filter Index Output]{\textbf{Figure \ref{fig-dynamic-filter-example}: Dynamic Filter Index Output}}
    \label{fig-dynamic-filter-example}
\end{figure}


\section{Stats calculations}
