\section{Benchmark Variance}
Variance ($(\sigma_{\overline{x}})^{2}$) and standard deviation ($\sigma_{\overline{x}}$) are worked out for each of the benchmarks used from the admissions data using via a similar process to other analysis. A single nETL task loads a database with Benchmarks data for all students who attended CSC1015F in either 2014, 2015, or 2016 (or attended the course multiple times). The JSON used for configuring nETL to load data from \textit{Benchmarks (2014 - 2016).csv} into CouchDB is included in Appendix \ref{netl-task-6-config}, which is identical to the \textit{Task 1} configuration, except that a different database is used.

\subsection{Map Function}
Data in CouchDB is mapped to an index ordered by a single value (\textit{Student ID}), and not a compound key as used in previous analyses. The index consists as a list of \mintinline{text}{[StudentID:[b1, b2, b3, b4, etc.]]} key:value pairs where \mintinline{text}{b} stands for ``benchmark''. All the benchmarks shown in Table \ref{tbl-correlation-variance} are included in the index either as a direct emission of values from the \textit{benchmarks} documents, or as aggregations (averages) of individual benchmark scores calculated by the JavaScript runtime environment.

Because all numbers in JavaScript are 64-bit floating-point (following the IEEE 754 standard \cite{floatingPoint}), working with rational numbers with decimal points results in peculiarities compared to expectations formed from working with a base-10-framed mindset - for example, the sum: $0.1 + 0.2 = 0.30000000000000004$. Decimals such as $0.1$ and $0.2$ cannot be accurately represented in binary format within 64-bit address space (or any amount of memory). As such, rounding errors occur. Quantifying the margin for such errors and handling these cases correctly is difficult \cite{Goldberg1991}, so an open-source library (\textit{decimal.js} \cite{decimaljs}) is used to handle integer summing and division in the Map function since many of the values are decimals. (CouchDB supports the commonJS specification for module loading of JavaScript libraries in the context of Map function execution \cite{commonJsMapFn}). The map function code is included in Appendix \ref{variance-map-function}.

Map output is reduced using the \_stats function, but with grouping by key set to false. This results in a reduction of values output by the index as if they all shared a common key - in fact, in analyzing the variance of this dataset, the key isn't required at all and it would be a reasonable option to output \mintinline{text}{null} in the place of StudentID (except that this is quite difficult to test correctness for since it is then impossible to compare Map output to CSV data). In other works, a single row of reduced output is retrieved from the index via the address \url{http://127.0.0.1:5984/variance/_design/variance/_view/variance.csv?reduce=true&group=false} as shown in Figure \ref{fig-variance-reduce-output}. This row consists of a tuple of objects. Each object comprises a set of numerical descriptions of a single admission benchmark - that is, each admissions benchmark is described in terms of the sum of all students scores for that particular admissions benchmark, the count of how many students are included in the sum, the worst (min) and best (max) scores achieved for each test (or average of tests) by a student, and the sum of squares of each students scores.

\begin{figure}[H]
    \centering
    \begin{mdframed}
        \centering
        \begin{minted}{text}
{
    "rows": [
    {
        "key": null,
        "value": [
        {"sum": 71751,"count": 908,"min": 50,"max": 97,"sumsqr": 5720599},
        {"sum": 74174,"count": 908,"min": 48,"max": 100,"sumsqr": 6151326},
        {"sum": 78802,"count": 908,"min": 63,"max": 100,"sumsqr": 6900682},
        {"sum": 67207,"count": 908,"min": 33,"max": 94,"sumsqr": 5057191},
        {"sum": 69713,"count": 908,"min": 27,"max": 98,"sumsqr": 5512393},
        {"sum": 69136,"count": 908,"min": 29,"max": 98,"sumsqr": 5439872},
        {"sum": 74908.99,"count": 908,"min": 61,"max": 97.67,"sumsqr": 6227518.0489},
        {"sum": 75882.25,"count": 908,"min": 61.5,"max": 98,"sumsqr": 6389865.9375},
        {"sum": 75540.59999999999,"count": 908,"min": 60.6,"max": 98.2,"sumsqr": 6338060.76},
        {"sum": 68685.20999999999,"count": 908,"min": 39.33,"max": 94,"sumsqr": 5288410.2247},
        {"sum": 68315.75,"count": 908,"min": 40.25,"max": 94,"sumsqr": 5221242.8125},
        {"sum": 68942.25,"count": 908,"min": 36.25,"max": 94.5,"sumsqr": 5337860.9375},
        {"sum": 68798.0,"count": 908,"min": 40,"max": 94.75,"sumsqr": 5315191.625},
        {"sum": 68595.2,"count": 908,"min": 38.8,"max": 94.4,"sumsqr": 5272344.32},
        {"sum": 68412.20000000001,"count": 908,"min": 40.4,"max": 93.9,"sumsqr": 5238198.504999999},
        {"sum": 68981.0,"count": 908,"min": 37.4,"max": 95,"sumsqr": 5346677.08},
        {"sum": 71797.14,"count": 908,"min": 52,"max": 94.16,"sumsqr": 5730956.546399999},
        {"sum": 74132.19,"count": 908,"min": 55.67,"max": 96.11,"sumsqr": 6102577.2419},
        {"sum": 74142.46,"count": 908,"min": 56,"max": 96.83,"sumsqr": 6107676.100800001
        }]
    }]
}      
        \end{minted}
    \end{mdframed}
    \caption[Analysis 3 MapReduce Output]{\textbf{Figure \ref{fig-variance-reduce-output}: Analysis 3 MapReduce Output}}
    \label{fig-variance-reduce-output}
\end{figure}


\subsection{Map Function}
\textit{Variance} and \textit{standard deviation} are calculated directly from the reduced index output by the \textit{List} function (Appendix \ref{variance-list-function}) via these the formula (with references to the fields in Figure \ref{fig-variance-reduce-output}):

\begin{spreadlines}{15pt}
    \begin{gather*}
        \intertext{\textit{Variance:}}
        (\sigma_{\overline{x}})^{2} = \frac{sumsqr}{count-1} - (\frac{sum}{count})^{2}\\
        \intertext{\textit{Standard Deviation:}}
        \sigma_{\overline{x}} = \sqrt{\frac{sumsqr}{count-1} - (\frac{sum}{count})^{2}}
    \end{gather*}
\end{spreadlines}

Without the need for CSV output, the List function is configured to display results as HTML - meaning that navigating to the List function URI in a browser returns a web page that contains a table listing benchmarks, and the variance and standard deviation of each benchmark across all students.