\section{Future Work}
CouchDB views are optimized when using built-in reduce functions, with custom reduce functions performing most poorly on Windows machines. As this project was completed on the Windows OS, the analysis on how best to aggregate the different entities was confined to using just the built-in reduce functions.

It may be worth investigating using custom reduce functions for MapReduce jobs. This would allow for much more varied map function output structure and provide a means of performing joins during index reduction instead of on index scanning. Reduce function calculation represents a small percentage of computer resource usage overall \cite{slack1Nov}, and different environment configurations provide different performance contexts in terms of what overheads are/are not acceptable. And in any case, a system that utilizes CouchDB is likely to be based on a cluster of Linux machines rather than a single Windows machine.

Feasibility rather than performance was the focus of this work; if implemented on a network cluster of computers, performance and scalability could be investigated. It would be worth investigating the benefit of clustering CouchDB across many separate nodes with varying configurations for replication and data redundancy (node copies). CouchDB is configured to use 8 shards by default (even on a single server) and processes data in parallel (across 2 physical cores for this project), it is likely that deploying shards to separate servers would greatly increase performance and decrease indexing time. CouchDB disperses documents evenly across shards in a random fashion, suggesting that the workload of indexing the documents would be distributed evenly across all the shards of the database \cite{slack7Nov}. It is likely sharding would benefit large datasets, but not smaller datasets since the cost of network interactions would also increase if shards were distributed across separate nodes.

There is further scope to test implementing relational operations in a similar software stack that were not implemented in this project. These are: \textit{division}, \textit{set union}, and \textit{set difference}.

There is also scope to develop a GUI for the \textit{nETL} application.