\section{Slack conversation 2}
\label{appendix:slack2}
zach [3:53 PM]
at the risk of sounding like a repetitive novice... what is wrong with indexes that get bigger and bigger? Is it best to avoid them because a) there is almost always a better way of doing something? or b) they have some other effect on the CouchDB process (other than whatever overhead is required to use a larger-than-requried index)

rnewson (IRC) APP [3:57 PM]
'bigger and bigger'?

jan [4:03 PM]
@zach not sure what you mean, indexes are expected to grow with more documents

zach [4:22 PM]
writing a reduce function I sometimes get a warning that reduce output is not shrinking fast enough - in this case i can either rewrite my reduce function or turn off the setting. As you mentioned previously @jan, option 1 (rewriting) is definitely the way to go. But what is the reason for this? Is it because a reduce function that doesn't shrink output requires continuous balancing of the B+ tree which means poor performance? (i don't have a background in data structures). what are the theoretical problems of in a reduce function like this: function(keys, values, rereduce){return values (with code to do the same in the rereduce)} if there was an unlimited amount of hardrive space available.

[4:23]
I know this is a 'relational' approach.. and that there are better options. but I'm still interested in why

jan [4:49 PM]
that reduce function will copy all values from the leaves of the b-tree into the root node, and all intermediate nodes get their share of leaves copied in as well. You’re effectively stacking a reverse tree on top of the actual tree, approaching $ O(n^2) $ or worse performance and disk use.

zach [4:51 PM]
ah. thank you

jan [4:55 PM]
I.e. it subverts everything an index is meant to do

rnewson (IRC) APP [5:12 PM]
I'm curious to know what the reduce function is

jan (IRC) APP [5:13 PM]
" a reduce function like this: function(keys, values, rereduce){return values (with code to do the same in the rereduce)}"

afinne [5:33 PM]
zach, just to double check: you are aware that a reduce function is not mandatory?

zach [5:42 PM]
Hi @afinne - yes. why do you ask? (I don't think I was able to group without the reduce function)

Wohali (IRC) APP [5:43 PM]
just use a built-in reduce like \_sum

[5:43]
if all you want is grouping

zach [5:43 PM]
if I say group=true and reduce = false I get an error

rnewson (IRC) APP [5:44 PM]
yes, you need a reduce if you want to group.

[5:44]
though 'return null' is sufficient

zach [5:45 PM]
would that guarantee all values returned for a particular key?

[5:48]
I re-wrote the reduce function to aggregate. though I must have done something wrong since 5 hours later it's still calculating

...

[5:49]
and the database is only about 3.5GB

rnewson (IRC) APP [5:49 PM]
well, you'll @fabsolute least get compound rounding errors from the divide

    [5:49]
but the whole thing looks wrong tbh

    [5:49]
what does the map look like? what does a doc look like?

[5:51]
hm, I bet you could get the built-in's to do this for you

    [5:51]
if you emitted an array as your value in map, then \_sum will sum each item, etc

    [5:52]
for an average, you should calculate both values and then divide @fabsolute the client, you can't do it as you go

    [5:52]
and a note of caution, if any of your values happen not to be numbers, you'll end up doing string concatenation and blowing things up that way

    [5:53]
so I suggest typeof(number) checks

zach [5:53 PM]
thanks. the map function looks like this: (url)

...

zach [6:03 PM]
Thanks for pointing out the rounding and average points @rnewson. I think I understand how I could use the \_sum, but I don't know how I would include the average

rnewson (IRC) APP [6:04 PM]
you couldn't, but you could include the numbers you need to calculate the average from the result of the \_view request

    [6:05]
that is, don't bother calculating r.gAvgS1 or r.gAvgS2 in the view

    [6:05]
just calculate it from the other two fields from the view response

zach [6:07 PM]
oh. i guess i could just sum the grades percent and divide by total grades at the end.

rnewson (IRC) APP [6:07 PM]
aye

    [6:08]
the built-in \_stats endpoint would give you the sum and count of the values, which you could then use to get mean average

    [6:08]
afk