\subsection{CouchDB}

\subsubsection{Setup script}
\label{appendix:couch-setup}
\begin{minted}{sh}
# Set hostname of server
hostname <hostname>; rm /etc/hostname; touch /etc/hostname; echo <hostname> >> /etc/hostname; chmod 466 /etc/hostname;

## Install basic tooling 
# GCC collection (GNU make and GNU compiler tools)
apt-get update
apt-get install build-essential -y

# Update openssl to 1.0.2l
cd /usr/src
wget https://www.openssl.org/source/openssl-1.0.2l.tar.gz
tar -zxf openssl-1.0.2l.tar.gz
cd openssl-1.0.2l
./config
make
make test
make install
mv /usr/bin/openssl /root/
ln -s /usr/local/ssl/bin/openssl /usr/bin/openssl

# Python
apt-get update
apt-get install python -y

# libcurl
apt-get update
apt-get install libcurl4-openssl-dev -y

# ICU
apt-get update
apt-get install libicu-dev -y

# Pre-seed debconf to answer CouchDB installation wizard automatically
debconf-set-selections <<< 'couchdb couchdb/bindaddress string 0.0.0.0'
debconf-set-selections <<< 'couchdb couchdb/cookie string monster'
debconf-set-selections <<< 'couchdb couchdb/mode string clustered'
debconf-set-selections <<< 'couchdb couchdb/nodename string couchdb@<hostname>'
debconf-set-selections <<< 'couchdb couchdb/adminpass password <password>'
debconf-set-selections <<< 'couchdb couchdb/adminpass_again password <password>'

# register CouchDB package with the server package manager and install
echo 'deb https://apache.bintray.com/couchdb-deb xenial main' | sudo tee -a /etc/apt/sources.list
curl -L https://couchdb.apache.org/repo/bintray-pubkey.asc | sudo apt-key add -
apt-get update
apt-get install couchdb -y

# And then once those commands have been run to setup all the CouchDB nodes, the CouchDB cluster can be configured using a few commands on any one node (temporarily referred to as the CoOrdinatingNodeHost)
# Run these two lines to add a node to the CouchDB cluster
curl -X POST -H \"Content-Type: application/json\" http://<username>:<password>@<CoOrdinatingNodeHost>:<port>/_cluster_setup -d '{\"action\": \"enable_cluster\", \"bind_address\":\"CoOrdinatingNodeHost\", \"username\": \"<username>\", \"password\":\"<password>\", \"port\": <port>, \"node_count\": \"<intented node count>\", \"remote_node\": \"<remote hostname>\", \"remote_current_user\": \"<username>\", \"remote_current_password\": \"<password>\" }'
curl -X POST -H \"Content-Type: application/json\" http://<username>:<password>@<CoOrdinatingNodeHost>:<port>/_cluster_setup -d '{\"action\": \"add_node\", \"host\":\"<remote hostname>\", \"port\": \"<port>\", \"username\": \"<username>\", \"password\":\"<password>\"}'

# Finalize the cluster setup
curl -X POST -H \"Content-Type: application/json\" http://<username>:<password>@<CoOrdinatingNodeHost>:<port>/_cluster_setup -d '{\"action\": \"finish_cluster\"}'
\end{minted}