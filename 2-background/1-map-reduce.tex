\section{MapReduce}
In response to dealing with huge amounts of data on a daily basis, authors at Google (Jeffrey Dean and Sanjay Ghemawat) outlined the \textit{MapReduce} framework as a means of abstracting complications associated with distributed computing such as data distribution, fault tolerance, load balancing and how to parallelize processing \cite{Dean:2008}. \textit{MapReduce} provides programmers a conceptually-simple interface for specifying dispersed data computations succinctly and hides implementation details.

The framework relies on an astoundingly simple programming model described by \cite{Dean:2008} as a computation that takes a set of input \textit{key:value} pairs and produces a set of output \textit{key:value} pairs via the following 3 steps:

\begin{itemize}
    \item A \textit{mapping} stage in which distributed \textit{key:value} pairs are produced from input data as described by a user-defined \textit{map} function
    \item A \textit{grouping} stage where distributed \textit{key:value} output from the mapping stage is collected to common \textit{keys} - i.e. \textit{key:[value, value, value]} datasets
    \item And a \textit{reduce} stage where \textit{values} per key \textit{key} are processed as described by a user-defined \textit{reduce} function
\end{itemize}

Due to the distributed and isolated nature of \textit{map} and \textit{reduce} tasks, \textit{MapReduce} as an idea is greatly fault tolerant (fault tolerance is implemented via reexecution), which has in turn resulted in the ``New Software Stack'' \cite{mining2011}: large scale computing clusters built on commodity (cheap) hardware and software that computes in parallel. Such an approach to building/maintaining systems represents the potential to process ever-greater amounts of data at ever cheaper rates. This has spurred information explosion across all manner of software applications.

Starting with the development of the \textit{Hadoop} framework as an open-source alternative to Google's proprietary file system and \textit{MapReduce} framework, \textit{MapReduce} implementations have become mainstream. Companies that make use of this programming model including Yahoo, Facebook, Amazon, and many more \cite{chandar2010}. For example, the Apache Foundation maintains a list of companies that use the Hadoop framework \cite{hadoopPower:2017}.

Use-cases for systems utilizing \textit{MapReduce} greatly overlap a business domain that has long been cornered by RDBMSs. As such, much research exists on leveraging \textit{MapReduce} as a means of implementing operations traditionally associated with RDBMSs - specifically those of \textit{relational-algebra} \cite{mining2011,chandar2010}. The full spectrum of relational operations: \textit{selection}, \textit{projection}, \textit{union}, \textit{intersection}, \textit{difference}, \textit{joins} (excepting non-equi joins, which cannot be implemented via MapReduce), \textit{grouping} and \textit{aggregation} are translatable to \textit{MapReduce} \cite{mining2011}. \textit{MapReduce} allows for implementing \textit{joins}, including both \textit{Two-way} and \textit{Multi-way}, via a variety of algorithms. A masters thesis from the University of Edinburgh demonstrates joining in Hadoop via \textit{Map-Side joins}, \textit{Reduce-Side One-Shot joins},\textit{Reduce-Side Cascade joins} and more \cite{chandar2010}.