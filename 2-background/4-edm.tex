\section{Educational Data Mining}
Falling under the umbrella category of \textit{Educational Data Mining} (EDM), much work has been done with the intent of modeling student performance as dependent on certain markers such as attendance, assignment and test grades, high school marks, demographic data, etc. Different means of model generation have been discussed by the EDM community such as predictive analysis via decision tree generation \cite{Qasem20016,Balestra2017,casper2017,Dimitris,zebun2005,Mierle:2005} with varying results. Many other models have been applied within the field of EDM as discussed in a review of EDM up to 2009 \cite{bakerEdMiningSummary}.

Although these studies discuss analysis-frameworks at length, very little work has been done on the underlying data stores that form a basis of such analysis. As such, it appears that the work done so far looks to research feasible models in terms of accuracy and implementation rather than feasible means of implementing such mining techniques. Where \textit{Extraction, Transformation, and Loading} (ETL) processes have been discussed with regards to data preparation, SQL is used but without much discussion as to the implementation of the underlying relational data stores (\cite{Balestra2017,casper2017,Mierle:2005}). No attempts have been made to use newer and less conventional NoSQL stores such as CouchDB despite that benefit that such data stores provide in terms of the unique features that these newer data stores offer (\textit{schema-less} entites, easier scaling, lower costs of implementation and licensing, etc.).