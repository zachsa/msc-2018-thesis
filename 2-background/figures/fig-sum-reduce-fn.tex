\begin{figure}[H]
    \centering
    \begin{mdframed}[rightline=true,leftline=true]
        \begin{minted}{text}
/* A: Map output*/
{["key"]: 7} // (_id: x)
{["key"]: [3,1,3]} // (_id: y)
{["key2"]: [2,2,2]} // (_id: z)

/* B: Map output grouped by key */
{["key"]: [7,[3,1,3]]}
{["ke2"]: [[2,2,2]]}

/* C: Map output passed to _sum function as input */
reduce([key:id tuples], [values groupted by key], rereduce)
reduce([["key", "x"], ["key", "y"]], [7,[3,1,3]], false)
reduce([["key2", "z"]], [[2,2,2]], false)

/* D: Logical treatment of values argument (arg 2) during reduction */
{["key"]: [sum([7,3]), sum([0,1]),  sum([0,3])]}
{["key2"]: [sum([2]), sum([2]),  sum([2])]}

/* E: Reduce output (group = true) */
{
  ["key"]: [10, 1, 3],
  ["key2"]: [2,2,2]
}
    \end{minted}
    \end{mdframed}
    \caption[\_sum Reduce Function Logic]{\textbf{Figure \ref{fig-sum-reduce-fn}: Logical description of the \_sum Reduce function contract shown as a step-based series of transformations from input (map function output) to output}}
    \label{fig-sum-reduce-fn}
\end{figure}