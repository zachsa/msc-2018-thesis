\begin{figure}[H]
  \centering
  \begin{mdframed}[rightline=true,leftline=true]
    \begin{minted}{text}
/* A: Map output*/
{["key"]: [1,1,0]} // (_id: x)
{["key"]: [3,1,3]} // (_id: y)
{["key2"]: [2,2,2]} // (_id: z)

/* B: Map output grouped by key */
{["key"]: [[1,1,0],[3,1,3]]}
{["ke2"]: [[2,2,2]]}

/* C: Map output passed to _stats function as input */
reduce([key:id tuples], [values groupted by key], rereduce)
reduce([["key", "x"], ["key", "y"]], [[1,1,0],[3,1,3]], false)
reduce([["key2", "z"]], [[2,2,2]], false)

/* D: Logical treatment of values argument (arg 2) during reduction */
{["key"]: [aggregate([1,3]), aggregate([1,1]),  aggregate([0,3])]}
{["key2"]: [aggregate([2]), aggregate([2]),  aggregate([2])]}

/* E: Reduce output (group = true) */
{
  ["key"]: [
    {"sum":4,"count":2,"min":1,"max":3,"sumsqr":10},
    {"sum":2,"count":2,"min":1,"max":1,"sumsqr":2},
    {"sum":3,"count":2,"min":0,"max":3,"sumsqr":9}
  ],
  ["key2"]: [
    {"sum":2,"count":1,"min":2,"max":2,"sumsqr":4}
  ]
}
    \end{minted}
  \end{mdframed}
  \caption[\_stats Reduce Function Logic]{\textbf{Figure \ref{fig-stats-reduce-fn}: Logical description of the \_stats Reduce function contract shown as a step-based series of transformations from input (map function output) to output}}
  \label{fig-stats-reduce-fn}
\end{figure}