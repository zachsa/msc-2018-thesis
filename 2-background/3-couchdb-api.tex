\subsection{API}
\subsubsection{HTTP Interface}
All communication with CouchDB takes place via the HTTP protocol. The interface is resource orientated, and strives to be RESTful. This makes such interactions very clear since the HTTP endpoints used are logical and easy to remember - especially compared to clients used for interacting with relational databases. For example, the following endpoints (among many) are shown here, with the full API very well documented online at \cite{couch-api}.

\begin{itemize}
    \item \textit{GET} /dbName/:id (retrieves a document with the specified ID)
    \item \textit{PUT} /dbName/[:id] (inserts a document, optionally specifying an ID)
    \item \textit{POST} /\_bulk\_docs (inserts multiple documents - atomicity of batch insert is configurable but defaults to false)
    \item \textit{POST/GET} /\_all\_docs (Fetches multiple documents, specifying keys can be done in in the body of a post request)
\end{itemize}

\subsubsection{Design Documents}
CouchDB allows a user to specify several different types of functions that are executed on the server-side Erlang application via JSON - as a regular database document but with an ID of ``\_design/documentName''. Such documents are known as ``Design'' documents and are treated as special by the CouchDB application. A sample design document is included in the appendix (see \ref{couchdb-design-doc-sample}), showing the format in which JavaScript functions are included in CouchDB JSON documents (languages other than JavaScript can be specified as well). There are 6 types of functions that are executed on the server:

\begin{itemize}
    \item \textit{views}
    \item \textit{shows}
    \item \textit{lists}
    \item \textit{updates}
    \item \textit{filters}
    \item \textit{validations}
\end{itemize}

Views are B+tree indexes created by specifying a \textit{Map}-function and an optional \textit{Reduce}-function to be executed on every document in a database; Contracts of these functions are shown in the appendix (see \ref{couchdb-mapreduce-contracts}). Map functions iteratively take each document as an argument and output key:value pair(s), as specified by the Map function definition, that are incorporated into the B+tree. Reduce functions are evaluated taking Map-function output as input and returning \textit{reduced} results from that input and storing it as internal nodes on the B+tree.

Show functions act as a type of 'middleware', by allowing users to specify transformations on a single document requested from the database and returning that document to the user - for example, transforming a document from JSON representation to HTML representation for better display in a browser. List functions, similarly to Show functions, allow for transforming documents retrieved from CoucDB and presenting a different representation of those documents to the user, but work on sets of documents instead of single documents. For example, list functions can retrieve results from a view index. List functions invoked via an HTTP request to the CouchDB server take the name of a view as an argument incorporated into the URI (view parameters are specified as URL parameters). The List function makes a number of helper methods available to users - most notably the \texttt{getRow()} function which allows for iterating over the specified view, one key:value pair at a time (a single key:value pair is a `row' in this case). The \texttt{send()} function emits data during execution of the List function, but without executing the function context.

Updates allow for specifying document updates indirectly via the HTTP API (functionally equivalent to retrieving a document, updating it and then reinserting it). Filters, true to their name, allow users to specify filters that can be used during view retrieval, document retrieval and in replicating data between CouchDB databases. Validations allow users to specify rules on when databases contained documents can be updated and by whom.

\subsection{Views: Indexes via MapReduce}
Essentially, CouchDB provides users a simple-to-use HTTP interface that allows for fine-grained control over b+tree structures with no means of querying data directly. Instead, it allows for processing documents in database files via MapReduce to produce \textit{Views}. These files are separate from the database files and are structured as b+trees (as are the database files). In other words, CouchDB allows users to create b+tree indexes via MapReduce, and then to retrieve (view) projections, selections, aggregations, etc. of the database contents (documents).

In the current CouchDB release (as of January 2018), there is actually an additional means of querying CouchDB databases called `Mango'; a selection-based syntax inspired by MongoDB, but it is not enabled by default and will likely never cover the full functionality available via bespoke MapReduce functions according to an answer on the StackExchnge network that allegedly references a core developer \cite{Mango}. The Mango syntax still processes documents via MapReduce under the hood, but is usually faster than with JavaScript functions since the MapReduce engine can be executed directly within the Erlang process on the server.

\textit{Map} functions must be specified by a user and are always executed externally to the main Erlang process via marshalling between the main Erlang process and the indexing engine. \textit{Reduce} functions, however, can either by executed via the main Erlang process (by specifying a built-in reduce function) or externally by the indexing engine when specifying custom reduce functions. CouchDB ships with an executable "couchjs.exe" - a JavaScript query-server (indexing engine) - coupled with Mozilla's \textit{SpiderMonkey} runtime engine. This query engine is drop-in replaceable by alternative implementations in JavaScript or other languages if required. CouchDB spawns a single \textit{couchjs} process per shard (\cite{slack2Nov}, \cite{slack7Nov}) and the map index is calculated sequentially according to the order in which the database was changed.

Conceptually \textit{``fetching all documents of type x''} requires specifying an iteration through every document in the database and fetching documents with content indicative of \textit{``type x''}. View indices are built in CouchDB in this fashion; via iterating over every document in the database, and passing that document to the user-defined map function. A user writes code in this map function to evaluate each document, and emit key:value pairs depending on the content of the document. The map output is then grouped by key and passed to the reduce function for reduction.

Users can specify whether to retrieve reduced results, or to access unreduced map results (so a reduce function isn't actually required to produce a CouchDB view). However, when passing output from the map function to reducer, where map output is grouped by key, there is no guarantee that all map output for a particular key will be sent to the same reduce function \cite{reduceFunctions}. As such, a reduce function may operate on the same same output of the map function more than once; necessitating `rereduction' of reduced output. The reduce function requires handling of the case when rereduce = true, and when rereduce = false within the same function body. As such, writing custom reduce functions that adhere to the reduce function contract is fairly difficult for anything but the simplest examples. The reduce function is described here, with an implementation of the \textit{\_count} reduce function honoring this contract included in the appendix (see \ref{couchdb-mapreduce-contracts}) as an example and reference.

\begin{itemize}
    \item \textit{keys}: a list of tuples of the form \textless \textit{key}, \textit{id} \textgreater. \textit{key} is the key emitted by the map function defined by the user, and \textit{id} is the id of the document that was processed by the map function to emit the key (this is implicit and not defined by a user). This argument is null in the case of rereduce = true.
    \item \textit{values}: a list of the values emitted by the map function as defined by the user, with each value correlating to a the respective element in the \textit{keys} list (when rereduce = false). When rereduce = true, this argument is a list of values previously output by this same reduce function on a previous execution.
    \item \textit{rereduce}: a boolean field indicating whether the function is invoked with output from the map function (rereduce = false), or previous output from this reduce function (rereduce = true). Reduce function results are cached on internal nodes in the B+tree view indexes to facilitate incremental tree updates without having to recalculate all reduce output \cite{slack25Oct}, making the rereduce contract necessary.
\end{itemize}

Because of the nature of the \textit{reduce} function contract (allowing for rereduce = true), and in conjunction with how reduce output is stored in the B+tree, joins via MapReduce as described by \cite{chandar2010} are simply not possible. A SQL query of the form: \textit{R(a,b)} joined with \textit{S(b,c)} joined with \textit{T(c,d)} (where \textit{R}, \textit{S} and \textit{T} are relations and \textit{a}, \textit{b}, \textit{c}, \textit{d} are attributes) cannot be achieved without prior processing of all or some of the relations, or in the spirit of CouchDB, via multiple interactions with a view.

Although a 2-way join is conceptually possible as CouchDB MapReduce output (for example a reduce function could just return a list of all documents with the same key), this would result in expanding reduce output that may cause out-of-memory exceptions in the \textit{couchjs} MapReduce engine, and increasingly expensive memory and space usage towards the root node of the B+tree. This is a misuse of CouchDB's query engine \cite{reduceFunctions}. Reduce functions in CouchDB views are specifically optimized to allow incremental index alterations (i.e. incrementally altering the view index in response to a document added to the database). Because reduce output is stored as internal nodes of the B+tree index of map-output (which is ordered by map key), reduce results are still identifiable by output key in the tree and can be further grouped at higher levels in the tree with the reduce results from other nodes with the same key \cite{slack25Oct}. This is unlike other implementations of MapReduce, for example for Hadoop, where MapReduce tasks may comprise 'pipelines' of multiple map and reduce phases (contracts of these functions are defined by users). Composite return structures are acceptable output for MapReduce tasks or data structures other than used in indices. Hadoop (for example) is a less opinionated MapReduce implementation in this sense since since output data-structures are defined by the MapReduce tasks and not the Hadoop software in itself.

\subsubsection{The \textit{\_sum} Reduce Function}
Adhering to the contract of the \_sum reduce function requires that values output by the map function be either numerical, or a list of numerical values - an example is shown in \ref{fig-sum-reduce-fn} A of both allowable map function output formats. The reduce function receives values grouped by key (Figure \ref{fig-sum-reduce-fn} B) as the \nth{2} argument during execution. The reduce function signature is shown in Figure \ref{fig-sum-reduce-fn} C; the \nth{1} argument is a list of grouped key:id pairs for every document that output a specific key, and the indices of the \nth{1} argument correspond to indices in the the \nth{2} argument (list) in terms of which document produced which value.

Figure \ref{fig-sum-reduce-fn} D shown the logical treatment of the grouped values by the \_sum function; for a list of values of the same key, values at corresponding indices are summed. In the case where an index is out of bounds (when grouped values differ by list length, or some values are numerical and some are lists), the value '0' is used. Output of the \_sum function is shown in Figure \ref{fig-sum-reduce-fn} E.

\begin{figure}[H]
    \centering
    \begin{mdframed}[rightline=true,leftline=true]
        \begin{minted}{text}
/* A: Map output*/
{["key"]: 7} // (_id: x)
{["key"]: [3,1,3]} // (_id: y)
{["key2"]: [2,2,2]} // (_id: z)

/* B: Map output grouped by key */
{["key"]: [7,[3,1,3]]}
{["ke2"]: [[2,2,2]]}

/* C: Map output passed to _sum function as input */
reduce([key:id tuples], [values groupted by key], rereduce)
reduce([["key", "x"], ["key", "y"]], [7,[3,1,3]], false)
reduce([["key2", "z"]], [[2,2,2]], false)

/* D: Logical treatment of values argument (arg 2) during reduction */
{["key"]: [sum([7,3]), sum([0,1]),  sum([0,3])]}
{["key2"]: [sum([2]), sum([2]),  sum([2])]}

/* E: Reduce output (group = true) */
{
  ["key"]: [10, 1, 3],
  ["key2"]: [2,2,2]
}
    \end{minted}
    \end{mdframed}
    \caption[\_sum Reduce Function Logic]{\textbf{Figure \ref{fig-sum-reduce-fn}: Logical description of the \_sum Reduce function contract shown as a step-based series of transformations from input (map function output) to output}}
    \label{fig-sum-reduce-fn}
\end{figure}

\subsubsection{The \textit{\_stats} Reduce Function}
To adhere to the \_stats function contract, key:value output by the map function should follow the constraints:

\begin{itemize}
    \item All values must be a single number
    \item Or, all values must be an array of numbers (each array must be the same length). An example of this format is shown in Figure \ref{fig-stats-reduce-fn} A
\end{itemize}

This is unlike contract for the \_sum function, where value output is \textit{normalized} during reduction of lists of unequal length or reduction of lists and numerical values. Aggregation of grouped of values (as shown in \ref{fig-stats-reduce-fn} B) is done by the \_stats function. For values that are groups of numbers (i.e. where the map function outputs key:number) then a single \_stats object is output for each unique key. Map output is received as input to the \_stats reduce function as 3 parameters as shown in Figure \ref{fig-stats-reduce-fn} C; these parameters are:

\begin{itemize}
    \item A list of \textit{MapOutputKey}:\textit{DocumentID} key:value pairs (all the keys are the same, but the documents that \textit{emitted} them are different)
    \item A list of values for a particular key
    \item true / false (rereduction)
\end{itemize}

The values (argument 2 for the reduce contract) are aggregated by grouping values at common indices and reducing these values to a single \textit{\_stats object} as shown in \ref{fig-stats-reduce-fn} D. (Or, if the map output was a single number and not a list of numbers, then the output is an aggregation of those numbers). If the values as output by the map function are lists of lists (of numbers), then a list of \textit{\_stats objects} is output by the \_stats function as shown in Figure \ref{fig-stats-reduce-fn} E (one \_stats object per index in the list). The object output by the \_stats function for every aggregated number includes a count of how many items are included in the aggregation, the minimum value, the maximum value, the sum of all the values, and the sum of squares of all the values. The output of the \_stats function includes output of both the \_sum and \_count built-in reduce functions. However the stricter contract (values all being lists of the same length or numerical) makes this function cumbersome to use when only a sum or count is required.

\begin{figure}[H]
  \centering
  \begin{mdframed}[rightline=true,leftline=true]
    \begin{minted}{text}
/* A: Map output*/
{["key"]: [1,1,0]} // (_id: x)
{["key"]: [3,1,3]} // (_id: y)
{["key2"]: [2,2,2]} // (_id: z)

/* B: Map output grouped by key */
{["key"]: [[1,1,0],[3,1,3]]}
{["ke2"]: [[2,2,2]]}

/* C: Map output passed to _stats function as input */
reduce([key:id tuples], [values groupted by key], rereduce)
reduce([["key", "x"], ["key", "y"]], [[1,1,0],[3,1,3]], false)
reduce([["key2", "z"]], [[2,2,2]], false)

/* D: Logical treatment of values argument (arg 2) during reduction */
{["key"]: [aggregate([1,3]), aggregate([1,1]),  aggregate([0,3])]}
{["key2"]: [aggregate([2]), aggregate([2]),  aggregate([2])]}

/* E: Reduce output (group = true) */
{
  ["key"]: [
    {"sum":4,"count":2,"min":1,"max":3,"sumsqr":10},
    {"sum":2,"count":2,"min":1,"max":1,"sumsqr":2},
    {"sum":3,"count":2,"min":0,"max":3,"sumsqr":9}
  ],
  ["key2"]: [
    {"sum":2,"count":1,"min":2,"max":2,"sumsqr":4}
  ]
}
    \end{minted}
  \end{mdframed}
  \caption[\_stats Reduce Function Logic]{\textbf{Figure \ref{fig-stats-reduce-fn}: Logical description of the \_stats Reduce function contract shown as a step-based series of transformations from input (map function output) to output}}
  \label{fig-stats-reduce-fn}
\end{figure}