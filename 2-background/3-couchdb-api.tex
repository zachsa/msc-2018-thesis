\subsection{API}
\subsubsection{HTTP Interface}
All communication with CouchDB takes place via the HTTP protocol. The interface is resource orientated and strives to be RESTful. This makes such interactions very clear since the HTTP endpoints are logical and easy to remember. For example, the following sample endpoints (among many) are shown here, though the full API is very well documented online \cite{couch-api}.

\begin{itemize}
    \item \textit{GET} /dbName/:id (retrieves a document with the specified ID)
    \item \textit{PUT} /dbName/[:id] (inserts a document, optionally specifying an ID)
    \item \textit{POST} /\_bulk\_docs (inserts multiple documents - atomicity of batch insert is configurable but defaults to false)
    \item \textit{POST/GET} /\_all\_docs (Fetches multiple documents, specifying keys can be done in in the body of a post request)
\end{itemize}

\subsubsection{Design Documents}
CouchDB allows users to specify several different types of functions that can be executed on the server-side Erlang application via JSON - as a regular database document, but with an \textit{\_id} of ``\_design/$<$name$>$''. Such documents are known as \textit{design} documents and are treated as special by the CouchDB application. CouchDB executes functions that are defined in design documents on the server - functions can be defined in a variety of languages, including JavaScript. There are 6 types of functions that users can define for server-side execution:

\begin{itemize}
    \item \textit{views}
    \item \textit{shows}
    \item \textit{lists}
    \item \textit{updates}
    \item \textit{filters}
    \item \textit{validations}
\end{itemize}

Views are b+tree indexes created by specifying a mapping function and an optional reduce function to be executed on every document in a database. Map functions iteratively take each document as an argument and output key-value pair(s) - as specified by the Map function definition - that are then incorporated into the b+tree. Reduce functions are evaluated by taking Map-function outputs as inputs and returning reduced results from that input, which are stored as internal nodes on the b+tree.

Show functions act as a type of middleware, allowing users to specify transformations on a single document that has been requested from the database and then returning that document to the user - for example, transforming a document from JSON representation to HTML representation for better display in a browser. List functions are similar to show functions, but they transform sets of documents that have been iteratively retrieved from an index instead of single documents that have been retrieved from a database.

Update functions allow document updates to be performed indirectly via the HTTP API (functionally equivalent to retrieving a document, updating it, and then reinserting it). Filters, true to their name, allow users to specify filters that can be used during view retrieval, document retrieval, and in replicating data between CouchDB databases. Validations allow users to specify rules regarding when a database’s documents can be updated and by whom.

\subsection{Indices (Views)}
Essentially, CouchDB provides users a simple-to-use HTTP interface that allows for fine-grained control over b+tree structures. Documents can either be retrieved from trees directly or indices can be built as representations of databases using MapReduce and then retrieved. Index creation involves iteratively processing every document in a database and outputting a new b+tree structure (and index), providing a \textit{view} of the underlying database. Views provide a means for applying selections, projections, and aggregations of databases.

In the current CouchDB release (as of January 2018), there is an additional means of querying CouchDB databases called `Mango', which is a selection-based syntax inspired by MongoDB. The Mango syntax still processes documents via MapReduce, but it is generally faster than JavaScript functions since the MapReduce engine can be executed directly within the Erlang process on the server.

Map functions must be specified by a user and are always executed externally to the main Erlang process via marshalling between the main Erlang process and the indexing engine. Reduce functions, however, can either by executed via the main Erlang process (by specifying a built-in reduce function) or externally by the indexing engine when specifying custom reduce functions. CouchDB automatically includes a MapReduce engine (\textit{couchjs.exe}), which is a JavaScript query-server (indexing engine) coupled with Mozilla's SpiderMonkey runtime engine. This query engine is replaceable by dropping in alternative implementations in JavaScript or other languages, if required. CouchDB spawns a single couchjs.exe process per shard and executes map/reduce functions in the context of new documents sequentially according to the order in which the database was changed \cite{slack2Nov,slack7Nov}.

Conceptually, \textit{fetching all documents of type x} requires specifying an iteration through every document in the database and fetching documents with content indicative of type \textit{x}. View indices are built into CouchDB in this fashion, via iterating over every document in the database, and passing that document to the user-defined map function. A user writes code in this map function that evaluates each document and emits key-value pairs. The map output is then grouped by key and passed to the reduce function for reduction.

Users can specify whether they want to retrieve reduced results or access the unreduced map results (meaning, a reduce function isn't actually required to produce a CouchDB view). However, when passing output from the map function to the reducer, where map output is grouped by key, there is no guarantee that all the map output for a particular key will be sent to the same reduce function \cite{reduceFunctions}. As such, a reduce function may repeatedly operate on the same output as the map function, necessitating the `rereduction' of already reduced output. The reduce function requires handling in cases where \mintinline{text}{rereduce=true} and \mintinline{text}{rereduce=false} within the same function body. Therefore, writing custom reduce functions that adhere to the reduce function contract is fairly difficult for anything other than the simplest of examples. CouchDB's reduce function contract is as follows:

\begin{itemize}
    \item \textit{keys}: a list of tuples of the form of \textless \textit{key}, \textit{id} \textgreater. \textit{key} is the key emitted by the map function defined by the user and \textit{id} is the ID of the document that was processed by the map function in order to emit the key (this is implicit and not defined by a user). This argument is null in the case of \mintinline{text}{rereduce=true}.
    \item \textit{values}: a list of the values emitted by the map function as defined by the user, with each value correlating to the respective element in the keys list (when \mintinline{text}{rereduce=false}). When \mintinline{text}{rereduce=true}, this argument is a list of values that were output previously by this same reduce function on an earlier execution.
    \item \textit{rereduce}: a boolean field indicating whether the function is invoked with output from the map function (\mintinline{text}{rereduce=false}) or with previous output from the reduce function (\mintinline{text}{rereduce=true}). Reduce function results are cached on internal nodes in the b+tree view indexes to facilitate incremental tree updates without requiring the recalculation of all reduce output \cite{slack25Oct}, making the rereduce contract necessary.
\end{itemize}

\subsubsection{\textit{\_sum} Reduce Function}
Adherence to the contract of the \_sum reduce function requires that values output by the map function are either numerical or a list of numerical values - examples of both allowable map function output formats are shown in Figure \ref{fig-sum-reduce-fn} A. The reduce function receives values grouped by key (Figure \ref{fig-sum-reduce-fn} B) as the \nth{2} argument during execution. The reduce function signature is shown in Figure \ref{fig-sum-reduce-fn} C; the \nth{1} argument is a separate list of grouped key-id pairs for every document that output a specific key, and the indices of the \nth{1} argument correspond to indices in the \nth{2} argument (list), in terms of which document produced which value.

Figure \ref{fig-sum-reduce-fn} D shows the logical treatment of the grouped values by the \_sum function; for a list of values of the same key, values at corresponding indices are summed. In the case where an index is out of bounds (when grouped values differ by list length, or when some values are numerical and others are lists), the value '0' is used. Output of the \_sum function is shown in Figure \ref{fig-sum-reduce-fn} E.

\begin{figure}[H]
    \centering
    \begin{mdframed}[rightline=true,leftline=true]
        \begin{minted}{text}
/* A: Map output*/
{["key"]: 7} // (_id: x)
{["key"]: [3,1,3]} // (_id: y)
{["key2"]: [2,2,2]} // (_id: z)

/* B: Map output grouped by key */
{["key"]: [7,[3,1,3]]}
{["ke2"]: [[2,2,2]]}

/* C: Map output passed to _sum function as input */
reduce([key:id tuples], [values groupted by key], rereduce)
reduce([["key", "x"], ["key", "y"]], [7,[3,1,3]], false)
reduce([["key2", "z"]], [[2,2,2]], false)

/* D: Logical treatment of values argument (arg 2) during reduction */
{["key"]: [sum([7,3]), sum([0,1]),  sum([0,3])]}
{["key2"]: [sum([2]), sum([2]),  sum([2])]}

/* E: Reduce output (group = true) */
{
  ["key"]: [10, 1, 3],
  ["key2"]: [2,2,2]
}
    \end{minted}
    \end{mdframed}
    \caption[\_sum Reduce Function Logic]{\textbf{Figure \ref{fig-sum-reduce-fn}: Logical description of the \_sum Reduce function contract shown as a step-based series of transformations from input (map function output) to output}}
    \label{fig-sum-reduce-fn}
\end{figure}

\subsubsection{\textit{\_stats} Reduce Function}
To adhere to the \_stats function contract, key-value output by the map function should follow specific constraints:

\begin{itemize}
    \item All values must be a single number
    \item Or, all values must be an array of numbers (each array must be the same length). An example of this format is shown in Figure \ref{fig-stats-reduce-fn} A
\end{itemize}

This is unlike the contract for the \_sum function, where value output is \textit{normalized} during the reduction of lists of unequal length or the reduction of lists and numerical values. The aggregation of grouped of values (as shown in Figure \ref{fig-stats-reduce-fn} B) is completed by the \_stats function. For values that are groups of numbers (i.e. where the map function outputs key-number) then a single \_stats object is output for each unique key. Map output is received as input to the \_stats reduce function within 3 parameters as shown in Figure \ref{fig-stats-reduce-fn} C:

\begin{itemize}
    \item A list of \textit{MapOutputKey}:\textit{Document ID} key-value pairs (all the keys are the same, but the documents that \textit{emitted} them are different)
    \item A list of values for a particular key
    \item true / false (rereduction)
\end{itemize}

The values (argument 2 for the reduce contract) are aggregated by grouping them at common indices and reducing to a single \textit{\_stats object} as shown in \ref{fig-stats-reduce-fn} D. (Or, if the map output was a single number and not a list of numbers, the output becomes an aggregation of those numbers). If the values as output by the map function are lists of lists (of numbers), then a list of \_stats objects is output by the \_stats function as shown in Figure \ref{fig-stats-reduce-fn} E (one \_stats object per index within the list). The object output by the \_stats function for each aggregated number includes a count of how many items are included in the aggregation, the minimum value, the maximum value, the sum of all the values, and the sum of the squares of all the values. The output of the \_stats function includes output of both the \_sum and \_count built-in reduce functions. However, the stricter contract (when all values are lists of the same length or are numerical) makes this function cumbersome to use when only a sum or count is required.

\begin{figure}[H]
  \centering
  \begin{mdframed}[rightline=true,leftline=true]
    \begin{minted}{text}
/* A: Map output*/
{["key"]: [1,1,0]} // (_id: x)
{["key"]: [3,1,3]} // (_id: y)
{["key2"]: [2,2,2]} // (_id: z)

/* B: Map output grouped by key */
{["key"]: [[1,1,0],[3,1,3]]}
{["ke2"]: [[2,2,2]]}

/* C: Map output passed to _stats function as input */
reduce([key:id tuples], [values groupted by key], rereduce)
reduce([["key", "x"], ["key", "y"]], [[1,1,0],[3,1,3]], false)
reduce([["key2", "z"]], [[2,2,2]], false)

/* D: Logical treatment of values argument (arg 2) during reduction */
{["key"]: [aggregate([1,3]), aggregate([1,1]),  aggregate([0,3])]}
{["key2"]: [aggregate([2]), aggregate([2]),  aggregate([2])]}

/* E: Reduce output (group = true) */
{
  ["key"]: [
    {"sum":4,"count":2,"min":1,"max":3,"sumsqr":10},
    {"sum":2,"count":2,"min":1,"max":1,"sumsqr":2},
    {"sum":3,"count":2,"min":0,"max":3,"sumsqr":9}
  ],
  ["key2"]: [
    {"sum":2,"count":1,"min":2,"max":2,"sumsqr":4}
  ]
}
    \end{minted}
  \end{mdframed}
  \caption[\_stats Reduce Function Logic]{\textbf{Figure \ref{fig-stats-reduce-fn}: Logical description of the \_stats Reduce function contract shown as a step-based series of transformations from input (map function output) to output}}
  \label{fig-stats-reduce-fn}
\end{figure}