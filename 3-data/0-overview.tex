With special thanks to data owners Jane Hendry (UCT's CIO) and Stephen Marquard (the Learning Technologies Coordinator from the Center for Innovation in Learning and Teaching at UCT), student data was made available encompassing UCT admissions data, course grades, and LMS (Learning Management Software) - Sakai \cite{sakai} - interactions. This data was received as three separate datasets that are classified in terms of this project as listed below:

\begin{itemize}
    \item \textit{A grades dataset}: UCT course results
    \item \textit{A admissions dataset}: Student matric results and national benchmark test (NBT) results
    \item \textit{An events dataset}: Student interactions with the Sakai platform, measured as browser interactions
\end{itemize}

All three datasets were received as CSV exports with student IDs anonymized prior to receiving them. Anonymization was consistent across all three datasets, preserving the association of particular students across the admissions, grades and events datasets.

\section{Grades}
CSV exports received for course grade data for the years 2014, 2015 and 2016 are consistent across all three years in terms of the fields and the ordering of these fields - the only structural difference is that the 2014 CSV is tab-delimited instead of comma delimited. Using Microsoft Excel the three files were combined into a single CSV, which is described in Table \ref{tbl-data-grades}; separate CSV files could just as easily be used, but at no benefit to this analysis (and it is easier to work with a single CSV file of grades than several grades CSV files). A sample of the combined CSV file is shown in Figure \ref{fig-sample-grades}.

\begin{table}[H]
    \begin{threeparttable}
        \textbf{Table \ref{tbl-data-grades}}\par\medskip\par\medskip
        \caption[Grades data fields]{\textbf{Grades data fields}}
        \label{tbl-data-grades}
        \begin{tabularx}{\textwidth}{>{\hsize=0.8\hsize}X>{\hsize=0.6\hsize}X>{\hsize=1.6\hsize}X}
            \toprule
            \mC{c}{Field Name} & \mC{c}{Data type} & \mC{c}{Description}                              \\
            \midrule
            DownloadedDate     & date              & Excel date format                                \\
            RegAcadYear        & number            & Year                                             \\
            RegTerm            & number            & Integer ID                                       \\
            anonIDnew          & number            & Student number                                   \\
            RegProgram         & string            & Program abbreviation                             \\
            RegCareer          & string            & Academic level                                   \\
            Degree             & string            & Degree code                                      \\
            DegreeDescr        & string            & Degree title                                     \\
            Subject            & string            & Three letter abbreviation                        \\
            Catalog.           & string            & Catalog sub-component\tnote{\textsuperscript{1}} \\
            Course             & string            & Full course code description                     \\
            CourseSuffix       & string            & Course session identifier                        \\
            Session            & string            & Session name                                     \\
            Percent            & string            & Grade achieved by student                        \\
            Symbol             & string            & Symbol achieved by student                       \\
            UnitsTaken         & number            & Total units taken by student                     \\
            CourseID           & number            & Numerical course Identifier                      \\
            CourseDescr        & string            & Description of course                            \\
            CourseCareer       & string            & Academic level of course                         \\
            Faculty            & string            & Course faculty                                   \\
            Dept               & string            & Faculty department                               \\
            MaximumCrseUnits   & number            & N/A                                              \\
            CourseCount        & number            & N/A                                              \\
            CourseLevel        & number            & N/A                                              \\
            CESM               & number            & N/A                                              \\
            Sub-CESM           & number            & N/A                                              \\
            \bottomrule
        \end{tabularx}
        \scriptsize
        \begin{tablenotes}
            \item[\textsuperscript{1}]Course number and session identifier
        \end{tablenotes}
    \end{threeparttable}
\end{table}
\begin{sidewaysfigure}
  \centering
  \begin{mdframed}[topline=false,rightline=false,leftline=false]
    \centering
\begin{BVerbatim}
+--------+-------------+---------+----+------------+-----------+--------+-------------+---------+----------+----------+--------+--------------+--
| DnldDt | RegAcadYear | RegTerm | ID | RegProgram | RegCareer | Degree | DegreeDescr | Subject | Catalog. |  Course  | Suffix |   Session    |
+--------+-------------+---------+----+------------+-----------+--------+-------------+---------+----------+----------+--------+--------------+--
| txt    |        2016 |    1161 |  1 | CB003      | UGRD      | QCB007 | txt         | MAM     | 1000W    | MAM1000W | W      | Full Year    |
| txt    |        2016 |    1161 |  1 | CB003      | UGRD      | QCB007 | txt         | CSC     | 1015F    | CSC1015F | F      | Semester One |
| txt    |        2016 |    1161 |  2 | CB004      | UGRD      | QCB002 | txt         | CSC     | 1015F    | CSC1015F | F      | Semester One |
| txt    |        2016 |    1161 |  3 | EB022      | UGRD      | QEB028 | txt         | EEE     | 2036S    | EEE2036S | S      | Semester Two |
| txt    |        2015 |    1151 |  3 | EB022      | UGRD      | EB28   | txt         | CSC     | 1015F    | CSC1015F | F      | Semester One |
| txt    |        2016 |    1161 |  4 | CB004      | UGRD      | QCB002 | txt         | CSC     | 1015F    | CSC1015F | F      | Semester One |
| txt    |        2015 |    1151 |  4 | CB003      | UGRD      | CB07   | txt         | CSC     | 1015F    | CSC1015F | F      | Semester One |
| txt    |        2015 |    1151 |  5 | CB003      | UGRD      | CB07   | txt         | CSC     | 1015F    | CSC1015F | F      | Semester One |
+--------+-------------+---------+----+------------+-----------+--------+-------------+---------+----------+----------+--------+--------------+--
--+---------+--------+------------+----------+------------+-----------+---------+------+--------------+-----------+-----------+------+----------+
  | Percent | Symbol | UnitsTaken | CourseID | CourseDesc | CourseCrr | Faculty | Dept | MaxCrseUnits | CrseCount | CrseLevel | CESM | Sub-CESM |
--+---------+--------+------------+----------+------------+-----------+---------+------+--------------+-----------+-----------+------+----------+
  |      71 | 2+     |         36 |   107088 | txt        | UGRD      | SCI     | MAM  |           36 |         1 |        41 | 1501 |        1 |
  |      70 | 2+     |         18 |   103034 | txt        | UGRD      | SCI     | CSC  |           18 |       0.5 |        41 |  606 |        6 |
  |      55 | 3      |         18 |   103034 | txt        | UGRD      | SCI     | CSC  |           18 |       0.5 |        41 |  606 |        6 |
  |      63 | 2-     |         12 |     1642 | txt        | UGRD      | EBE     | EEE  |           12 |         1 |        42 |  809 |        9 |
  |      77 | 1      |         18 |   103034 | txt        | UGRD      | SCI     | CSC  |           18 |       0.5 |        41 |  606 |        6 |
  |      54 | 3      |         18 |   103034 | txt        | UGRD      | SCI     | CSC  |           18 |       0.5 |        41 |  606 |        6 |
  |      39 | F      |         18 |   103034 | txt        | UGRD      | SCI     | CSC  |           18 |       0.5 |        41 |  606 |        6 |
  |      39 | F      |         18 |   103034 | txt        | UGRD      | SCI     | CSC  |           18 |       0.5 |        41 |  606 |        6 |
--+---------+--------+------------+----------+------------+-----------+---------+------+--------------+-----------+-----------+------+----------+
\end{BVerbatim}
  \end{mdframed}
  \caption[Grades data sample]{\textbf{Figure \ref{fig-sample-grades}: Grades data sample}}
  \label{fig-sample-grades}
\end{sidewaysfigure}


\section{Admissions}
CSV exports received for student admissions data for the the 2014, 2015 and 2016 student cohorts. The fields in these CSVs are not consistent across all three years; certain fields are capitalized differently, fields are ordered differently from year to year. Additionally, the 2014 and 2015 CSV files contain fields that show student's UCT academic performance for years subsequent to each student's initial registration. For example, benchmarks data from 2014 includes academic performance for the years 2015/2016, and the 2015 benchmarks data includes academic performance for the year 2016 - resulting in many repeating fields and variable row lengths.

Although CouchDB is lenient in terms of data structure when modeling data (JSON documents are semi-structured), repeated fields in CSV data that is serialized to JSON strings results in unpredictably losing information (CouchDB discards all but one duplicated JSON key at random).

For automated CSV exports, it is reasonable to expect that there are no repeated field named. It is reasonable to re-request CSVs that are malformed, have duplicated field names, etc. However since these CSVs are only received once, it is a reasonable procedure to perform corrections in Microsoft Excel or similar tools if possible - such as done in this project. Microsoft Excel is used to normalize the Benchmark data across all three years as a result. Since Excel processing is required in any case, field names are also normalized where capitalization, spelling or other inconsistencies across the CSVs are present. (Although CouchDB is quite good at handling such inconsistencies - in fact it's on of the big advantages of using CouchDB - this is not something being assessed in this project, and just results in overcomplicated MapReduce functions). CSVs containing admissions data are processed in Excel according to the logic:

\begin{itemize}
    \item Adjusting fields to be spelled the same, normalizing whitespace, and making capitalization consistent for fields with the same name
    \item Removing duplicated fields for years subsequent to student's enrollment date
    \item Removing data not considered ethical to work with - race, gender, etc.
    \item Compile all the admissions data into a single CSV (for ease of use)
\end{itemize}

The single, processed admissions CSV file contains an anonymized list of students that enrolled at UCT in 2014, 2015, and 2016 as well as these students matric results and NBT test scores. The CSV file is described in Table \ref{tbl-data-admissions} in terms of field names, a description of these fields and each field's data type. A sample of the file is shown in Figure \ref{fig-sample-admissions}.

\begin{table}[H]
    \begin{threeparttable}
        \textbf{Table \ref{tbl-data-admissions}}\par\medskip\par\medskip
        \caption[Admissions data fields]{\textbf{Admissions data fields}}
        \label{tbl-data-admissions}
        \begin{tabularx}{\textwidth}{>{\hsize=1.3\hsize}X>{\hsize=0.2\hsize}X>{\hsize=1.5\hsize}X}
            \toprule
            \mC{c}{Field Name}    & \mC{c}{Data type} & \mC{c}{Description}       \\
            \midrule
            anonIDnew             & number            & Anonymized student number \\
            Career                & string            & Academic level            \\
            Citizenship Residency & string            & Student citizenship       \\
            SA School             & string            & School name (if in RSA)   \\
            Eng Grd12 Rslt        & string            & Grade 12 English          \\
            Math Grd12 Rslt       & string            & Grade 12 Math             \\
            Mth Lit Grd12 Rslt    & string            & Grade 12 Math Literacy    \\
            Adv Mth Grd12 Rslt    & string            & Grade 12 Advanced Math    \\
            Phy Sci Grd12 Rslt    & string            & Grade 12 Science          \\
            NBT AL Score          & string            & NBT                       \\
            NBT QL Score          & string            & NBT                       \\
            NBT Math Score        & string            & NBT                       \\
            RegAcadYear           & number            & First registration at UCT \\
            \bottomrule
        \end{tabularx}
    \end{threeparttable}
\end{table}
\begin{sidewaysfigure}
    \centering
    \begin{mdframed}[topline=false,rightline=false,leftline=false]
        \centering
        \begin{BVerbatim}
+----+-------------+-----------------------+------------------------------+-----------+------------+---------------+--
| ID |   Career    | Citizenship Residency |          SA School           | Eng Grd12 | Math Grd12 | Mth Lit Grd12 |
+----+-------------+-----------------------+------------------------------+-----------+------------+---------------+--
|  1 | UGRD        | SA Citizen            | St Anne's College            |        84 |         88 |               |
|  2 | UGRD        | SA Citizen            | St Andrews Girls' School     |        76 |         78 |               |
|  3 | Second Year | C                     | Clifton College              |        75 |         78 |               |
|  4 | Second Year | C                     | Crawford North Coast College |        82 |         85 |               |
|  5 | Second Year | C                     | Ignore                       |        82 |         85 |               |
+----+-------------+-----------------------+------------------------------+-----------+------------+---------------+--
--+---------------+---------------+--------------+--------------+----------------+-------------+
| Adv Mth Grd12 | Phy Sci Grd12 | NBT AL Score | NBT QL Score | NBT Math Score | RegAcadYear |
--+---------------+---------------+--------------+--------------+----------------+-------------+
|               |            94 |           80 |           76 |             89 |        2016 |
|               |            76 |           75 |           58 |             61 |        2016 |
|               |            84 |            0 |            0 |              0 |        2015 |
|               |            94 |           73 |           71 |             86 |        2015 |
|               |            94 |           73 |           71 |             86 |        2016 |
--+---------------+---------------+--------------+--------------+----------------+-------------+
        \end{BVerbatim}
    \end{mdframed}
    \caption[Admissions data sample]{\textbf{Figure \ref{fig-sample-admissions}: Admissions data sample}}
    \label{fig-sample-admissions}
\end{sidewaysfigure}


\section{Events}
The CSV export received for Events data is for 2016, with a description of the fields shown in Table \ref{tbl-data-events}. The Events CSV export cannot be opened using Microsoft Excel due to its large size (\textgreater 5GB). This CSV contains no header rows, and information on the headers had to be obtained separately. A small sample of the CSV was taken using a BASH terminal - the \mintinline{text}{head -n 100 <file.csv>} will show the first 100 lines of a CSV. A sample of this CSV file is shown in Figure \ref{fig-sample-events}.

\begin{table}[H]
    \begin{threeparttable}
        \textbf{Table \ref{tbl-data-events}}\par\medskip\par\medskip
        \caption[Events data fields]{\textbf{Events data fields}}
        \label{tbl-data-events}
        \begin{tabularx}{\textwidth}{>{\hsize=0.8\hsize}X>{\hsize=0.6\hsize}X>{\hsize=1.6\hsize}X}
            \toprule
            \mC{c}{Field Name} & \mC{c}{Data type} & \mC{c}{Description}                                             \\
            \midrule
            event\_date        & date              & yyyy-mm-dd hh:mm:ss                                             \\
            event\_id          & number            & Number\tnote{\textsuperscript{1}}                               \\
            uct\_id            & number            & Anonymized student number                                       \\
            site\_key          & number            & Foreign key reference to course site\tnote{\textsuperscript{2}} \\
            ref                & string            & Detail of event                                                 \\
            \bottomrule
        \end{tabularx}
        \scriptsize
        \begin{tablenotes}
            \item[\textsuperscript{1}]This column represents a foreign key, allowing for a more general classification of events
            \item[\textsuperscript{2}]This key references Sakai-specific fields and not grades data recieved reparately
        \end{tablenotes}
    \end{threeparttable}
\end{table}
\begin{figure}[H]
  \centering
  \begin{mdframed}
    \centering
\begin{BVerbatim}

+------------------+----------+--------+----------+-----+
|    event_date    | event_id | uct_id | site_key | ref |
+------------------+----------+--------+----------+-----+
| 2/18/2016 17:12  |      281 |      1 |    50680 | txt |
| 10/18/2016 17:12 |      281 |      1 |    27933 | txt |
| 5/18/2016 17:12  |      281 |      2 |    50680 | txt |
| 5/19/2016 17:12  |      281 |      2 |    27933 | txt |
| 9/18/2016 17:12  |      281 |      2 |    27933 | txt |
| 4/18/2016 17:12  |      281 |      3 |    50680 | txt |
| 4/18/2016 17:12  |      281 |      3 |    27933 | txt |
| 5/18/2016 17:12  |      281 |      3 |    27933 | txt |
| 6/18/2016 17:12  |      281 |      3 |    31585 | txt |
| 11/18/2016 17:12 |      281 |      4 |    31585 | txt |
| 11/18/2016 17:12 |      281 |      5 |    31585 | txt |
+------------------+----------+--------+----------+-----+

\end{BVerbatim}
  \end{mdframed}
  \caption[Events data sample]{\textbf{Figure \ref{fig-sample-events}: Events data sample}}
  \label{fig-sample-events}
\end{figure}
