The focus of this project is the manipulation of data using CouchDB and associated technologies, whilst specifically using data from the business domain of Educational Data Mining (EDM). This field is well developed and many studies have sought to model student performance based on markers such as attendance, assignment and test grades, high school marks, demographic data, etc. Different means of model generation have been discussed by the EDM community, such as predictive analysis via decision tree generation \cite{Qasem20016,Balestra2017,casper2017,Dimitris,zebun2005,Mierle:2005}, with varying results. Other models have been applied within the field of EDM, as discussed in a review of EDM up to 2009 \cite{bakerEdMiningSummary}.

UCT’s Chief Information Officer (CIO) Jane Hendry and Stephen Marquard (the Learning Technologies Coordinator from the Center for Innovation in Learning and Teaching at UCT), made anonymized student data available for use in the present study. This encompasses UCT admissions data, course grades, and Sakai interactions. The three separate datasets were recieved in CSV format and are classified as:

\begin{itemize}
    \item \textit{grades}: UCT course results
    \item \textit{admissions}: Student Grade 12 and National Benchmark Test (NBT) results
    \item \textit{events}: Student interactions with the Sakai platform, measured as browser interactions
\end{itemize}

\section{Grades}
Three years of data are available (2014, 2015, and 2016) and were received as separate CSV files, then compiled into a single file with normalized delimiters. The fields in the compiled file are described in Table \ref{tbl-data-grades}, and a sample of this file is shown in Figure \ref{fig-sample-grades}. Essentially this data comprises a list of the grades achieved for specific courses in specific years, along with the anonymized student numbers. The individual files could have remained separate, but for the purposes of this investigation it is easier to work with a single file rather than multiple files.

\begin{table}[H]
    \begin{threeparttable}
        \textbf{Table \ref{tbl-data-grades}}\par\medskip\par\medskip
        \caption[Grades data fields]{Grades data fields}
        \label{tbl-data-grades}
        \begin{tabularx}{\textwidth}{>{\hsize=0.8\hsize}X>{\hsize=0.6\hsize}X>{\hsize=1.6\hsize}X}
            \toprule
            \mC{c}{Field Name} & \mC{c}{Data type} & \mC{c}{Description}                              \\
            \midrule
            DownloadedDate     & date              & Excel date format                                \\
            RegAcadYear        & number            & Year                                             \\
            RegTerm            & number            & Integer ID                                       \\
            anonIDnew          & number            & Student number\tnote{\textsuperscript{1}}        \\
            RegProgram         & string            & Program abbreviation                             \\
            RegCareer          & string            & Academic level                                   \\
            Degree             & string            & Degree code                                      \\
            DegreeDescr        & string            & Degree title                                     \\
            Subject            & string            & Three letter abbreviation                        \\
            Catalog.           & string            & Catalog sub-component\tnote{\textsuperscript{2}} \\
            Course             & string            & Full course code description                     \\
            CourseSuffix       & string            & Course session identifier                        \\
            Session            & string            & Session name                                     \\
            Percent            & string            & Grade achieved by student                        \\
            Symbol             & string            & Symbol achieved by student                       \\
            UnitsTaken         & number            & Total units taken by student                     \\
            CourseID           & number            & Numerical course Identifier                      \\
            CourseDescr        & string            & Description of course                            \\
            CourseCareer       & string            & Academic level of course                         \\
            Faculty            & string            & Course faculty                                   \\
            Dept               & string            & Faculty department                               \\
            MaximumCrseUnits   & number            & N/A                                              \\
            CourseCount        & number            & N/A                                              \\
            CourseLevel        & number            & N/A                                              \\
            CESM               & number            & N/A                                              \\
            Sub-CESM           & number            & N/A                                              \\
            \bottomrule
        \end{tabularx}
        \scriptsize
        \begin{tablenotes}
            \item[\textsuperscript{1}]Anonymization was performed by Associate Professor Sonia Berman
            \item[\textsuperscript{2}]Course number and session identifier
        \end{tablenotes}
    \end{threeparttable}
\end{table}
\begin{sidewaysfigure}
  \centering
  \begin{mdframed}[topline=false,rightline=false,leftline=false]
    \centering
\begin{BVerbatim}
+--------+-------------+---------+----+------------+-----------+--------+-------------+---------+----------+----------+--------+--------------+--
| DnldDt | RegAcadYear | RegTerm | ID | RegProgram | RegCareer | Degree | DegreeDescr | Subject | Catalog. |  Course  | Suffix |   Session    |
+--------+-------------+---------+----+------------+-----------+--------+-------------+---------+----------+----------+--------+--------------+--
| txt    |        2016 |    1161 |  1 | CB003      | UGRD      | QCB007 | txt         | MAM     | 1000W    | MAM1000W | W      | Full Year    |
| txt    |        2016 |    1161 |  1 | CB003      | UGRD      | QCB007 | txt         | CSC     | 1015F    | CSC1015F | F      | Semester One |
| txt    |        2016 |    1161 |  2 | CB004      | UGRD      | QCB002 | txt         | CSC     | 1015F    | CSC1015F | F      | Semester One |
| txt    |        2016 |    1161 |  3 | EB022      | UGRD      | QEB028 | txt         | EEE     | 2036S    | EEE2036S | S      | Semester Two |
| txt    |        2015 |    1151 |  3 | EB022      | UGRD      | EB28   | txt         | CSC     | 1015F    | CSC1015F | F      | Semester One |
| txt    |        2016 |    1161 |  4 | CB004      | UGRD      | QCB002 | txt         | CSC     | 1015F    | CSC1015F | F      | Semester One |
| txt    |        2015 |    1151 |  4 | CB003      | UGRD      | CB07   | txt         | CSC     | 1015F    | CSC1015F | F      | Semester One |
| txt    |        2015 |    1151 |  5 | CB003      | UGRD      | CB07   | txt         | CSC     | 1015F    | CSC1015F | F      | Semester One |
+--------+-------------+---------+----+------------+-----------+--------+-------------+---------+----------+----------+--------+--------------+--
--+---------+--------+------------+----------+------------+-----------+---------+------+--------------+-----------+-----------+------+----------+
  | Percent | Symbol | UnitsTaken | CourseID | CourseDesc | CourseCrr | Faculty | Dept | MaxCrseUnits | CrseCount | CrseLevel | CESM | Sub-CESM |
--+---------+--------+------------+----------+------------+-----------+---------+------+--------------+-----------+-----------+------+----------+
  |      71 | 2+     |         36 |   107088 | txt        | UGRD      | SCI     | MAM  |           36 |         1 |        41 | 1501 |        1 |
  |      70 | 2+     |         18 |   103034 | txt        | UGRD      | SCI     | CSC  |           18 |       0.5 |        41 |  606 |        6 |
  |      55 | 3      |         18 |   103034 | txt        | UGRD      | SCI     | CSC  |           18 |       0.5 |        41 |  606 |        6 |
  |      63 | 2-     |         12 |     1642 | txt        | UGRD      | EBE     | EEE  |           12 |         1 |        42 |  809 |        9 |
  |      77 | 1      |         18 |   103034 | txt        | UGRD      | SCI     | CSC  |           18 |       0.5 |        41 |  606 |        6 |
  |      54 | 3      |         18 |   103034 | txt        | UGRD      | SCI     | CSC  |           18 |       0.5 |        41 |  606 |        6 |
  |      39 | F      |         18 |   103034 | txt        | UGRD      | SCI     | CSC  |           18 |       0.5 |        41 |  606 |        6 |
  |      39 | F      |         18 |   103034 | txt        | UGRD      | SCI     | CSC  |           18 |       0.5 |        41 |  606 |        6 |
--+---------+--------+------------+----------+------------+-----------+---------+------+--------------+-----------+-----------+------+----------+
\end{BVerbatim}
  \end{mdframed}
  \caption[Grades data sample]{\textbf{Figure \ref{fig-sample-grades}: Grades data sample}}
  \label{fig-sample-grades}
\end{sidewaysfigure}


\section{Admissions}
Upon receipt, the files containing data from the three years (2014, 2015, 2016) were formatted inconsistently: corresponding fields were spelled/capitalized differently across the files, the files had completely different sets of fields, the ordering of the fields was inconsistent, and the files were delimited differently. It is possible to work with these files directly but doing so would complicate the analytical procedure and would not add any value to the investigation. As such, a single CSV was compiled addressing the following:

\begin{itemize}
    \item Field names were adjusted to feature uniform spelling with normalized whitespace (where field names contain whitespace)
    \item Duplicated fields were removed (UCT performance is appended to each student's admissions data for subsequent years of enrollment)
    \item Potentially contentious data (race, gender, etc.) that this project is not ethically cleared to work with was removed
\end{itemize}

The single, processed admissions CSV file contains an anonymized list of students that enrolled at UCT along with corresponding Grade 12 results and NBT scores. The CSV file is described in Table \ref{tbl-data-admissions} and a sample of the file is shown in Figure \ref{fig-sample-admissions}.

\section{Events}
Sakai usage data is available for 2016 only and was received as a single CSV file that is 5.2GB in size. Due to its large size, the file cannot be opened in Microsoft Excel and so no scrubbing/amendment was performed on it. Fields in this file are described in Table \ref{tbl-data-events}. The file as received doesn't contain header information, which had to be obtained separately; for ease of reference, these headers are included in the CSV sample shown in Figure \ref{fig-sample-events} \footnote{Lines from files can be printed to BASH terminal windows via the command: \mintinline{text}{head -n <no. of lines> <file.ext>}}.

\begin{table}[H]
    \begin{threeparttable}
        \textbf{Table \ref{tbl-data-admissions}}\par\medskip\par\medskip
        \caption[Admissions data fields]{Admissions data fields}
        \label{tbl-data-admissions}
        \begin{tabularx}{\textwidth}{>{\hsize=0.3\hsize}Y>{\hsize=1.3\hsize}X>{\hsize=0.6\hsize}X>{\hsize=1.8\hsize}X}
            \toprule
            \mC{c}{Include\tnote{\textsuperscript{1}}  } & \mC{c}{Field Name}    & \mC{c}{Data type} & \mC{c}{Description}                       \\
            \midrule
            \cmark                                       & anonIDnew             & number            & Student number\tnote{\textsuperscript{2}} \\
            \xmark                                       & Career                & string            & Academic level                            \\
            \xmark                                       & Citizenship Residency & string            & Student citizenship                       \\
            \xmark                                       & SA School             & string            & School name (if in RSA)                   \\
            \cmark                                       & Eng Grd12 Rslt        & string            & Grade 12 English                          \\
            \cmark                                       & Math Grd12 Rslt       & string            & Grade 12 Math                             \\
            \cmark                                       & Mth Lit Grd12 Rslt    & string            & Grade 12 Math Literacy                    \\
            \cmark                                       & Adv Mth Grd12 Rslt    & string            & Grade 12 Advanced Math                    \\
            \cmark                                       & Phy Sci Grd12 Rslt    & string            & Grade 12 Science                          \\
            \cmark                                       & NBT AL Score          & string            & NBT\tnote{\textsuperscript{3}}            \\
            \cmark                                       & NBT QL Score          & string            & NBT                                       \\
            \cmark                                       & NBT Math Score        & string            & NBT                                       \\
            \cmark                                       & RegAcadYear           & number            & First registration at UCT                 \\
            \bottomrule
        \end{tabularx}
        \scriptsize
        \begin{tablenotes}
            \item[\textsuperscript{1}]Attributes were included via a white-listing process using Microsoft Excel
            \item[\textsuperscript{2}]Anonymization was performed by Associate Professor Sonia Berman
            \item[\textsuperscript{3}]National Benchmark Test
        \end{tablenotes}
    \end{threeparttable}
\end{table}
\begin{sidewaysfigure}
    \centering
    \begin{mdframed}[topline=false,rightline=false,leftline=false]
        \centering
        \begin{BVerbatim}
+----+-------------+-----------------------+------------------------------+-----------+------------+---------------+--
| ID |   Career    | Citizenship Residency |          SA School           | Eng Grd12 | Math Grd12 | Mth Lit Grd12 |
+----+-------------+-----------------------+------------------------------+-----------+------------+---------------+--
|  1 | UGRD        | SA Citizen            | St Anne's College            |        84 |         88 |               |
|  2 | UGRD        | SA Citizen            | St Andrews Girls' School     |        76 |         78 |               |
|  3 | Second Year | C                     | Clifton College              |        75 |         78 |               |
|  4 | Second Year | C                     | Crawford North Coast College |        82 |         85 |               |
|  5 | Second Year | C                     | Ignore                       |        82 |         85 |               |
+----+-------------+-----------------------+------------------------------+-----------+------------+---------------+--
--+---------------+---------------+--------------+--------------+----------------+-------------+
| Adv Mth Grd12 | Phy Sci Grd12 | NBT AL Score | NBT QL Score | NBT Math Score | RegAcadYear |
--+---------------+---------------+--------------+--------------+----------------+-------------+
|               |            94 |           80 |           76 |             89 |        2016 |
|               |            76 |           75 |           58 |             61 |        2016 |
|               |            84 |            0 |            0 |              0 |        2015 |
|               |            94 |           73 |           71 |             86 |        2015 |
|               |            94 |           73 |           71 |             86 |        2016 |
--+---------------+---------------+--------------+--------------+----------------+-------------+
        \end{BVerbatim}
    \end{mdframed}
    \caption[Admissions data sample]{\textbf{Figure \ref{fig-sample-admissions}: Admissions data sample}}
    \label{fig-sample-admissions}
\end{sidewaysfigure}


\begin{table}[H]
    \begin{threeparttable}
        \textbf{Table \ref{tbl-data-events}}\par\medskip\par\medskip
        \caption[Events data fields]{\textbf{Events data fields}}
        \label{tbl-data-events}
        \begin{tabularx}{\textwidth}{>{\hsize=0.8\hsize}X>{\hsize=0.6\hsize}X>{\hsize=1.6\hsize}X}
            \toprule
            \mC{c}{Field Name} & \mC{c}{Data type} & \mC{c}{Description}                                             \\
            \midrule
            event\_date        & date              & yyyy-mm-dd hh:mm:ss                                             \\
            event\_id          & number            & Number\tnote{\textsuperscript{1}}                               \\
            uct\_id            & number            & Anonymized student number                                       \\
            site\_key          & number            & Foreign key reference to course site\tnote{\textsuperscript{2}} \\
            ref                & string            & Detail of event                                                 \\
            \bottomrule
        \end{tabularx}
        \scriptsize
        \begin{tablenotes}
            \item[\textsuperscript{1}]This column represents a foreign key, allowing for a more general classification of events
            \item[\textsuperscript{2}]This key references Sakai-specific fields and not grades data recieved reparately
        \end{tablenotes}
    \end{threeparttable}
\end{table}
\begin{sidewaysfigure}
    \centering
    \begin{mdframed}[topline=false,rightline=false,leftline=false]
    \centering
\begin{BVerbatim}
+-----------------------+----------+---------+----------+---------------------------------------------------------+
|      event_date       | event_id | uct_id  | site_key |                           ref                           |
+-----------------------+----------+---------+----------+---------------------------------------------------------+
| 1/6/2016  11:13:18 AM |      281 | 3045582 |     5401 | /presence/192bdcbd-f604-4494-9a9b-3fa602333b3d-presence |
| 1/6/2016  11:13:20 AM |      281 | 2939634 |    30512 | /presence/8f2abfb4-fd78-43bb-bd63-40b5cc3871cb-presence |
| 1/6/2016  11:13:25 AM |      281 | 2933318 |    16602 | /presence/4e168959-8ac0-40c0-a0f2-dc16a315bd48-presence |
| 1/6/2016  11:13:43 AM |      281 | 3004534 |     4459 | /presence/14a625b8-99d6-46d9-8690-cbfc31b7a402-presence |
| 1/6/2016  11:13:45 AM |      281 | 3006684 |    18894 | /presence/5922686b-8b49-4529-8a82-2bb3e11dae1b-presence |
| 1/6/2016  11:13:56 AM |      281 | 2762064 |    17926 | /presence/544d2135-e1cc-4f25-8869-ca35688e9c04-presence |
| 1/6/2016  11:13:59 AM |      281 | 2820866 |    45041 | /presence/d3d13f65-1cd7-4d50-a53d-26c11617e37e-presence |
| 1/6/2016  11:14:01 AM |      281 | 2884124 |     2723 | /presence/0c5f41df-0169-4095-b8f3-dc0e7bb2960b-presence |
| 1/6/2016  11:14:06 AM |      281 | 2884124 |    21919 | /presence/67a4d56d-d6ef-48c7-a207-420094cfab92-presence |
| 1/6/2016  11:14:12 AM |      281 | 2848240 |     6098 | /presence/1c8a2336-ab3a-445c-aad9-a1b96f63e9ef-presence |
| 1/6/2016  11:14:12 AM |      281 | 2884124 |     6933 | /presence/202b7cd1-70a9-4fe9-ae81-57eb9d41232d-presence |
| 1/6/2016  11:14:15 AM |      281 | 2929368 |    21653 | /presence/6655872d-f96b-49a4-869a-7f32f9be827a-presence |
| 1/6/2016  11:14:29 AM |      281 | 2653788 |    32925 | /presence/9aadfa0a-cab3-44e4-9266-2cfcd83115fa-presence |
+-----------------------+----------+---------+----------+---------------------------------------------------------+
\end{BVerbatim}
  \end{mdframed}
  \caption[Events data sample]{\textbf{Figure \ref{fig-sample-events}: Events data sample}}
  \label{fig-sample-events}
\end{sidewaysfigure}
