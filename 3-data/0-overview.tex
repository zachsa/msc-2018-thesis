With special thanks to data owners Jane Hendry (UCT's CIO) and Stephen Marquard (the Learning Technologies Coordinator from the Center for Innovation in Learning and Teaching at UCT), student data was made available encompassing UCT entrance assessments, course grades, and LMS (Learning Management Software) - Sakai \cite{xxx} - interactions. This data was received as three separate datasets that are classified in terms of this project as listed below:

\begin{itemize}
    \item \textit{grades}: UCT course results
    \item \textit{benchmarks}: Student matric results and national benchmark test results
    \item \textit{events}: Student interactions with the Sakai platform, measured as browser interactions
\end{itemize}

All three datasets were received as CSV exports with student IDs anonymized prior to receiving them. Anonymization was consistent across all three datasets, preserving the association of particular students with correct Benchmarks, Grades and Events. Some Excel manipulation on the files was required on the Grades and Benchmarks CSVs (discussed below).

\section{Grades}
CSV exports received for grades data for the years 2014, 2015 and 2016 are consistent across all three years in terms of the fields and the ordering of these fields - the only structural difference is that the 2014 CSV is tab-delimited instead of comma delimited. Using Microsoft Excel the three files were combined into a single CSV, which is described in Table \ref{tbl-data-grades}; all the data in the combined CSV is used in this project. A sample of this CSV file is shown in Figure \ref{fig-grades-csv-sample}.

\begin{table}[H]
    \begin{threeparttable}
        \textbf{Table \ref{tbl-data-grades}}\par\medskip\par\medskip
        \caption[Grade Data CSV]{Fields received in the CSV export of Grade data}
        \label{tbl-data-grades}
        \begin{tabularx}{\textwidth}{>{\hsize=0.8\hsize}X>{\hsize=0.6\hsize}X>{\hsize=1.6\hsize}X}
            \toprule
            \mC{c}{Field Name} & \mC{c}{Data type} & \mC{c}{Description}                              \\
            \midrule
            DownloadedDate     & date              & Excel date format                                \\
            RegAcadYear        & number            & Year                                             \\
            RegTerm            & number            & Integer ID                                       \\
            anonIDnew          & number            & Student number\tnote{\textsuperscript{1}}        \\
            RegProgram         & string            & Program abbreviation                             \\
            RegCareer          & string            & Academic level                                   \\
            Degree             & string            & Degree code                                      \\
            DegreeDescr        & string            & Degree title                                     \\
            Subject            & string            & Three letter abbreviation                        \\
            Catalog.           & string            & Catalog sub-component\tnote{\textsuperscript{2}} \\
            Course             & string            & Full course code description                     \\
            CourseSuffix       & string            & Course session identifier                        \\
            Session            & string            & Session name                                     \\
            Percent            & string            & Grade achieved by student                        \\
            Symbol             & string            & Symbol achieved by student                       \\
            UnitsTaken         & number            & Total units taken by student                     \\
            CourseID           & number            & Numerical course Identifier                      \\
            CourseDescr        & string            & Description of course                            \\
            CourseCareer       & string            & Academic level of course                         \\
            Faculty            & string            & Course faculty                                   \\
            Dept               & string            & Faculty department                               \\
            MaximumCrseUnits   & number            & N/A                                              \\
            CourseCount        & number            & N/A                                              \\
            CourseLevel        & number            & N/A                                              \\
            CESM               & number            & N/A                                              \\
            Sub-CESM           & number            & N/A                                              \\
            \bottomrule
        \end{tabularx}
        \scriptsize
        \begin{tablenotes}
            \item[\textsuperscript{1}]Anonymization was performed by Associate Professor Sonia Berman
            \item[\textsuperscript{2}]Course number and session identifier
        \end{tablenotes}
    \end{threeparttable}
\end{table}
\input{3-data/figures/fig-grades-csv-sample}

\section{Benchmarks}
CSV exports received for benchmarks contain data for the 2014, 2015 and 2016 student cohorts. The fields in these CSVs are not consistent across all three years; certain fields are capitalized differently, fields are ordered differently from year to year. Additionally, fields are included to show UCT academic performance for years subsequent to each student's registration. For example, benchmarks data from 2014 includes academic performance for the years 2015/2016, and the 2015 benchmarks data includes academic performance for the year 2016 - resulting in many repeating fields and variable row lengths.

CouchDB is an excellent tool for normalizing schemas, since user defined map functions can query every document with unique logic according to schema version and output a normalized document representation. But although CouchDB is lenient in terms of data structure when modeling data (JSON documents are semi-structured), some structuring beyond a 'visual' structure is still required. As such, the Benchmark data as received in CSV form could be broadly classified as 'unstructured'. Microsoft Excel was used to normalize the Benchmark data across all three years (Excel is an excellent tool for working with data structured visually).

In order to take advantage of CouchDB as a means of `flattening' inconsistent schemas, different CSV exports compared to the ones received would be required from the system housing the Benchmark and other demographic data. CouchDB doesn't allow for duplicated keys in JSON documents; duplicated keys don't actually cause errors, but one of the duplicate keys is discarded (theoretically at random since keys are not ordered according to the JSON specification). Although it's possible to process CSVs with inconsistent formatting such as these, there is little incentive or reason to do so since every rule has to be manually specified and the logic could never be reused. CSVs are normalized in Excel according to the logic:

\begin{itemize}
    \item Adjusting fields to be spelled the same, normalizing whitespace, and making capitalization consistent for fields with the same name
    \item Removing duplicated fields for years subsequent to student's enrollment date
    \item Removing data not considered ethical to work with - race, gender, etc.
\end{itemize}

The normalized Benchmark data comprises a single CSV containing an anonymized list of students that enrolled at UCT in 2014, 2015, and 2016 and the benchmark results of these students. A description of this CSV is shown in Table \ref{tbl-data-benchmarks} in terms of field names, a description of these fields and each field's data type. A sample of this CSV file is shown in Figure \ref{fig-benchmarks-csv-sample}.

\begin{table}[H]
    \begin{threeparttable}
        \textbf{Table \ref{tbl-data-benchmarks}}\par\medskip\par\medskip
        \caption[Benchmark Data CSV]{Fields received in the CSV export of Benchmark data}
        \label{tbl-data-benchmarks}
        \begin{tabularx}{\textwidth}{>{\hsize=0.3\hsize}Y>{\hsize=1.3\hsize}X>{\hsize=0.6\hsize}X>{\hsize=1.8\hsize}X}
            \toprule
            \mC{c}{Include\tnote{\textsuperscript{1}}  } & \mC{c}{Field Name}     & \mC{c}{Data type} & \mC{c}{Description}                                  \\
            \midrule
            \cmark                                       & anonIDnew              & number            & Anonymized student number\tnote{\textsuperscript{2}} \\
            \xmark                                       & Career                 & string            & Academic level                                       \\
            \xmark                                       & Citizenship Residency  & string            & Student citizenship                                  \\
            \xmark                                       & SA School              & string            & School name (if in RSA)                              \\
            \cmark                                       & Eng Grd12 Fin Rslt     & string            & Grade 12 English result                              \\
            \cmark                                       & Math Grd12 Fin Rslt    & string            & Grade 12 Math result                                 \\
            \cmark                                       & Mth Lit Grd12 Fin Rslt & string            & Grade 12 Math Literacy result                        \\
            \cmark                                       & Adv Mth Grd12 Fin Rslt & string            & Grade 12 Advanced Math result                        \\
            \cmark                                       & Phy Sci Grd12 Fin Rslt & string            & Grade 12 Science result                              \\
            \cmark                                       & NBT AL Score           & string            & National benchmark test (NBT)                        \\
            \cmark                                       & NBT QL Score           & string            & NBT                                                  \\
            \cmark                                       & NBT Math Score         & string            & NBT                                                  \\
            \cmark                                       & RegAcadYear            & number            & First registration at UCT                            \\
            \bottomrule
        \end{tabularx}
        \scriptsize
        \begin{tablenotes}
            \item[\textsuperscript{1}]Attributes were included via a white-listing process using Microsoft Excel
            \item[\textsuperscript{3}]Anonymization was performed by Associate Professor Sonia Berman
        \end{tablenotes}
    \end{threeparttable}
\end{table}
\input{3-data/figures/fig-benchmarks-csv-sample}

\section{Events}
The CSV export received for Events data is for 2016, with a description of the fields shown in  Table \ref{tbl-data-events}. The Events CSV export cannot be opened using Microsoft Excel due to its large size (\textgreater 5GB). This CSV contains no header rows, and information on the headers had to be obtained separately. A small sample of the CSV was taken using a BASH terminal - the \mintinline{bash}{head -n 100 <file.csv>} will show the first 100 lines of a CSV. A sample of this CSV file is shown in Figure \ref{fig-events-csv-sample}.

\begin{table}[H]
    \begin{threeparttable}
        \textbf{Table \ref{tbl-data-events}}\par\medskip\par\medskip
        \caption[Events Data CSV]{Fields received in the CSV export of Events data}
        \label{tbl-data-events}
        \begin{tabularx}{\textwidth}{>{\hsize=0.8\hsize}X>{\hsize=0.6\hsize}X>{\hsize=1.6\hsize}X}
            \toprule
            \mC{c}{Field Name} & \mC{c}{Data type} & \mC{c}{Description}                                             \\
            \midrule
            event\_date        & date              & yyyy-mm-dd hh:mm:ss                                             \\
            event\_id          & number            & Number\tnote{\textsuperscript{1}}                               \\
            uct\_id            & number            & Anonymized student number\tnote{\textsuperscript{2}}            \\
            site\_key          & number            & Foreign key reference to course site\tnote{\textsuperscript{3}} \\
            ref                & string            & Detail of event                                                 \\
            \bottomrule
        \end{tabularx}
        \scriptsize
        \begin{tablenotes}
            \item[\textsuperscript{1}]This column represents a foreign key, allowing for a more general classification of events
            \item[\textsuperscript{2}]Anonymization of the event data was performed by Stephen Marquard
            \item[\textsuperscript{3}]This column represents a foreign key, allowing each event to be associated with a particular course
        \end{tablenotes}
    \end{threeparttable}
\end{table}
\begin{figure}[H]
  \centering
  \begin{mdframed}
    \centering
\begin{BVerbatim}

+------------------+----------+--------+----------+-----+
|    event_date    | event_id | uct_id | site_key | ref |
+------------------+----------+--------+----------+-----+
| 2/18/2016 17:12  |      281 |      1 |    50680 | txt |
| 10/18/2016 17:12 |      281 |      1 |    27933 | txt |
| 5/18/2016 17:12  |      281 |      2 |    50680 | txt |
| 5/19/2016 17:12  |      281 |      2 |    27933 | txt |
| 9/18/2016 17:12  |      281 |      2 |    27933 | txt |
| 4/18/2016 17:12  |      281 |      3 |    50680 | txt |
| 4/18/2016 17:12  |      281 |      3 |    27933 | txt |
| 5/18/2016 17:12  |      281 |      3 |    27933 | txt |
| 6/18/2016 17:12  |      281 |      3 |    31585 | txt |
| 11/18/2016 17:12 |      281 |      4 |    31585 | txt |
| 11/18/2016 17:12 |      281 |      5 |    31585 | txt |
+------------------+----------+--------+----------+-----+

\end{BVerbatim}
  \end{mdframed}
  \caption[Raw Events CSV Sample]{\textbf{Figure \ref{fig-events-csv-sample}: Sample of Events CSVs used in the ETL process.} A header row has been added to this sample for readability purposes, but the nETL application loaded a CSV without this header row. The ``ref'' values have been shortened to ``txt'' since these values are long and unused in the analysis. Also, although the Events CSV shows event\_ids of only 281, the CSV used in the ETL process contains many values for this field.}
  \label{fig-events-csv-sample}
\end{figure}
