With special thanks to Jane Hendry (UCT's CIO) and Stephen Marquard (the Learning Technologies Coordinator from the Center for Innovation in Learning and Teaching at UCT), student data was made available encompassing UCT entrance assessments, course grades, and LMS (Sakai) interactions. This data was received as three separate datasets that are classified in terms of this project as listed below:

\begin{itemize}
    \item \textit{Grades}: Student course results provided by Jane Hendry
    \item \textit{Benchmarks}: Student matric results and admission-acceptance test results provided by Jane Hendry
    \item \textit{Events}: Student interactions with the Sakai platform, measured as browser interactions. This is was also provided by Stephen Marquard
\end{itemize}

All three datasets were received as CSV exports with student IDs anonymized prior to receiving them thanks to Associate Professor, Sonia Berman and Stephen Marquard. Anonymization was consistent across all three datasets, preserving the association of particular students with correct Benchmarks, Grades and Events. Some Excel manipulation on the files was required on the Grades and Benchmarks CSVs (discussed below). Figure \ref{fig-sample-csv-files} shows a sample of each CSV type as used in the analysis.

\section{Grades}
CSV exports received for Grade data for the years 2014, 2015 and 2016 are consistent across all three years in terms of the fields and the ordering of these fields - the only structural difference is that the 2014 CSV is tab-delimited instead of comma delimited. Using Microsoft Excel the three files were combined into a single CSV, which is described in Table \ref{tbl-data-grades}; all the fields included in the CSV exports are included in the combined CSV that is used as a the data source for this project's analysis.

\section{Benchmarks}
CSV exports received for Benchmark contain data for the years 2014, 2015 and 2016. The fields in these CSVs are not consistent across all three years; certain fields are capitalized differently, fields are ordered differently from year to year. Additionally, fields are included to show UCT academic performance for years subsequent to each student's registration. For example, Benchmark data from 2014 includes academic performance for the years 2015/2016, and the 2015 Benchmark data includes academic performance for the year 2016 - resulting in many repeating fields.

CouchDB is an excellent tool for normalizing schemas, since user defined map functions can query every document with unique logic according schema version and output a normalized document representation. But although CouchDB is lenient in terms of data structure when modeling data (JSON documents are semi-structured), some structuring beyond a 'visual' structure is still required. As such, the Benchmark data as received in CSV form could be broadly classified as 'unstructured'; Microsoft Excel was used to normalize the Benchmark data across all three years, since this tool provides a means for processing data where structure relies on visual organization. In order to take advantage of CouchDB as a means of `flattening' inconsistent schema's, different CSV exports to as received would be required from the system housing the Benchmark and other demographic data. Steps taking during the normalization process are listed here:

\begin{itemize}
    \item Adjusting fields to be spelled the same, normalizing whitespace, and making capitalization consistent for fields with the same name
    \item Removing duplicated fields for years subsequent to student's enrollment date
    \item Removing data not considered ethical to work with - race, gender, etc.
\end{itemize}

The normalized Benchmark data comprises a single CSV containing an anonymized list of students that enrolled at UCT in 2014, 2015, and 2016 and the benchmark results of these students. A description of this CSV is shown in Table \ref{tbl-data-benchmarks} in terms of field names, a description of these fields and each field's data type.

\section{Events}
The CSV export received for Events data is for 2016, with a description of the fields shown in \ref{tbl-data-events}. The Events CSV export cannot be opened using Microsoft Excel due to it's large size (\textgreater 5GB) and was processed using nETL. This CSV contains no header rows, and information on the headers had to be obtained separately. A small sample of the CSV was taken using a BASH terminal - the \mintinline{bash}{head -n 100 <file.csv>} will show the first 100 lines of a CSV.

\newpage
\begin{figure}[H]
    \centering
    \begin{mdframed}[rightline=false,leftline=false]
        \centering
        \begin{BVerbatim}[fontsize=\tiny]

// Grade CSV test sample
+--------+-------------+---------+----+------------+-----------+--------+-------------+---------+----------+----------+--------+--------------+--
| DnldDt | RegAcadYear | RegTerm | ID | RegProgram | RegCareer | Degree | DegreeDescr | Subject | Catalog. |  Course  | Suffix |   Session    |
+--------+-------------+---------+----+------------+-----------+--------+-------------+---------+----------+----------+--------+--------------+--
| txt    |        2016 |    1161 |  1 | CB003      | UGRD      | QCB007 | txt         | MAM     | 1000W    | MAM1000W | W      | Full Year    |
| txt    |        2016 |    1161 |  1 | CB003      | UGRD      | QCB007 | txt         | CSC     | 1015F    | CSC1015F | F      | Semester One |
| txt    |        2016 |    1161 |  2 | CB004      | UGRD      | QCB002 | txt         | CSC     | 1015F    | CSC1015F | F      | Semester One |
| txt    |        2016 |    1161 |  3 | EB022      | UGRD      | QEB028 | txt         | EEE     | 2036S    | EEE2036S | S      | Semester Two |
| txt    |        2015 |    1151 |  3 | EB022      | UGRD      | EB28   | txt         | CSC     | 1015F    | CSC1015F | F      | Semester One |
| txt    |        2016 |    1161 |  4 | CB004      | UGRD      | QCB002 | txt         | CSC     | 1015F    | CSC1015F | F      | Semester One |
| txt    |        2015 |    1151 |  4 | CB003      | UGRD      | CB07   | txt         | CSC     | 1015F    | CSC1015F | F      | Semester One |
| txt    |        2015 |    1151 |  5 | CB003      | UGRD      | CB07   | txt         | CSC     | 1015F    | CSC1015F | F      | Semester One |
+--------+-------------+---------+----+------------+-----------+--------+-------------+---------+----------+----------+--------+--------------+--
--+---------+--------+------------+----------+------------+-----------+---------+------+--------------+-----------+-----------+------+----------+
  | Percent | Symbol | UnitsTaken | CourseID | CourseDesc | CourseCrr | Faculty | Dept | MaxCrseUnits | CrseCount | CrseLevel | CESM | Sub-CESM |
--+---------+--------+------------+----------+------------+-----------+---------+------+--------------+-----------+-----------+------+----------+
  |      71 | 2+     |         36 |   107088 | txt        | UGRD      | SCI     | MAM  |           36 |         1 |        41 | 1501 |        1 |
  |      70 | 2+     |         18 |   103034 | txt        | UGRD      | SCI     | CSC  |           18 |       0.5 |        41 |  606 |        6 |
  |      55 | 3      |         18 |   103034 | txt        | UGRD      | SCI     | CSC  |           18 |       0.5 |        41 |  606 |        6 |
  |      63 | 2-     |         12 |     1642 | txt        | UGRD      | EBE     | EEE  |           12 |         1 |        42 |  809 |        9 |
  |      77 | 1      |         18 |   103034 | txt        | UGRD      | SCI     | CSC  |           18 |       0.5 |        41 |  606 |        6 |
  |      54 | 3      |         18 |   103034 | txt        | UGRD      | SCI     | CSC  |           18 |       0.5 |        41 |  606 |        6 |
  |      39 | F      |         18 |   103034 | txt        | UGRD      | SCI     | CSC  |           18 |       0.5 |        41 |  606 |        6 |
  |      39 | F      |         18 |   103034 | txt        | UGRD      | SCI     | CSC  |           18 |       0.5 |        41 |  606 |        6 |
--+---------+--------+------------+----------+------------+-----------+---------+------+--------------+-----------+-----------+------+----------+

// Event CSV test sample
+------------------+----------+--------+----------+-----+
|    event_date    | event_id | uct_id | site_key | ref |
+------------------+----------+--------+----------+-----+
| 2/18/2016 17:12  |      281 |      1 |    50680 | txt |
| 10/18/2016 17:12 |      281 |      1 |    27933 | txt |
| 5/18/2016 17:12  |      281 |      2 |    50680 | txt |
| 5/19/2016 17:12  |      281 |      2 |    27933 | txt |
| 9/18/2016 17:12  |      281 |      2 |    27933 | txt |
| 4/18/2016 17:12  |      281 |      3 |    50680 | txt |
| 4/18/2016 17:12  |      281 |      3 |    27933 | txt |
| 5/18/2016 17:12  |      281 |      3 |    27933 | txt |
| 6/18/2016 17:12  |      281 |      3 |    31585 | txt |
| 11/18/2016 17:12 |      281 |      4 |    31585 | txt |
| 11/18/2016 17:12 |      281 |      5 |    31585 | txt |
+------------------+----------+--------+----------+-----+

// Benchmark CSV test sample
+----+-------------+-----------------------+------------------------------+-----------+------------+---------------+--
| ID |   Career    | Citizenship Residency |          SA School           | Eng Grd12 | Math Grd12 | Mth Lit Grd12 |
+----+-------------+-----------------------+------------------------------+-----------+------------+---------------+--
|  1 | UGRD        | SA Citizen            | St Anne's College            |        84 |         88 |               |
|  2 | UGRD        | SA Citizen            | St Andrews Girls' School     |        76 |         78 |               |
|  3 | Second Year | C                     | Clifton College              |        75 |         78 |               |
|  4 | Second Year | C                     | Crawford North Coast College |        82 |         85 |               |
|  5 | Second Year | C                     | Ignore                       |        82 |         85 |               |
+----+-------------+-----------------------+------------------------------+-----------+------------+---------------+--
--+---------------+---------------+--------------+--------------+----------------+-------------+
  | Adv Mth Grd12 | Phy Sci Grd12 | NBT AL Score | NBT QL Score | NBT Math Score | RegAcadYear |
--+---------------+---------------+--------------+--------------+----------------+-------------+
  |               |            94 |           80 |           76 |             89 |        2016 |
  |               |            76 |           75 |           58 |             61 |        2016 |
  |               |            84 |            0 |            0 |              0 |        2015 |
  |               |            94 |           73 |           71 |             86 |        2015 |
  |               |            94 |           73 |           71 |             86 |        2016 |
--+---------------+---------------+--------------+--------------+----------------+-------------+

        \end{BVerbatim}
    \end{mdframed}
    \caption[CSV Samples]{\textbf{Figure \ref{fig-sample-csv-files}: Sample of data CSVs used for analysis and, the actual CSV data used for testing.} The header names are slightly adjusted for this figure, however, to fit all the columns into this figure. Additionally, the Events CSV as used in the project does not contain a header row - but this is impractical for display purposes and so although the CSV was processed without a header row, a header row has been included in this figure. Some fields that are not relevant to the study have been redacted to "txt" to shorten these fields also for display purposes. Although the Events CSV shows event\_ids of only 281, the CSV contains many different values for this field}
    \label{fig-sample-csv-files}
\end{figure}


\newpage
\begin{table}[h]
    \begin{threeparttable}
        \textbf{Table \ref{tbl-sakai-grades}}\par\medskip\par\medskip
        \caption[Sakai grade data]{A description of the Sakai grade data as received in CSV format, and how these fields were treated in the ETL and analysis process}
        \label{tbl-sakai-grades}
        \begin{tabularx}{\textwidth}{>{\hsize=0.4\hsize}Y>{\hsize=1\hsize}X>{\hsize=0.7\hsize}X>{\hsize=1.9\hsize}X}
            \toprule
            \mC{c}{Include\tnote{\textsuperscript{1}}  } & \mC{c}{Field Name} & \mC{c}{Data type} & \mC{c}{Description}                                    \\
            \midrule
            \xmark                                       & DownloadedDate     & date              & Excel date format                                      \\
            \cmark                                       & RegAcadYear        & number            & Year                                                   \\
            \xmark                                       & RegTerm            & number            & Integer ID                                             \\
            \cmark                                       & anonIDnew          & number            & Anonymized student number                              \\
            \xmark                                       & RegProgram         & string            & Program abbreviation                                   \\
            \xmark                                       & RegCareer          & string            & Academic level\tnote{\textsuperscript{3}}              \\
            \xmark                                       & Degree             & string            & Degree code                                            \\
            \xmark                                       & DegreeDescr        & string            & Degree title                                           \\
            \xmark                                       & Subject            & string            & Three letter abbreviation                              \\
            \xmark                                       & Catalog.           & string            & Catalog sub-component (number and session)             \\
            \cmark                                       & Course             & string            & Full course code description                           \\
            \xmark                                       & CourseSuffix       & string            & Course session identifier  \tnote{\textsuperscript{4}} \\
            \xmark                                       & Session            & string            & Session name..                                         \\
            \cmark                                       & Percent            & string            & Grade achieved by student \tnote{\textsuperscript{5}}  \\
            \xmark                                       & Symbol             & string            & Symbol achieved by student                             \\
            \xmark                                       & UnitsTaken         & number            & Total units taken by student?                          \\
            \xmark                                       & CourseID           & number            & Numerical course Identifier                            \\
            \xmark                                       & CourseDescr        & string            & Description of course                                  \\
            \xmark                                       & CourseCareer       & string            & Academic level of course \tnote{\textsuperscript{6}}   \\
            \xmark                                       & Faculty            & string            & Course faculty                                         \\
            \xmark                                       & Dept               & string            & Faculty department                                     \\
            \xmark                                       & MaximumCrseUnits   & number            & ??                                                     \\
            \xmark                                       & CourseCount        & number            & ??                                                     \\
            \xmark                                       & CourseLevel        & number            & ??                                                     \\
            \xmark                                       & CESM               & number            & ??                                                     \\
            \xmark                                       & Sub-CESM           & number            & ??                                                     \\
            \bottomrule
        \end{tabularx}
        \scriptsize
        \begin{tablenotes}
            \item[\textsuperscript{1}]Attributes were included via a white-listing process.
            \item[\textsuperscript{3}]Only courses taken by students registered as undergraduates (UGRD) were considered in this study.
            \item[\textsuperscript{4}] A wide variety of course suffixes are present in the database, including: D,F,H,L,M,N,P,Q,R,S,W,X,Z.
            \item[\textsuperscript{5}]The percent field included both numbers and abbreviations that need to be normalized so that this field can be treated as always holding a numerical value. A table showing how the different suffixes and abbreviations of the percent field is shown in \ref{tbl-sakai-grades-percent}
            \item[\textsuperscript{6}]Only courses at undergraduate level (UGRD) were considered for this project.
        \end{tablenotes}
    \end{threeparttable}
\end{table}












\begin{table}[h]
    \begin{threeparttable}
        \textbf{Table \ref{tbl-sakai-grades-percent}}\par\medskip\par\medskip
        \caption{Grade results need to be treated as numbers for the purpose of this analysis, this table shows all different value types and the appropriate treatment for each. Because of the volume of data, it was not checked how many of these symbols apply to undergraduate students specifically, so these cases were handled generically}
        \label{tbl-sakai-grades-percent}
        \begin{tabularx}{\textwidth}{>{\hsize=0.6\hsize}X>{\hsize=1.3\hsize}X>{\hsize=1.1\hsize}X}
            \toprule
            \mC{c}{Symbol} & \mC{c}{Meaning}          & \mC{c}{Handling Logic}                     \\
            \midrule
            49A            & Absent for supplementary & Grade used                                 \\
            49S            & Supplementary pending    & Grade used                                 \\
            50C            & ?                        & Grade used                                 \\
            78             & Grade                    & Grade used                                 \\
            AB             & Absent (fail)            & 30\% Grade used\tnote{\textsuperscript{1}} \\
            ATT            & ?                        & N/A                                        \\
            DE             & Deferred                 & N/A                                        \\
            DPR            & Duly performed refused   & 20\% Grade used\tnote{\textsuperscript{2}} \\
            F              & Fail                     & 40\% Grade used\tnote{\textsuperscript{3}} \\
            GIP            & Thesis only              & N/A                                        \\
            INC            & Incomplete (fail)        & 20\% Grade used\tnote{\textsuperscript{2}} \\
            LOA            & Leave of absence         & N/A                                        \\
            OS             & Outstanding              & N/A                                        \\
            OSS            & Outstanding              & N/A                                        \\
            PA             & Pass (thesis)            & N/A                                        \\
            SAT            & Thesis only              & N/A                                        \\
            UF             & Unclassified Fail        & 30\% Grade used\tnote{\textsuperscript{1}} \\
            UNS            & Thesis only              & N/A                                        \\
            UP             & Unclassified pass        & 50\% Grade used\tnote{\textsuperscript{4}} \\
            \bottomrule
        \end{tabularx}
        \scriptsize
        \begin{tablenotes}
            \item[\textsuperscript{1}]30\% was applied on the estimate that failing in this case was slightly 'worse' than a regular fail.
            \item[\textsuperscript{2}]20\% was applied on the estimate that these students wouldn't necessarily have completed the coursework.
            \item[\textsuperscript{3}]40\% was applied on the estimate that this symbol would apply to students who participated in the course.
            \item[\textsuperscript{4}]50\% was applied on the estimate that these students passed without distinction or by concession.
        \end{tablenotes}
    \end{threeparttable}
\end{table}

\newpage
\begin{table}[H]
    \begin{threeparttable}
        \textbf{Table \ref{tbl-data-benchmarks}}\par\medskip\par\medskip
        \caption[Benchmark Data CSV]{Fields received in the CSV export of Benchmark data}
        \label{tbl-data-benchmarks}
        \begin{tabularx}{\textwidth}{>{\hsize=0.3\hsize}Y>{\hsize=1.3\hsize}X>{\hsize=0.6\hsize}X>{\hsize=1.8\hsize}X}
            \toprule
            \mC{c}{Include\tnote{\textsuperscript{1}}  } & \mC{c}{Field Name}     & \mC{c}{Data type} & \mC{c}{Description}                                  \\
            \midrule
            \cmark                                       & anonIDnew              & number            & Anonymized student number\tnote{\textsuperscript{2}} \\
            \xmark                                       & Career                 & string            & Academic level                                       \\
            \xmark                                       & Citizenship Residency  & string            & Student citizenship                                  \\
            \xmark                                       & SA School              & string            & School name (if in RSA)                              \\
            \cmark                                       & Eng Grd12 Fin Rslt     & string            & Grade 12 English result                              \\
            \cmark                                       & Math Grd12 Fin Rslt    & string            & Grade 12 Math result                                 \\
            \cmark                                       & Mth Lit Grd12 Fin Rslt & string            & Grade 12 Math Literacy result                        \\
            \cmark                                       & Adv Mth Grd12 Fin Rslt & string            & Grade 12 Advanced Math result                        \\
            \cmark                                       & Phy Sci Grd12 Fin Rslt & string            & Grade 12 Science result                              \\
            \cmark                                       & NBT AL Score           & string            & National benchmark test (NBT)                        \\
            \cmark                                       & NBT QL Score           & string            & NBT                                                  \\
            \cmark                                       & NBT Math Score         & string            & NBT                                                  \\
            \cmark                                       & RegAcadYear            & number            & First registration at UCT                            \\
            \bottomrule
        \end{tabularx}
        \scriptsize
        \begin{tablenotes}
            \item[\textsuperscript{1}]Attributes were included via a white-listing process using Microsoft Excel
            \item[\textsuperscript{3}]Anonymization was performed by Associate Professor Sonia Berman
        \end{tablenotes}
    \end{threeparttable}
\end{table}
\vspace{80px}
\begin{table}[H]
    \begin{threeparttable}
        \textbf{Table \ref{tbl-data-events}}\par\medskip\par\medskip
        \caption[Events Data CSV]{Fields received in the CSV export of Events data}
        \label{tbl-data-events}
        \begin{tabularx}{\textwidth}{>{\hsize=0.8\hsize}X>{\hsize=0.6\hsize}X>{\hsize=1.6\hsize}X}
            \toprule
            \mC{c}{Field Name} & \mC{c}{Data type} & \mC{c}{Description}                                             \\
            \midrule
            event\_date        & date              & yyyy-mm-dd hh:mm:ss                                             \\
            event\_id          & number            & Number\tnote{\textsuperscript{1}}                               \\
            uct\_id            & number            & Anonymized student number\tnote{\textsuperscript{2}}            \\
            site\_key          & number            & Foreign key reference to course site\tnote{\textsuperscript{3}} \\
            ref                & string            & Detail of event                                                 \\
            \bottomrule
        \end{tabularx}
        \scriptsize
        \begin{tablenotes}
            \item[\textsuperscript{1}]This column represents a foreign key, allowing for a more general classification of events
            \item[\textsuperscript{2}]Anonymization of the event data was performed by Stephen Marquard
            \item[\textsuperscript{3}]This column represents a foreign key, allowing each event to be associated with a particular course
        \end{tablenotes}
    \end{threeparttable}
\end{table}