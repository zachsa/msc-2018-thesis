\begin{table}[h]
    \begin{threeparttable}
        \textbf{Table \ref{tbl-sakai-events}}\par\medskip\par\medskip
        \caption[Sakai event data]{A description of the Sakai event data as received in CSV format, and how these fields were treated in the ETL and analysis process}
        \label{tbl-sakai-events}
        \begin{tabularx}{\textwidth}{>{\hsize=0.5\hsize}Y>{\hsize=0.8\hsize}X>{\hsize=0.8\hsize}X>{\hsize=1.9\hsize}X}
            \toprule
            \mC{c}{Include\tnote{\textsuperscript{1}}  } & \mC{c}{Field Name} & \mC{c}{Data type} & \mC{c}{Description}                                  \\
            \midrule
            \cmark                                       & event\_date        & date              & yyyy-mm-dd hh:mm:ss                                  \\
            \cmark                                       & event\_id          & number            & Number\tnote{\textsuperscript{2}}                    \\
            \cmark                                       & uct\_id            & number            & Anonymized student number\tnote{\textsuperscript{3}} \\
            \cmark                                       & site\_key          & number            & Foreign key reference to course site                 \\
            \xmark                                       & ref                & string            & Detail of event                                      \\
            \bottomrule
        \end{tabularx}
        \scriptsize
        \begin{tablenotes}
            \item[\textsuperscript{1}]Attributes were included via a white-listing process.
            \item[\textsuperscript{2}]This column represents a foreign key, pointing to a table of event types. filtering on this column cut down applicable rows from 44.4 million rows to just over 13.5 million rows.
            \item[\textsuperscript{3}]Anonymization of the event data was performed by Stephen Marquard.
        \end{tablenotes}
    \end{threeparttable}
\end{table}