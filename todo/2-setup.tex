\subsection{Setting up CouchDB}
CouchDB 2.0 onwards implements clustering functionality that makes large-scale, sharded databases feasible. For the purposes of this project, clustering was tested on 3 Hetzner servers CX20 servers with indexing tested first locally on a Windows machine and then on cheap CX20 servers. The commands to provision Ubuntu Xenial and CouchDB 2.1 have been produced below. For the purposes of this project, these commands are executed via a Ruby script so as to automate the server provisioning process for any number of (existing) virtual servers. Hetzner, Digital Ocean, Google, AWS and many other cloud providers make an API available that allows for automated provisioning of \textit{n}-virtual servers, where \textit{n} is limited only by one's ability to pay!

Since virtual servers are relatively inexpensive in comparison to dedicated servers, and easier to setup, dispersed data indexing is extremely-achievable using CouchDB. CouchDB MapReduce queries could easily be dispersed among thousands of computer nodes, as could massive amounts of data. In terms of database size limits, it would theoretically be as easy working with several PB of data and 10 000 + CouchBD nodes as it is working with a small cluster.The Ruby script is called 'rCluster', with code available at \url{https://github.com/zachsa/rcluster}.

Since CouchDB 2.0 onwards is designed primarily with clustering in mind, by default databases are created with several (8) shards. A database with multiple shards per a single node is considered to be \textit{oversharded}, which is in itself fine. For the purposes of this project the default of 8 shards (\textit{q=8} in Couch-speak) is used across all servers (since the maximum number of nodes tested on is 6).

A summary of the setup script is showin in \ref{appendix:couch-setup}; this script was used to create a linux cluster for testing. Most of the work was done on a Windows machine, however, where such an elaborate install procedure is not required; there is binary available specifically for Windows.