\section{Setup}
Running nETL requires an installation of node.js V8.9.0 +, which should include an installation of npm. After cloning the netl repository from Github to a local drive, dependencies should be restored using the npm CLI tool. Additionally, the directory $C:\log$ needs to be created, and then the nETL app can be started from a terminal. Once the CLI is running, typing anything into the terminal and pressing enter outputs help where further direction can be obtained.

In conjunction with setting up nETL, a CouchDB server needs to be configure. This is easy to do on Windows machines - simply download the executable from apache.org and use the installer. Once installed the server should be run in single node configuration, binded to the 127.0.0.1 address. This allows access to the CouchDB UI via the browser at the address: http://127.0.0.1:5985/\_utils, where first an admin user should be created. Working with databases via the CouchDB interface (called Fauxton) is straightforward.

Database creation involves only the single step of specifying a name and (optionally) security roles. CouchDB database configuration should be specified as part of creation - though this is only available when databases creation is specified via the HTTP interface and not the Fauxton GUI; examples of configurable settings are sharding (q) and replica (n) count, the default of which are configured at the server level. For a single node setup (such as used in this project), q = 8 and n = 1, meaning that a database has 8 shards and only 1 replica of each shard. This is the configuration used in this project; sharding on a single server allows for CouchDB to utilize processes for index calculation. There is not point in storing more than one copy of a single shard on a single server, which is why n = 1 by default for single node CouchDB server.

For CouchDBs operation in cluster mode the default setup is q = 8 and n = 3. For clusters with a large number of nodes it might make sense to increase the value of the q parameter.