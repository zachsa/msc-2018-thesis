

\begin{table}[h]
    \begin{threeparttable}
        \textbf{Table \ref{tbl-fu}}\par\medskip\par\medskip
        \caption[FU data]{A description of the FU demographic data as received in CSV format, and how these fields were treated in the ETL and analysis process}
        \label{tbl-fu}
        \begin{tabularx}{\textwidth}{>{\hsize=0.4\hsize}Y>{\hsize=1\hsize}X>{\hsize=0.7\hsize}X>{\hsize=1.9\hsize}X}
            \toprule
            \mC{c}{Include\tnote{\textsuperscript{1}}  } & \mC{c}{Field Name}     & \mC{c}{Data type} & \mC{c}{Description}                                  \\
            \midrule
            \cmark                                       & anonIDnew              & number            & Anonymized student number\tnote{\textsuperscript{2}} \\
            \xmark                                       & Career                 & string            & Academic level\tnote{\textsuperscript{3}}            \\
            \xmark                                       & Citizenship Residency  & string            & Student citizenship \tnote{\textsuperscript{4}}      \\
            \xmark                                       & SA School              & string            & School name (if in RSA)                              \\
            \cmark                                       & Eng Grd12 Fin Rslt     & string            & Grade 12 English result                              \\
            \cmark                                       & Math Grd12 Fin Rslt    & string            & Grade 12 Math result                                 \\
            \cmark                                       & Mth Lit Grd12 Fin Rslt & string            & Grade 12 Math Literacy result                        \\
            \cmark                                       & Adv Mth Grd12 Fin Rslt & string            & Grade 12 Advanced Math result                        \\
            \cmark                                       & Phy Sci Grd12 Fin Rslt & string            & Grade 12 Science result                              \\
            \cmark                                       & NBT AL Score           & string            & National benchmark test (NBT)                        \\
            \cmark                                       & NBT QL Score           & string            & NBT                                                  \\
            \cmark                                       & NBT Math Score         & string            & NBT                                                  \\
            \cmark                                       & RegAcadYear            & number            & First registration at UCT                            \\
            \bottomrule
        \end{tabularx}
        \scriptsize
        \begin{tablenotes}
            \item[\textsuperscript{1}]Attributes were included via a white-listing process.
            \item[\textsuperscript{3}]Anonymization was performed by Associate Professor Sonia Berman.
            \item[\textsuperscript{3}]Filtered for values of  "UGRD", "First Year", "Second Year" and "Third Year".
            \item[\textsuperscript{4}]Filtered for values of "F", "S", "H", and "W".
        \end{tablenotes}
    \end{threeparttable}
\end{table}

\begin{table}[h]
    \begin{threeparttable}
        \textbf{Table \ref{tbl-fu-grades-percent}}\par\medskip\par\medskip
        \caption{Logic for transposing grade symbols to numbers for student benchmark data}
        \label{tbl-fu-grades-percent}
        \begin{tabularx}{\textwidth}{>{\hsize=0.6\hsize}X>{\hsize=1.3\hsize}X>{\hsize=1.1\hsize}X}
            \toprule
            \mC{c}{Symbol} & \mC{c}{Meaning}             & \mC{c}{Handling Logic}  \\
            \midrule
            A*             & The highest mark achievable & A grade of 90\% is used \\
            A              &                             & A grade of 80\% is used \\
            B              &                             & A grade of 70\% is used \\
            C              &                             & A grade of 60\% is used \\
            D              &                             & A grade of 55\% is used \\
            E              & The lowest pass achievable  & A grade of 50\% is used \\
            \bottomrule
        \end{tabularx}
    \end{threeparttable}
\end{table}