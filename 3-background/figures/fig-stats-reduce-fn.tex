\begin{figure}[H]
  \begin{minted}{text}
// A
{["key"]: [1,1,0]} // (_id: x)
{["key"]: [3,1,3]} // (_id: y)
{["key2"]: [2,2,2]} // (_id: z)

// B
{["key"]: [[1,1,0],[3,1,3]]}
{["ke2"]: [[2,2,2]]}

// C
reduce([["key", "x"], ["key", "y"]], [[1,1,0],[3,1,3]], false)
reduce([["key2", "z"]], [[2,2,2]], false)

// D
{["key"]: [aggregate([1,3]), aggregate([1,1]),  aggregate([0,3])]}
{["key2"]: [aggregate([2]), aggregate([2]),  aggregate([2])]}

// E
{
  ["key"]: [
    {"sum":4,"count":2,"min":1,"max":3,"sumsqr":10},
    {"sum":2,"count":2,"min":1,"max":1,"sumsqr":2},
    {"sum":3,"count":2,"min":0,"max":3,"sumsqr":9}
  ],
  ["key2"]: [
    {"sum":2,"count":1,"min":2,"max":2,"sumsqr":4}
  ]
}
    \end{minted}
  \caption[\textit{\_stats} function contract]{\textbf{Figure \ref{fig-stats-reduce-fn}: Reduce function contract.} (A): Output of the map function. [``key''] is the key for which a value is emitted (in this case, they key is a compound key but with only one index). The value in this case is a tuple of [x, y, z]. (B): Results from the Map function are grouped by a common key, with values grouped into a list - in this case a list of lists since the value output of the map function is a list. (C): The signature with which the reduce function is called with corresponding arguments. (D): The \textit{\_stats} function then further groups values by corresponding indexes and aggregates these values. (E): The result of the \textit{\_stats} function aggregation.}
  \label{fig-stats-reduce-fn}
\end{figure}