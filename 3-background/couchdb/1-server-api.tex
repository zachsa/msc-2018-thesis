\subsection{Server API}
CouchDB allows a user to specify several different types of functions that are executed on the server-side Erlang application. These functions need to be specified on a 'Design' document, which is simply a normal JSON document with an attribute "\_id" having a value of "\_design/documentName". There are 6 types of functions that a user can specify and are executed by the CouchDB application on the server: \textit{views}, \textit{shows}, \textit{lists}, \textit{updates}, \textit{filters}, and \textit{validation} checks. A full list of the server API can be seen in the sample design document in \ref{couchdb-design-doc-sample}. Via this API much of the functionality that you typically find in an RDBMS can be implemented in couchDB (albeit in a roundabout way). For example a SQL trigger could be implemented via a CouchDB \textit{update}, a SQL query could be implemented as a \textit{list} function and a \textit{view} index, etc. While the full range of data-interactions available on an RDBMS are available via CouchDB, however, to work with such a different DBMS requires a vastly different mindset when compared to working with a RDBMS since there is very little similarity between these systems.

\subsubsection{Views}
Views comprise two components - a \textit{Map}-function component and a \textit{Reduce}-function component. The logic of these components can be specified by users as separate \textit{Map} and \textit{Reduce} functions - the contracts of which are shown in the appendix (see \ref{couchdb-mapreduce-contracts}). Such functions can be specified in a variety of languages including JavaScript and are executed via an engine external to the main CouchDB Erlang process. \textit{Map} functions must be specified by a user and are always executed external to the main Erlang process via marshalling between the main Erlang process and the view-calculating engine. \textit{Reduce} functions, however, can either by executed externally to the main Erlang process (as a custom reduce function) or within the main Erlang process via one of three built in functions: \_count, \_sum, and \_stats. These functions are presumably see enough common functionality between different types of queries that it was worthwhile implementing these functions within the main Erlang process, which according to the documentation offers a performance boost since the IO transfer cost between the Erlang process and the view engine (couchjs.exe by default) is negated. Working on a Windows machine the IO cost is apparently exaggerated (see the slack correspondence with Jan Lehnardt in appendix \ref{slack-1-nov}) due to the difference between Unix-based and Windows kernel implementations.

\subsubsection{Lists}
List functions fall under the category of \textit{CouchApps} and, as mentioned previously, are likely to be excluded from future releases. These functions iterate a view output to an HTTP(S) client, allowing a final phase of transformation - i.e. a data format switch from JSON documents to CSV format. These  functions take the name of a \textit{MapReduce} view as an argument and iterate through the results, passing those results to a user over HTTP(S). As with all of CouchDB interactions, list functions are invoked via HTTP and allow for an API of retrieving data from CouchDB views. As mentioned in \ref{slack-2-nov}, list functions are deprecated and should be replaced by code external to the CouchDB application. This would not be very challenging, and software such as the bespoke ETL tool written for this project (\textit{nETL}) could be easily configured to replace list functions with a small amount of coding. But since \textit{List} functions are likely to enjoy several more years of support in the CouchDB current release and possibly in the future, they are still a useful tool and are relevant enough that there use is not unwarranted in this project.

\subsubsection{Shows, Updates, Filters and Validation}
todo

\subsubsection{Inserting and retrieving documents}

todo:
\textit{\_bulk\_docs}