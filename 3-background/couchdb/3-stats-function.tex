\subsubsection{The \textit{\_stats} Function}
The \textit{\_stats} function signature requires that map output adhere to the following constraints: that values are either a single number or a list of numbers, that values (if lists) should be of the same length, and that indexes in value-lists each correlate with each other. Output of the map function is then grouped per key, and reduction involves aggregating values at corresponding indexes. Figure \ref{stats-reduce-fn} describes the contract of the \textit{\_stats} reduce function via representing how data is structured during index calculation at the mapping stage (A), the grouping stage (B), the signature with which the reduce function is called for this particular example (C), how the \textit{\_stats} function is applied to the grouped values (D), and the output of the \textit{\_stats} function (E).

Output of the \textit{\_stats} function includes a count of how many items are included in the aggregation, the minimum value, the maximum value, the sum of all the values, and the sum of squares of all the values. This information overlaps with the output produced by the \textit{\_sum} and \textit{\_count} built-in reduce functions.

\begin{figure}[H]
    \begin{minted}{javascript}
// A
{["key"]: [1,1,0]} // (_id: x)
{["key"]: [3,1,3]} // (_id: y)
{["key2"]: [2,2,2]} // (_id: z)

// B
{["key"]: [[1,1,0],[3,1,3]]}
{["ke2"]: [[2,2,2]]}

// C
reduce([["key", "x"], ["key", "y"]], [[1,1,0],[3,1,3]], false)
reduce([["key2", "z"]], [[2,2,2]], false)

// D
{["key"]: [aggregate([1,3]), aggregate([1,1]),  aggregate([0,3])]}
{["key2"]: [aggregate([2]), aggregate([2]),  aggregate([2])]}

// E
{
  ["key"]: [
    {"sum":4,"count":2,"min":1,"max":3,"sumsqr":10},
    {"sum":2,"count":2,"min":1,"max":1,"sumsqr":2},
    {"sum":3,"count":2,"min":0,"max":3,"sumsqr":9}
  ],
  ["key2"]: [
    {"sum":2,"count":1,"min":2,"max":2,"sumsqr":4}
  ]
}
    \end{minted}
    \caption[\textit{\_stats} function contract]{\textbf{Figure \ref{stats-reduce-fn}: Reduce function contract.} (A): Output of the map function. [``key''] is the key for which a value is emitted (in this case, they key is a compound key but with only one index). The value in this case is a tuple of [x, y, z]. (B): Results from the Map function are grouped by a common key, with values grouped into a list - in this case a list of lists since the value output of the map function is a list. (C): The signature with which the reduce function is called with corresponding arguments. (D): The \textit{\_stats} function then further groups values by corresponding indexes and aggregates these values. (E): The result of the \textit{\_stats} function aggregation.}
    \label{stats-reduce-fn}
\end{figure}