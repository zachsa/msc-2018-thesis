\section{CouchDB}
CouchDB falls under the very general classification of ``NoSQL'' (\textit{N}ot \textit{o}nly \textit{SQL}) databases - indicating a class of databases where data modeling is not strictly limited to tabular relations. CouchDB, for example, is a document store; persisted data comprises numerous JSON strings (i.e. the `documents') with no inherent limitation or specification as to the structure of these documents (other than that each is a of valid JSON format). Compared to databases that model data via tabular relations, NoSQL databases are often referred to as being `unstructured' or `semi-structured' in contrast.

But despite different storage models, conceptual requirements of data-persistence - such as well known \textit{ACID} properties are applicable to all database systems. \textit{ACID} properties provide certain guarantees required within a system to reliably persist information. Without such guarantees a database would not be useful since the integrity of any information stored in such a system would be questionable. In terms of the CouchDB software \textit{ACID} properties are applicable at the \textit{document} level meaning that interactions with single documents are either successful or not; a partial document could not be written nor read. In general terms \textit{ACID} properties stand for the following:

\begin{itemize}
    \item \textit{A}tomicity: a series of operations part of the same logical transaction should either pass or fail completely
    \item \textit{C}onsistency: building on the concept of atomicity, the result of a single logical transaction should leave the database in a working and valid state
    \item \textit{I}solation: transactions on a database are self contained and don't interact with each other
    \item \textit{D}urability: data is persisted permanently and reliably as a result of isolated transactions that leave the database in a consistent state due to their atomicity in nature
\end{itemize}

CouchDB serializes document interactions to guarantee isolation, document writes are guaranteed to be durable and result in a consistent database. In other words, in working with CouchDB, transactions involving single documents are completely reliable and fault tolerant, but transactions involving retrieving and storing information from multiple documents are not. Single-document \textit{ACID} guarantees make multistep transactions in CouchDB more difficult. "Multi-step transactional atomicity" is a key feature for many RDBMSs including \textit{MySQL}, \textit{SQL Server}, etc. and overcoming this limitation is something that is required in order to implement NoSQL databases in traditional RDBMS environments. This is possible, as shown by \cite{Rashmi2017} for CouchDB specifically and NoSQL in general \cite{LOTFY2016133} via implementing \textit{ACID}/transactional properties as bespoke middleware externally to these DBSs. But it would seem this is not an ideal alternative to systems where RDBMS-specific features are required! This is a trade-off that CouchDB (and other NoSQL) databases make in favor of less rigorous data models that are suitable for use-cases that RDBMSs are not.

As mentioned in the online CouchDB documentation \cite{couchguide} CouchDB's unstructured data model is centered around the idea of key:value pairs (where the value component of the key is a JSON document) written to disc via an `append-only' engine. These key:value are structured as b+trees to facilitate fast data retrieval from huge databases, but implemented to incorporate a model for `Multi-Version-Concurrency-Control' (MVCC) to facilitate versioning of nodes in the tree. Aside from write operations being serialized, MVCC negates the needs for locks that are typically implemented within RDBMSs and are expensive in terms of computer resources. Without locking on read/write operations, CouchDB data stores are infinitely available with the caveat that such available documents may not always be the most recent versions of those documents.

Research in 1999 by \cite{cap} introduced the idea of the \textit{CAP} theorem as a trade-off analysis of 3 properties of a data storage system, \textit{Consistency}, \textit{Availability}, and \textit{Partition-tolerance}, when taking into account the inevitability of network downtime (all three properties ARE fully achievable without network downtime). CouchDB represents a trade-off of \textit{consistency} for \textit{availability} to facilitate scalability in terms of handling large amounts of data in near-realtime. CouchDB emphasizes \textit{availability} and \textit{eventual consistency} of databases to allow for high partition tolerance of databases with unreliable communication between nodes. From CouchDB 2.0 onwards users can configure the desired level of consistency of document reads, but not document writes.

Because of the potential for inconsistency, CouchDB seeks to provide a 'relaxed' viewing model - i.e. a \textit{soft state} where data representation is not tied to the underlying entities (an entity can be updated whilst being viewed unaware of such changes). That is, sacrificing of \textit{availability} at the expense of \textit{consistency} as per the \textit{cap} theorem; data conflicts, where entities are updated separately and independently of each other are often acceptable in NoSQL databases, particularly in systems that are required to take an `offline-first' approach to data-handling. These systems provide a means of interacting with partial datasets, that when synchronized could result in conflicting state. CouchDB's document-versioning system (MVCC) allows for handling such consistency violations and for maintaining consistency within the data storage layer itself via providing an 'audit trail' on how to resolve inconsistent information.

\subsection{Entity Modeling}
Despite moving away from the relational model as provided by RDBMSs, the concept of 'entities' is usually still highly relevant in many NoSQL databases; these database can, as such, be grouped into two categories \cite{fowlerAggregate}:

\begin{itemize}
    \item \textit{aggregate orientated} stores that model data similarly to the relational model but with isolated entity boundaries and
    \item \textit{aggregate ignorant} stores where the concept of entities is fundamentally different (e.g. a graph database such as \textit{Neo4J} where the entities are edges and nodes)
\end{itemize}

The vast majority of databases operate within a domain where data is (for the most part) entity-driven. As \cite{GANESHCHANDRA201513} points out there are hundreds of NoSQL data stores and a comprehensive categorization of such products is not sensible, but familiar examples within the family of \textit{aggregate orientated} NoSQL databases include \textit{key/value} stores such such as Amazon's \textit{Dynamo} database, column based stores such as \textit{Cassandra}, \textit{HBase} and document stores such as \textit{CouchDB}, \textit{Mongo}.

Although NoSQL databases are said to be \textit{schema-less}, \cite{ATZENI2016} points out that this is not the case; instead NoSQL databases allow for inconsistent schema representation across different entity instances due to the nature of how entities are aggregated. Such flexibility is at the heart of document stores such as CouchDB and Mongo where loose-schema modeling is one of the properties that makes such technologies suitable for large systems that generate data from inconsistent sources (i.e. a general 'survey' entity with a configurable flow).

Data models in CouchDB are defined according to the JSON specification, and is fairly straightforward for entities that can be easily thought of as 'objects'. JSON allows for nesting child entities as sub-objects, forming hierarchical (tree) structures of infinite depth. As an alternative to tabular relations where relationships are defined by key references, hierarchical structures allow for easily structuring specific entities, but are less suitable for working with classes of entities. As such, data is typically grouped differently in JSON documents compared to relational tables; objects encourage groupings of specific entities, while tabular relations encourage grouping entity types.

While it is straightforward to model composition (child entities that belong exclusively to parent entities) as objects, it is more difficult to model aggregations (child entities that are used by many different parent entities) and associations (entities that interact with other entities). Nested hierarchical structures are also less suitable for comparisons across child-entities. For example, a relational database may have a table of child-entities 'products' where it would be easy to compare prices of the same product kept at many shops (parent entities); but a shop-product relationship modeled as hierarchical objects would first involve traversing every object of type `shop' to retrieve a list of products before a comparison could be made. Compared to the same analysis in a relational database, this is less performant. There is also a great deal of replicated data since a child entity would be repeated for every parent entity. As such, many NoSQL databases work with denormalized data models. The advantage of this is that entities of a single type may be structured differently according to requirement per object.

Because JSON is far less structured than relational tables, entities of the same type may be represented differently - i.e. to think of entities as 'rows' of a relational table, each entity is allowed it's own variety of 'columns' in a JSON store. In a relational database this would result in a sparse table where every row has to include the columns of every other row of the same entity type. Such a case would result in massive amounts of wasted storage and inefficient indexing. Since such categorizations in CouchDB are entirely logical, there is less wasted space.

But since CouchDB does not provide a means of differentiating between documents except by the content of each document, database content is not immediately classifiable by relations, fields, constraints and other objects that can be used to ascertain content type as can be done in RDBMSs; even MongoDB, an alternative JSON store to CouchDB uses the idea of 'collections' that allow for categorizing different types of documents. CouchDB documents are only classifiable by retrieving and reading the entire document. Logical entity classification in CouchDB is usually enforced (logically) by including a field ``type'', that allows for logical handling of different entity types after document retrieval. Alternatively, entity classification can be included in the id field of CouchDB documents, which has the benefit of allowing indexed entity retrieval.