\section{MapReduce}
In response to dealing with huge amounts of data on a daily basis, authors at Google (Jeffrey Dean and Sanjay Ghemawat) outlined a programming model that abstracted complications associated with distributed computing such as how to parallelize processing, data distribution, fault tolerance, load balancing and execution time \cite{Dean:2008}. This model, known as \textit{MapReduce}, provides programmers a conceptually-simple interface for specifying dispersed data computations succinctly and hides implementation details. The framework relies on an astoundingly simple programming model described by \cite{Dean:2008} as a computation that takes a set of input \textit{key:value} pairs and produces a set of output \textit{key:value} pairs via the following 3 steps:

\begin{enumerate}
    \item A \textit{mapping} stage in which distributed \textit{key:value} pairs are produced from input data as described by a user-defined \textit{map} function
    \item A \textit{grouping} stage where distributed \textit{key:value} output from the mapping stage is collected to common \textit{keys} - i.e. \textit{key:[value, value, value]} datasets
    \item And a \textit{reduce} stage where \textit{values} per key \textit{key} are processed as described by a user-defined \textit{reduce} function
\end{enumerate}

Due to the distributed and isolated nature of \textit{map} and \textit{reduce} tasks, \textit{MapReduce} as an idea is greatly fault tolerant (fault tolerance is implemented via reexecution), which has in turn resulted in the "New Software Stack" as mentioned by \cite{mining2011}: large scale computing clusters built on commodity (cheap) hardware and software that computes in parallel. The "New Software Stack" represents processing ever-greater amounts of data at ever cheaper rates and has spurred information explosion across all manor of software applications.

With the development of the \textit{Hadoop} framework as an open-source alternative to Google's proprietary file system and MapReduce framework, data computations within a MapReduce context have become mainstream. As \cite{chandar2010} discusses in his MSc thesis "Join Algorithms using Map/Reduce" made available by the University of Edinburgh, many companies now utilize this idea including Yahoo, Facebook, Amazon and many others (The Apache Foundation maintains a list of companies that use the Hadoop framework \cite{hadoopPower:2017}).

With increasing update within a data-analysis context, it is fair to say that many of the algorithms required on a day-to-day basis in common data-querying tasks can be implemented via the MapReduce framework including \textit{relational-algebra} operations such as \textit{selection}, \textit{projection} (selection of a subset of attributes from a tuple), \textit{union}, \textit{intersection}, \textit{difference}, \textit{joins} (non-equi joins cannot be implemented via MapReduce), \textit{grouping} and \textit{aggregation} \cite{mining2011}.

As the subject of his thesis, \cite{chandar2010} outlines and measures performance for \textit{Multi-Way} joins using \textit{Map-Side Join}, \textit{Reduce-Side One-Shot Join},\textit{Reduce-Side Cascade Join} algorithms. One take from this research is that both \textit{Two-Way} and \textit{Multi-Way} joins can be implemented via the MapReduce framework in general, though this is dependent on specific implementations and context of MapReduce engines.

MapReduce as implemented in the CouchDB software is used as a means of creating 'views' of databases, structured as indexes. Discussing CouchDB's MapReduce implementation in terms of relational operations is relevant since views often require some kind of relational grouping.