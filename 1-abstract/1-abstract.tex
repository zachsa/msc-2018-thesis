In the context of \textit{Educational Data Mining} it is shown that the University of Cape Town's chosen benchmarks for incoming student evaluations are all fairly reliable indicators of future academic achievement for undergraduate careers, and that Sakai general Learning Management Software (LMS) usage doesn't correspond strongly to course results. This analysis was conducted by joining datasets of student demographic and benchmarking information, university grades, and LMS (learning management software) usage. This dissertation outlines an example of how data aggregation can be achieved using CouchDB's \textit{MapReduce} implementation.

Querying and aggregating data in CouchDB is achieved by creating B+tree indexes derived from the main database files. These indexes are optimized for incremental updates, meaning that CouchDB is a suitable tool for near realtime analysis of huge databases dispersed across many, many servers. This dissertation shows a map function that achieves joining data with reference to the built-in \textit{\_stats} function. In support of this project, an open source, configurable ETL tool called \textit{nETL} was conceptualized to asynchronously extract, transform and load data from large flat files into CouchDB iteratively and in batches.