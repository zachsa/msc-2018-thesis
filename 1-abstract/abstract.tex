CouchDB is a NoSQL database that shines when working with large semi/un-structured datasets such as are typically used in data mining. Data mining comprises ETL processes in which multiple datasets are joined, and explored in terms of projection (attribute filtering) \& selection (row filtering) before processing via machine learning. Looking at three datasets - UCT grades, student admission benchmarks and Learning Management System (LMS) usage, this project discusses the application of CouchDB within the data-mining process.

Due to the elegance with which CouchDB separates handling of data from data modeling, CouchDB is an efficient tool for handling of large datasets with inconsistent schemas. In this project MapReduce is used to normalize data types during projection and to output derived indices suitable for performing joins and aggregations on related rows from separate datasets. Using a List function (as provided by the CouchDB API), data from indices is iteratively retrieved, joined, and exported in CSV format for assessing attribute variance and correlations between different attributes.

In the context of the analyzed data, it is shown that academic performance of students taking the CSC1015F course correlates better with National Benchmark Tests (NBT) than to Grade 12 (matric) performance.