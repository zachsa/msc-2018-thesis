\subsection{Data}
\label{appendix:data}
\subsubsection{Sakai Events}
\label{appendix:sakai-events}
\begin{table}[]
  \centering
  \caption{Sakai event data}
  \label{event-data-csv}
  \begin{tabular}{llll}
    Field name     & Field type & Notes                   & Include in scrape \\ \hline
    event\_date    & date       &                         & \cmark            \\
    event\_id      & int        & Filtered: only 281 used & \cmark            \\
    uct\_id (anon) & int        &                         & \cmark            \\
    site\_key      & int        &                         & \cmark            \\
    ref            & string     &                         & \xmark            \\ \hline
  \end{tabular}
\end{table}

An example of an event row as a JSON document
\begin{minted}{json}
{
  "_id": "000e569ee321b915bae59fe62e0051e3",
  "_rev": "1-7112afce121087818c33ebfd0fd7fed7",
  "event_date": "2016-04-17T14:04:20.000Z",
  "event_id": 281,
  "uct_id": 3018438,
  "site_key": 2297,
  "type_": "vulaEvent"
}
\end{minted}

\subsubsection{Grade data}
\label{appendix:grade-data}
\begin{table}[]
  \centering
  \caption{Grade data}
  \label{grade-data-csv}
  \begin{tabular}{lll}
    Field name       & Field type & Include in scrape \\
    DownloadedDate   & date       & \xmark            \\
    RegAcadYear      & int        & \cmark            \\
    RegTerm          & int        & \xmark            \\
    anonIDnew        & int        & \cmark            \\
    RegProgram       & string     & \xmark            \\
    RegCareer        & string     & \xmark            \\
    Degree           & string     & \xmark            \\
    DegreeDescr      & string     & \xmark            \\
    Subject          & string     & \xmark            \\
    Catalog.         & string     & \xmark            \\
    Course           & string     & \cmark            \\
    CourseSuffix     & string     & \xmark            \\
    Session          & string     & \xmark            \\
    Percent          & string     & \cmark            \\
    Symbol           & string     & \xmark            \\
    UnitsTaken       & int        & \xmark            \\
    CourseID         & int        & \xmark            \\
    CourseDescr      & string     & \xmark            \\
    CourseCareer     & string     & \xmark            \\
    Faculty          & string     & \xmark            \\
    Dept             & string     & \xmark            \\
    MaximumCrseUnits & int        & \xmark            \\
    CourseCount      & int        & \xmark            \\
    CourseLevel      & int        & \xmark            \\
    CESM             & int        & \xmark            \\
    Sub-CESM         & int        & \xmark            \\
  \end{tabular}
\end{table}

An example of a course grade as a JSON document:

\begin{minted}{json}
{
  "_id": "7530f4eed7e6bc3ef0d99a53be8ba9a2",
  "_rev": "8-232d0cf39728d41b4c5935f12469209d",
  "RegAcadYear": 2016,
  "RegTerm": 1161,
  "anonIDnew": 1,
  "RegCareer": "UGRD",
  "Degree": "QHB002",
  "Course": "PHI1010S",
  "CourseSuffix": "S",
  "Percent": "55",
  "CourseID": 109157,
  "Dept": "PHI",
  "type_": "courseGrade"
}
\end{minted}

% TODO: Add a description of each field
\begin{table}[]
  \centering
  \caption{List of white-listed columns for the Course Grade entity: only these fields are used in the analysis.}
  \label{Grades Columns}
  \begin{tabular}{ll}
    Column Name  & Description                 \\ \hline
    RegAcadYear  & Year of course registration \\
    RegTerm      & Registration year ID        \\
    anonIDnew    & Student identification      \\
    RegCareer    & Undergrad or other          \\
    Degree       & Degree code                 \\
    Course       & Catalog ID                  \\
    CourseSuffix & Session information         \\
    Percent      & Grade achieved              \\
    CourseID     & Numerical ID                \\
    Dept         & Department                  \\
    CourseCount  & Course credit information   \\ \hline
  \end{tabular}
\end{table}

% TODO: discuss the possible values for RegCareer and why only UGRD was used in the study
\begin{table}[]
  \centering
  \caption{RegCareer value list in terms of filtering}
  \label{RegCareerFilter}
  \begin{tabular}{ll}
    RegCareer & Included \\ \hline
    MAST      & FALSE    \\
    PDEV      & FALSE    \\
    BUSN      & FALSE    \\
    HONS      & FALSE    \\
    EMST      & FALSE    \\
    PGDP      & FALSE    \\
    UGRD      & TRUE     \\
    DOCT      & FALSE    \\
    NDGP      & FALSE    \\
    PDOC      & FALSE    \\
    MEDS      & FALSE    \\
    NDGU      & FALSE    \\ \hline
  \end{tabular}
\end{table}

% TODO: discuss which course suffixes were used and why
\begin{table}[]
  \centering
  \caption{CourseSuffixFilter}
  \label{CourseSuffixFilter}
  \begin{tabular}{lll}
    Suffix & Include & Treatment \\ \hline
    D      & FALSE   & /         \\
    F      & TRUE    & S1        \\
    H      & TRUE    & S1 \& S2  \\
    L      & TRUE    & S1        \\
    M      & FALSE   & /         \\
    N      & FALSE   & /         \\
    P      & TRUE    & S2        \\
    Q      & FALSE   & /         \\
    R      & FALSE   & /         \\
    S      & TRUE    & S2        \\
    W      & TRUE    & S1 \& S2  \\
    X      & FALSE   & /         \\
    Z      & FALSE   & /         \\ \hline
  \end{tabular}
\end{table}

% TODO Discuss this table
\begin{table}[]
  \centering
  \caption{Grade results need to be treated as numbers for the purpose of this analysis, this table shows all different value types and the appropriate treatment for each}
  \label{PercentTreatment}
  \begin{tabular}{lll}
    Symbol & Meaning                  & Handling Logic  \\ \hline
    49A    & Absent for supplementary & Grade used      \\
    49S    & Supplementary pending    & Grade used      \\
    50C    & ?                        & Grade used      \\
    78     & Grade                    & Grade used      \\
    AB     & Absent (fail)            & 40\% Grade used \\
    ATT    & ?                        & N/A             \\
    DE     & Deferred                 & N/A             \\
    DPR    & Duly performed refused   & 20\% Grade used \\
    F      & Fail                     & 40\% Grade used \\
    GIP    & Thesis only              & N/A             \\
    INC    & Incomplete (fail)        & 20\% Grade used \\
    LOA    & Leave of absence         & N/A             \\
    OS     & Outstanding              & N/A             \\
    OSS    & Outstanding              & N/A             \\
    PA     & Pass (thesis)            & N/A             \\
    SAT    & Thesis only              & N/A             \\
    UF     & Unclassified Fail        & 30\% Grade used \\
    UNS    & Thesis only              & N/A             \\
    UP     & Unclassified pass        & 50\% Grade used \\ \hline
  \end{tabular}
\end{table}

\subsubsection{Demographic data}
\label{appendix:demographic-data}
\begin{table}[]
  \centering
  \caption{This table shows a variety of test scores used to benchmark incoming students to UCT. Included are certain Matric subject results (Eng, Math, Sci) as well as national benchmark test (NBT) scores}
  \label{demographic-data-csv}
  \begin{tabular}{llll}
    Field name             & Field type & Notes                           & Include in scrape \\ \hline
    anonIDnew              & int        &                                 & \cmark            \\
    Career                 & string     & filtered: only UGRAD used       & \xmark            \\
    Citizenship Residency  & string     & filtered: only SA citizens used & \xmark            \\
    SA School              & string     &                                 & \xmark            \\
    Eng Grd12 Fin Rslt     & string     &                                 & \cmark            \\
    Math Grd12 Fin Rslt    & string     &                                 & \cmark            \\
    Mth Lit Grd12 Fin Rslt & string     &                                 & \cmark            \\
    Adv Mth Grd12 Fin Rslt & string     &                                 & \cmark            \\
    Phy Sci Grd12 Fin Rslt & string     &                                 & \cmark            \\
    NBT AL Score           & string     &                                 & \cmark            \\
    NBT QL Score           & string     &                                 & \cmark            \\
    NBT Math Score         & string     &                                 & \cmark            \\
    RegAcadYear            & int        &                                 & \cmark            \\ \hline
  \end{tabular}
\end{table}

\begin{minted}{json}
{
  "_id": "6587aa5b36bba2a70aeba96d06f05d2b",
  "_rev": "1-8c7d395e046e452442907e3388c74b41",
  "anonIDnew": 3103212,
  "Eng Grd12 Fin Rslt": 58,
  "Math Grd12 Fin Rslt": 73,
  "Mth Lit Grd12 Fin Rslt": "",
  "Adv Mth Grd12 Fin Rslt": "",
  "Phy Sci Grd12 Fin Rslt": 67,
  "NBT AL Score": 53,
  "NBT QL Score": 43,
  "NBT Math Score": 60,
  "RegAcadYear": 2016,
  "type_": "demographic"
}
\end{minted}


