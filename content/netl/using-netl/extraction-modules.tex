\subsubsection{Extraction Modules}
The framework above has been conceptualized primarily as a means of working with CSV data as exported from the Sukai platform. Specifically, this MSc has the dual requirements of extracting data from an 80GB CSV and snonymising column(s) in this 80GB CSV in a predictable way. To this end, included in the nETL source code used to complete this thesis is a FLATFILE extraction module.

Ethical clearance to use student data, although granted by the University of Cape Town still makes the condition that identities of the students be anonymized. Partly as a way of testing the nETL framework as mentioned above, I offered to provide a 'one-click' installation of the nETL application that provides an anonymizing pipeline for use by Professor Sonia Berman to prepare the student data for this project.

Extracting data from a flatfile is only tricky if it's size is such that the memory footprint of file contents is greater than the memory made available to the process. For an 80GB CSV that is certainly the case. The only feasible means of extracting data from such a file is to utilize the concept of an 'iterator'. Effectively this involves specifying the size of a reading block via two pointers (a 'begin' and 'end' pointer) that then traverse the data structure of the file from start to finish. These pointers maintain a constant width apart and at each incrementation of the iteration retrieve the data between those pointers.

Using an iterator, it is straightforward to process portions of a file at time, so long as only partial retrieval of a file's data can still be understood. Flatfile's work well for this with data structured as any number of 'line' entities that don't need to be read in the context of other marker's that may be found within the file.

Almost all high-level languages provide numerous abstractions for reading flatfile's on a line-by-line basis. For example, in a few different languages file iterators can be generated in the following ways:

\paragraph*{Python}
\begin{minted}{python}
with open('/path/to/file') as file:
   for line in file:
       # Process the line
\end{minted}

\paragraph*{Ruby}
\begin{minted}{ruby}
IO.foreach('/path/to/file') do |line|
  # Process the line
end
\end{minted}

\paragraph*{C\#}
\begin{minted}{csharp}
using (StreamReader streamReader = File.OpenText(/path/to/file))
{
    string line = String.Empty;
    while ((line = streamReader.ReadLine()) != null)
    {
        // Process the line
    }
}
\end{minted}

JavaScript provides a similar API for reading files line-by-line to the languages mentioned above, and a user could write such a simple iterator module provided that the module adhere to the nETL module interface for extractions and implements a \mintinline{javascript}{getNext()} function that effectively allows for pausing such extraction. It's likely that there are numerous ways of achieving this with the standard ECMAScript 5 standard. But partly as a means of exploring more state-of-the art JavaScript features, and partly of a means of demonstrating these new features, the file extraction module as used in this thesis implements EcmaScript 6 generators.

As described by \cite{mozillaGenerators}, JavaScript generators allow for quickly implementing arbitrary iterators, including iterators over generated iterators. Using open source code provided by \cite{bower16}, nETL makes use of a generator function to create a file iterator, and then a higher level generator to iterate over results of the line generator:

\begin{minted}{javascript}
var pointer = 0;
var buffersize = 64KB; // As close to a disc read size as possible
var filesize = FileLength;

// Generate the filereader
function* _readLines() {
    while (pointer < filesize) {
        let lineBuffer = [];
        // 1. Create a dataBuffer (byte array) of the data between pointer and filesize
        // 2. Find the start of a line
        // 3. Iterator through dataBuffer until a newline marker is found
            // Load each byte into lineBuffer
        // 4. yield lineBuffer.toString
    };
};
var lineReader = _readLines();

// Simplified implimentation of getNext() public API
function getNext() {
    return lineReader.next();
};

// Simplified batch generator
function getBatch* () {
    let data = [];
    for (0..batchSize) {
        data.push(lineExtraction.getNext());
    };
    yield data;
};

// Then batches of lines can be extracted via the following code
var batch = getBatch.next();
\end{minted}

compared to implementing an iterating-linereader in other languages, the JavaScript syntax in this case is more verbose. However this is largely because of the opinionated, asynchronous approach to IO operations. This approach makes the nETL framework easier to implement as a whole, however, since it allows many JavaScript tasks to run asynchronously. This would be harder to implement in languages such as Ruby/Python/C\#/etc and would probably involve some kind of thread management. This is not required with JavaScript.