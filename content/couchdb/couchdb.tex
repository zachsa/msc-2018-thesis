\section{CouchDB}

\subsection{Environment Setup}
Similarly to the ubiquity of Micrososft Office metaphors to the point that it is easy to forget that they are metaphors at all, it is easy to forget that the personal computer is in itself a metaphor for working with a processing chip. While it is possible to install CouchDB on a Windows personal computer, there is very little incentive to do so since you would effectively nullify the benefits of working with a dispersed, JSON-based, HTTP-fronted database.

CouchDB is a new animal on the database market, with development started by xxx in xxx primarily as a tool to xxx. The current release, CouchDB 2.1 has a variety of features that may finally make it feasible to replace the way technology has traditionally been used in workplace roles. A summary of CouchDB's key features is:

\begin{enumerate}
    \item Semi-structured data storage allows for easier use by people without prior database experience
    \item An HTTP interface effectively brings database communication out of the stone age and makes interacting with a database on a low level accessible to anyone who can code a line of JavaScript
    \item A MapReduce-based query engine allows for easily creating indexes that can be calculated via distributed computing
    \item And very easy/cheap cluster setup when compared to traditional DBMSs
\end{enumerate}

With these benefits in mind, this project explores the usage of CouchDB as a persistent (but temporary) dispersed data-processing engine for student data in the form of:

\begin{enumerate}
    \item A tool for easily configuring CouchDB clusters with enough nodes to handle the required data volume
    \item An approach for easily designing CouchDB MapReduce indexes, implementing them and running them
\end{enumerate}

\section{legalities}
If organizations were to start to use virtual servers offered by consumer-facing provides such as Google/Hetzner/AWS/Digital Ocean/etc. Data flows over networks that are no-longer internally maintained and may cross intercontinental borders. This may or may not be desirable and needs to be looked at. For a company to protect it's IP, data should not be observable by anyone except a company that owns it. Is there a way to deal with this with cloud servers?