\section{nETL}
Many software packages exist to facilitate data processing; spreadsheet programs such as Microsoft Excel (or their online competition Google Sheets), databases such as Microsoft Access, MySQL are designed for this time of work.

But these tools fail when looking at scale of storage that Sakai's event data requires either in terms of ability (Microsoft Excel can work with a theoretical maximum of xxx lines, Access databases can be maximum 2GB), complexity or cost. Relational databases don't scale well horizontally speaking xxx, and this problem is going become more pronounced with time.

This thesis looks at creating a cost-effective 'analysis-engine' capable of scaling to many times the amount of data that a single machine can hold in a way that is still effective to analyze (distributed computing) and not too mentally taxing to implement.

specifically this project analyzed the viability of a software package called CouchDB, a NoSQL DBMS as a replacement for conventional RDMSs when analyzing data that has conventionally been housed RDBMSs (usually with an Oracle, Microsoft or SAP price tag associated with it).

\begin{figure}[h]
    \centering
    \includegraphics[scale=0.4]{./resources/figures/netlUML.pdf}
    \caption[nETL]{nETL}
    \label{nETL}
\end{figure}

Coming from a relational database environment there are a plethora of tools avaialable that facilitate transfer of CSV dato to a DBMS. These tools are available at a variety of different levels of extraction depending on a users technical skillset, time constraints and requirements. xxx: list some of these tools.

In a SQL Server environment SSDT (formerly SSIS) is considered the de facto standard for extracting/transforming and loading data between different data sources. xxx: find a graph on the usage of SSDT/SSIS in companies.

In fact it's likely that the availability of of SSDT/SSIS has influenced the uptake of SQL Server in operations that require dealing with large amount of data. It's fair to say that a barrier to using open source software such as CouchDB is the LACK of such software. Bespoke scripts are currently the only viable way of interfacing with CouchDB in a way that is comparable to SQL Server and SSDT. But with high-level languages such as node.js maturing, and the proliferation of small, focused libraries in these languages that abstract much of the unpleasant and gnarly aspects of bespoke scripting (xxx examples), bespoke data-scripting is nowhere near as difficult as it would be within the Microsoft environment (C\# or VB).

In line with the requirement of transferring large amounts of CSV data from a CSV source to CouchDB, and taking into account the comparatively low entry barrier to bespoke data-transformation scripts, a component of this MSc is an exploration of a possible alternative to SSDT for an environment other than Microsoft's SQL Server. This MSc project actually has several requirements that fall within the ETL spectrum that such a framework could easily be adapted to handle in a generic way. The framework has been published as an npm library and is available at ..., with source code available at github somewhere xxx.

Figure \ref{nETL} shows a potential architecture for a configurable component-based ETL tool. The intention of the framework is that it works on the basis of a pipeline of tasks. These tasks can be divided into 3 steps; an extraction step, a transformation step and a loading step. xxx insert something about ETL variations and why I chose the ETL pipeline.

\begin{enumerate}
    \item Extract data from a source specified by a user into memory
    \item Apply any number of transformations to the data in memory
    \item Load the data from memory into a destination specified by a user
\end{enumerate}

The framework itself is quite lightweight and comprises just the NETL, TaskManager and general purpose 'Tools' classes. For the purposes of this thesis, the framework as described by Figure \ref{nETL} has been prototyped in node.js with the source code available at xxx (there is also an npm package available at xxx). JavaScript is a suitable language to prototype this application for a number of reasons:

\begin{enumerate}
    \item It has a very succinct API making it fast to write code in (i.e. it is a highly abstracted language similarly to Ruby or Python)
    \item But unlike Ruby or Python (and other high level languages), it is opinionated in that it handles IO asynchronously by default
    \item The JavaScript implementation of object-orientation allows for easy runtime manipulation of the object model
    \item Another reason for choosing JavaScript is that it is very much in line with the spirit of CouchDB and the web in general
\end{enumerate}

\subsection{Using nETL}
\subsubsection{nETL Module}
Packaged as a node.js module, the intention is that a user can simply import the \mintinline{javascript}{NETL} module and instantiate the \mintinline{javascript}{NETL} class as a singleton. Thereafter a user can add their own extraction / transformation / load modules via the singleton's API (the returned object). The format of these modules should conform to the interface as represented in the diagram. (Actually, JavaScript is not a suitable language for specifying interfaces since tooling that allows interface implementation in the language easy is lacking).

As specified in Figure \ref{nETL}, modules should return an object with the properties 'name' and 'exe'. 'name' should be the naming identifier of the function, and 'exe' should return the starting function of the module. Adding modules to the framework makes them available to tasks as specified by configuration objects. These objects are implemented as a JSON, and are used as instructions to start Tasks. Effectively this allows users to specify custom data processing tasks as a configuration + module. The configuration is passed to the program on startup, and the tasks are run. A possible netl startup script with in line configuration and modules may look like this:

\begin{minted}{javascript}
/* File <user entry point>.js */

// Import the nETL module
const NETL = require('./path/to/netl.js');
var netl = NETL();

// Specify the extraction/transformation/load configuration
const config = [{
    "ID": "TaskName",
    "Extraction": {
        "Name": "ExtractionType",
        // ...
        "afterTaskRunCBs": ["function(moduleConfig) {console.log('Run on task-end')}"]
    },
    "Transformations": [{
            "Name": "TransType",
            // ...
            "afterTaskRunCBs": ["function(moduleConfig) {console.log('Run on task-end')}"]
        },
        {
            // ...
        },
        // ...
    ],
    "Load": {
        "Name": "LoadType",
        // ...
        "afterTaskRunCBs": ["function(moduleConfig) {console.log('Run on task-end')}"]
    }
}];

// Load your own bespoke extraction module
netl.loadExtractionModule((function() {
    function exe(obj) {
        // ...
    };
    return {
        name: "ExtractionType",
        exe: exe
    };
})());

// Load your own bespoke Transformation module
netl.loadTransformationModule((function() {
    function exe(obj) {
        // ...
    };
    return {
        name: "TransType",
        exe: exe
    };
})());

// Load your own bespoke Load module
netl.loadLoadModule((function() {
    function exe(obj) {
        // ...
    };
    return {
        name: "LoadType",
        exe: exe
    };
})());

// Run your task till completion
netl.taskManager.newTask(config);
\end{minted}
\subsubsection{Extraction Modules}
The framework above has been conceptualized primarily as a means of working with CSV data as exported from the Sukai platform. Specifically, this MSc has the dual requirements of extracting data from an 80GB CSV and snonymising column(s) in this 80GB CSV in a predictable way. To this end, included in the nETL source code used to complete this thesis is a FLATFILE extraction module.

Ethical clearance to use student data, although granted by the University of Cape Town still makes the condition that identities of the students be anonymized. Partly as a way of testing the nETL framework as mentioned above, I offered to provide a 'one-click' installation of the nETL application that provides an anonymizing pipeline for use by Professor Sonia Berman to prepare the student data for this project.

Extracting data from a flatfile is only tricky if it's size is such that the memory footprint of file contents is greater than the memory made available to the process. For an 80GB CSV that is certainly the case. The only feasible means of extracting data from such a file is to utilize the concept of an 'iterator'. Effectively this involves specifying the size of a reading block via two pointers (a 'begin' and 'end' pointer) that then traverse the data structure of the file from start to finish. These pointers maintain a constant width apart and at each incrementation of the iteration retrieve the data between those pointers.

Using an iterator, it is straightforward to process portions of a file at time, so long as only partial retrieval of a file's data can still be understood. Flatfile's work well for this with data structured as any number of 'line' entities that don't need to be read in the context of other marker's that may be found within the file.

Almost all high-level languages provide numerous abstractions for reading flatfile's on a line-by-line basis. For example, in a few different languages file iterators can be generated in the following ways:

\paragraph*{Python}
\begin{minted}{python}
with open('/path/to/file') as file:
   for line in file:
       # Process the line
\end{minted}

\paragraph*{Ruby}
\begin{minted}{ruby}
IO.foreach('/path/to/file') do |line|
  # Process the line
end
\end{minted}

\paragraph*{C\#}
\begin{minted}{csharp}
using (StreamReader streamReader = File.OpenText(/path/to/file))
{
    string line = String.Empty;
    while ((line = streamReader.ReadLine()) != null)
    {
        // Process the line
    }
}
\end{minted}

JavaScript provides a similar API for reading files line-by-line to the languages mentioned above, and a user could write such a simple iterator module provided that the module adhere to the nETL module interface for extractions and implements a \mintinline{javascript}{getNext()} function that effectively allows for pausing such extraction. It's likely that there are numerous ways of achieving this with the standard ECMAScript 5 standard. But partly as a means of exploring more state-of-the art JavaScript features, and partly of a means of demonstrating these new features, the file extraction module as used in this thesis implements EcmaScript 6 generators.

As described by \cite{mozillaGenerators}, JavaScript generators allow for quickly implementing arbitrary iterators, including iterators over generated iterators. Using open source code provided by \cite{bower16}, nETL makes use of a generator function to create a file iterator, and then a higher level generator to iterate over results of the line generator:

\begin{minted}{javascript}
var pointer = 0;
var buffersize = 64KB; // As close to a disc read size as possible
var filesize = FileLength;

// Generate the filereader
function* _readLines() {
    while (pointer < filesize) {
        let lineBuffer = [];
        // 1. Create a dataBuffer (byte array) of the data between pointer and filesize
        // 2. Find the start of a line
        // 3. Iterator through dataBuffer until a newline marker is found
            // Load each byte into lineBuffer
        // 4. yield lineBuffer.toString
    };
};
var lineReader = _readLines();

// Simplified implimentation of getNext() public API
function getNext() {
    return lineReader.next();
};

// Simplified batch generator
function getBatch* () {
    let data = [];
    for (0..batchSize) {
        data.push(lineExtraction.getNext());
    };
    yield data;
};

// Then batches of lines can be extracted via the following code
var batch = getBatch.next();
\end{minted}

compared to implementing an iterating-linereader in other languages, the JavaScript syntax in this case is more verbose. However this is largely because of the opinionated, asynchronous approach to IO operations. This approach makes the nETL framework easier to implement as a whole, however, since it allows many JavaScript tasks to run asynchronously. This would be harder to implement in languages such as Ruby/Python/C\#/etc and would probably involve some kind of thread management. This is not required with JavaScript.
\input{netl/using-netl/transformation-modules}
\input{netl/using-netl/load-modules}

\subsection{Framework implementation}
\subsubsection{Implementation}
nETL is primarily a task-managing application, and as such, the TaskManager class is effectively the core of the application. It is intended that it can be instantiated as an instance (i.e. there can be many TaskManager instances hosted within a single running application). In JavaScript this is best implemented via a constructor:

\begin{minted}{javascript}
/* File taskmanager.js */
function TaskManager(extractions, transformations, loads) {
    this.tasks = {};

    // Hold references to extractions/transformation/loads of the main application process
    this.extractions = extractions;
    this.transformations = transformations;
    this.loads = loads;
};

TaskManager.prototype.newTask = function(task) {
    // Add task to this.tasks and execute the task
};

module.exports = TaskManager;
\end{minted}

The application itself is intended to be singleton instance of \mintinline{javascript}{class NETL{}}, which provides an IO interface (either a console terminal or otherwise) to TaskManager instances, and modular extraction/transformation and load operations. Singleton's are typically implemented via the modular pattern in JavaScript, which is typically how libraries are delivered to users by package managers and invoked by \mintinline{javascript}{var library = require('library-name')();}. One possible way of implementing the NETL class in JavaScript (i.e the method that is used within this project's code base) is shown here:

\begin{minted}{javascript}
/* File netl.js */
module.exports = function() {
    // Private properties / methods
    const _extractions = {};
    const _transformations = {};
    const _loads = {};        
    const _taskManager = new TaskManager(_extractions, _transformations, _loads);
    function _loadExtractionModule(extractionOperation){};
    function _loadTransformationModule(transformOperation){};
    function _loadLoadModule(loadOperation){};

    // Make process API available
    return {
        taskManager: _taskManager,
        loadExtractionModule: _loadExtractionModule,
        loadTransformationModule: _loadTransformationModule,
        loadLoadModule: _loadLoadModule
    };
};
\end{minted}
\input{netl/implementing-netl/netl-engine}