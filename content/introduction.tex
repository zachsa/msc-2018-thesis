\section{Introduction}
Despite major disruptions to the University of Cape Town's teaching curriculum in 2016 xxx, the academic year was completed successfully with only minimal disruptions to the calendar schedule. Despite xxx hours of teaching time lost, xxx days of closing the university, course curriculums remained constant and exams tested a similar amount of material as had been previously tested.

One of the reasons the university could absorb such disruptions is due to their implementation of learning management software - their 'Vula' system, which is based on the 'Sukai' platform.

The trend of educational institutions has been to increasingly implement and rely on such software and UCT is no exception. Looking at a graph showing the uptake of such systems throughout globally acclaimed educational institutes shows just how important they have become xxx. A plethora of competing educational tools exist; Blackboard, Canvas, Sukai to name a few, all with a fairly comparable feature list xxx.

A key feature is ability to host and deliver course content via an online platform, taking advantage of functionality that such a platform can provide - in this case with reference to tracking of platform and content usage. With reference specifically to the Sakai platform, it is possible to follow which students accessed which resources and the timing of such events.

The striking at the University of Cape Town provides a unique opportunity to analyze this data as a comparison between classroom learning augmented by online tools vs primary reliance on online tools over an extended period by the same learners. Effectively a blind-control was inadvertently created that allows assessment of a single group of students in two isolated environments in terms of their resultant grades.