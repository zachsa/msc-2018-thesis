% \bibliography{../bibliography/msc_citations}

\section{Introduction}
Despite major disruptions to the University of Cape Town's teaching curriculum in 2016 xxx, the academic year was completed successfully with only minimal disruptions to the calendar schedule. Despite xxx hours of teaching time lost, xxx days of closing the university, course curriculum remained constant and exams tested a similar amount of material as had been previously tested xxx.

Grades (todo: compare grades between different years) also appear to have been largely unaffected/affected xxx. And the point of this thesis is to attempt to put these differences/similarities between different year's grades within context of 'learning management software'. The University of Cape Town (UCT) implements the 'Sukai' platform that makes such an analysis possible. As part of this project, data has kindly been provided by Professor Sonia Berman and xxx from three different data sources to make this possible, with student IDs anonymized consistently across all three data sources.

\begin{enumerate}
    \item Demographic data of students entering the university
    \item Course result data for every student
    \item Vula usage data in terms of individual student's 'presence' on the Sakai platform
\end{enumerate}

\begin{figure}[h]
    \centering
    \scalebox{.87}{
        \begin{tikzpicture}[node distance=1.5cm, every edge/.style={link}]
            \node[entity] (d) {DemographicData};
            \node[entity] (g) [below left=2cm of d] {GradeData} edge (d);
            \node[entity] (e) [below right=2cm of d] {SakaiEvent} edge (d) edge (g);
            \node[entity] (eD) [below=1.5cm of e] {EventDesc} edge (e);
            \node[entity] (eT) [below=1.5cm of eD] {EventType} edge (eD);
        \end{tikzpicture}
    }
    \caption[DataModel]{Model of data exports provided for this project}
    \label{DataModel}
\end{figure}

Within the tried and tested relational database model, such a schema is straightforward to query and is well understood. But with greater amounts of data becoming normal on a day to day basis it is worth testing such a relationship via a distributed processing model. As mentioned by \cite{mining2011} the MapReduce framework allows for such processing, and work from a 2010 MSc thesis from the University of Edinburgh \cite{chandar2010} Shows that MapReduce as a tool is viable for implementing JOINS in the traditional SQLesque sense. Specifically, \cite{chandar2010} mentions that MapReduce joins can be subdivided into two categories (\textit{Two-Way} and \textit{Multi-Way} joins), and he provides examples of algorithms for implementing each type.

With regards to the data provided for analysis in this project described in \ref{DataModel}, considering the data-pipeline there is room for decisions on modeling data in CouchDB. However, in the interest of exploration of the CouchDB software, entities have been left as close to the data export as possible since it is my intention to use MapReduce not only as a means of querying, but also as a tool for entity transformation as part of the ETL process. For this reason, a \textit{Multi-Way} is required with reference to \ref{DataModel}. Specifically, the join of \textit{DemographicData}, \textit{GradeData} and \textit{SakaiEvent} data requires a subjoin of \textit{SakaiEvent}, \textit{EventDesc}, \textit{EventType} to allow for querying events based on type. For this reason, possible algorithms for implementing \textit{Multi-Way} joins as mentioned by \cite{chandar2010} are examined.

\subsection{Multi-Way JOINS in MapReduce}

\section{Other introduction?}

The trend of educational institutions has been to increasingly implement and rely on such software and UCT is no exception. Looking at a graph showing the uptake of such systems throughout globally acclaimed educational institutes shows just how important they have become xxx. A plethora of competing educational tools exist; Blackboard, Canvas, Sukai to name a few, all with a fairly comparable feature list xxx.

A key feature is ability to host and deliver course content via an online platform, taking advantage of functionality that such a platform can provide - in this case with reference to tracking of platform and content usage. With reference specifically to the Sakai platform, it is possible to follow which students accessed which resources and the timing of such events.

The striking at the University of Cape Town provides a unique opportunity to analyze this data as a comparison between classroom learning augmented by online tools vs primary reliance on online tools over an extended period by the same learners. Effectively a blind-control was inadvertently created that allows assessment of a single group of students in two isolated environments in terms of their resultant grades.

\section{Tech approach}
As adopted all over the working world, Microsoft Excel (and other spreadsheet tools) have made using computing power available to all the general population such that global economies have been able to proliferate as quickly as they have. xxx

But we are fast reaching the limits of what Excel-based computing is able to achieve for two reasons:

\begin{enumerate}
    \item Local memory will not accommodate the increasing amount of data available
    \item Moor's law will eventually create a hard limit on the amount of data that a personal machine can process.
\end{enumerate}

As such, a new user technology stack needs to be thought out that allows user's easier access to more computing power. It is the opinion of this author that such a tech-stack will no-longer be the over-opinionated GUIs as current spreadsheets represent, but a regression back to text-based terminals that allow for more configurable and bespoke computer usage. With increasingly easy-to-learn programming APIs and increasingly complicated GUIs, and according to xxx, it is no longer more difficult to learn a high-level programming language like JavaScript/Python/Ruby/etc than it is to learn a GUI. Configuration is also easier than it has been with the emergence of YAML, JSON, etc. over XML. And programming directly allows for far more configuration than a GUI can ever provide.

This is not to say that the prevalence of GUIs will diminish over time, but rather that their usage will shift to facilitate easier implementation of high-level programming languages.

It will definitely require a shift in mentality, however, to move away from Microsoft-provided metaphors that have become so ubiquitous that most users don't even use that they are using metaphors at all. Users think of MS Word 'documents' and their 'pages' as if they were just as real as the tangible items that they are based on.

Part of this project will involve a breakdown of common visual metaphors used in every-day data analysis / processing and discuss how those metaphors could evolve to accommodate similar processes as currently in use but with the ability to process exponentially greater amounts of data.

\section{Metaphors}
Looking behind job-requirement listings that specify the need to work with 'MS Office' as as skill for considering employment, it is important to look at the 'why'. Computers as a workplace tool have become synonymous throughout the modern economy due to their usefulness in scaling output potential of individual employees; this is not to say that economic output in itself has scaled proportionally with the uptake of computers (research by xxx, xxx, etc has indicated that economic evolution has NOT been indicative of such economic gains contrary to expectations), but it is undeniable that productivity measurable on an individual's level has benefited substantially by the ability to use computers.

When a job advertises the need for comfort on software such as MS Office, they are in fact stating that they have a requirement for the productivity gains that these software suites provide. At it's core, this effectively means being able to store text, spreadsheet-data, along with associated comments, track changes, formatting requirements, etc. etc. in a specified digital format. Such a format allows for productivity gains when individuals work together. For example, writing text with pen and ink may not be much of a time-saver compared to typing up a document in MS Word, but if you need it to be proofread by someone in a different country, email is substantially faster than the postal service. (Especially the south African Postal Service!)

When a job states 'MS Office' as a skills requirement, the reason is that they need the efficiency gains that come from using a computer. In other words, they are looking for candidates that can exert some measure of control on hardware related to 'computing' and in themselves form part of a computing system. Employees need to compute on regular basis, and MS Office is a current measurement of that computing ability.

However, MS Office was envisioned and developed in a world vastly different to the one we live in today; computing power was expensive and limited and digital information exchange was constricted in terms of bandwidth and (also) data expense. As both of these technologies experience exponential cost decreases it can be expected that businesses computing requirements will expand to fit into whatever technology becomes available and affordable.

In fact this is already happening. The graph below shows the increasing data consumption across many businesses over the past x years xxx. And especially noteworthy is that many of these businesses aren't actually centered around providing technology in itself. A medical-aid, for example, currently uses data-driven analysis to define plan-pricing options whereas that data wasn't available in the past. And the marketing industry is another good example, made possible by increasingly prolific 'networks' of people interacting with each other entirely online (Facebook, Twitter, etc. etc.).

Tools such as MS Office are fast reaching their capacity, and will afford control over 'computing' that is no longer sufficient for employees of companies. At this point, 'metaphors' that have become so ingrained in our digital world that most people don't even see them as metaphors any more (a 'document' in MS Word, that looks like a 'page', which displays 'text' that a user can read), will come under scrutiny. Then they will change, and they will evolve to fit a new world order - another component in the every expanding information age.

This MSc thesis explores a potentially different different metaphor-stack compared to the one that is currently in wide spread use, and how a user may use such metaphors in the future. This is tested via an analysis of student grades from the University of Cape Town in 2016 - a year drastically effected by student striking.

\section{Student Grades}

