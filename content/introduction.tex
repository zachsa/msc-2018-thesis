\section{Introduction}
Despite major disruptions to the University of Cape Town's teaching curriculum in 2016 xxx, the academic year was completed successfully with only minimal disruptions to the calendar schedule. Despite xxx hours of teaching time lost, xxx days of closing the university, course curriculum remained constant and exams tested a similar amount of material as had been previously tested.

One of the reasons the university could absorb such disruptions is due to their implementation of learning management software - their 'Vula' system, which is based on the 'Sukai' platform.

The trend of educational institutions has been to increasingly implement and rely on such software and UCT is no exception. Looking at a graph showing the uptake of such systems throughout globally acclaimed educational institutes shows just how important they have become xxx. A plethora of competing educational tools exist; Blackboard, Canvas, Sukai to name a few, all with a fairly comparable feature list xxx.

A key feature is ability to host and deliver course content via an online platform, taking advantage of functionality that such a platform can provide - in this case with reference to tracking of platform and content usage. With reference specifically to the Sakai platform, it is possible to follow which students accessed which resources and the timing of such events.

The striking at the University of Cape Town provides a unique opportunity to analyze this data as a comparison between classroom learning augmented by online tools vs primary reliance on online tools over an extended period by the same learners. Effectively a blind-control was inadvertently created that allows assessment of a single group of students in two isolated environments in terms of their resultant grades.

\section{Tech approach}
As adopted all over the working world, Microsoft Excel (and other spreadsheet tools) have made using computing power available to all the general population such that global economies have been able to proliferate as quickly as they have. xxx

But we are fast reaching the limits of what Excel-based computing is able to achieve for two reasons:

\begin{enumerate}
    \item Local memory will not accommodate the increasing amount of data available
    \item Moor's law will eventually create a hard limit on the amount of data that a personal machine can process.
\end{enumerate}

As such, a new user technology stack needs to be thought out that allows user's easier access to more computing power. It is the opinion of this author that such a tech-stack will no-longer be the over-opinionated GUIs as current spreadsheets represent, but a regression back to text-based terminals that allow for more configurable and bespoke computer usage. With increasingly easy-to-learn programming APIs and increasingly complicated GUIs, and according to xxx, it is no longer more difficult to learn a high-level programming language like JavaScript/Python/Ruby/etc than it is to learn a GUI. Configuration is also easier than it has been with the emergence of YAML, JSON, etc. over XML. And programming directly allows for far more configuration than a GUI can ever provide.

This is not to say that the prevalence of GUIs will diminish over time, but rather that their usage will shift to facilitate easier implementation of high-level programming languages.

It will definitely require a shift in mentality, however, to move away from Microsoft-provided metaphors that have become so ubiquitous that most users don't even use that they are using metaphors at all. Users think of MS Word 'documents' and their 'pages' as if they were just as real as the tangible items that they are based on.

Part of this project will involve a breakdown of common visual metaphors used in every-day data analysis / processing and discuss how those metaphors could evolve to accommodate similar processes as currently in use but with the ability to process exponentially greater amounts of data.