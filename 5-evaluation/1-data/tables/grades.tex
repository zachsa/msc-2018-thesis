\begin{table}[h]
    \begin{threeparttable}
        \textbf{Table \ref{tbl-sakai-grades}}\par\medskip\par\medskip
        \caption[Sakai grade data]{A description of the Sakai grade data as received in CSV format, and how these fields were treated in the ETL and analysis process}
        \label{tbl-sakai-grades}
        \begin{tabularx}{\textwidth}{>{\hsize=0.4\hsize}Y>{\hsize=1\hsize}X>{\hsize=0.7\hsize}X>{\hsize=1.9\hsize}X}
            \toprule
            \mC{c}{Include\tnote{\textsuperscript{1}}  } & \mC{c}{Field Name} & \mC{c}{Data type} & \mC{c}{Description}                                    \\
            \midrule
            \xmark                                       & DownloadedDate     & date              & Excel date format                                      \\
            \cmark                                       & RegAcadYear        & number            & Year                                                   \\
            \xmark                                       & RegTerm            & number            & Integer ID                                             \\
            \cmark                                       & anonIDnew          & number            & Anonymized student number                              \\
            \xmark                                       & RegProgram         & string            & Program abbreviation                                   \\
            \xmark                                       & RegCareer          & string            & Academic level\tnote{\textsuperscript{3}}              \\
            \xmark                                       & Degree             & string            & Degree code                                            \\
            \xmark                                       & DegreeDescr        & string            & Degree title                                           \\
            \xmark                                       & Subject            & string            & Three letter abbreviation                              \\
            \xmark                                       & Catalog.           & string            & Catalog sub-component (number and session)             \\
            \cmark                                       & Course             & string            & Full course code description                           \\
            \xmark                                       & CourseSuffix       & string            & Course session identifier  \tnote{\textsuperscript{4}} \\
            \xmark                                       & Session            & string            & Session name..                                         \\
            \cmark                                       & Percent            & string            & Grade achieved by student \tnote{\textsuperscript{5}}  \\
            \xmark                                       & Symbol             & string            & Symbol achieved by student                             \\
            \xmark                                       & UnitsTaken         & number            & Total units taken by student?                          \\
            \xmark                                       & CourseID           & number            & Numerical course Identifier                            \\
            \xmark                                       & CourseDescr        & string            & Description of course                                  \\
            \xmark                                       & CourseCareer       & string            & Academic level of course \tnote{\textsuperscript{6}}   \\
            \xmark                                       & Faculty            & string            & Course faculty                                         \\
            \xmark                                       & Dept               & string            & Faculty department                                     \\
            \xmark                                       & MaximumCrseUnits   & number            & ??                                                     \\
            \xmark                                       & CourseCount        & number            & ??                                                     \\
            \xmark                                       & CourseLevel        & number            & ??                                                     \\
            \xmark                                       & CESM               & number            & ??                                                     \\
            \xmark                                       & Sub-CESM           & number            & ??                                                     \\
            \bottomrule
        \end{tabularx}
        \scriptsize
        \begin{tablenotes}
            \item[\textsuperscript{1}]Attributes were included via a white-listing process.
            \item[\textsuperscript{3}]Only courses taken by students registered as undergraduates (UGRD) were considered in this study.
            \item[\textsuperscript{4}] A wide variety of course suffixes are present in the database, including: D,F,H,L,M,N,P,Q,R,S,W,X,Z.
            \item[\textsuperscript{5}]The percent field included both numbers and abbreviations that need to be normalized so that this field can be treated as always holding a numerical value. A table showing how the different suffixes and abbreviations of the percent field is shown in \ref{tbl-sakai-grades-percent}
            \item[\textsuperscript{6}]Only courses at undergraduate level (UGRD) were considered for this project.
        \end{tablenotes}
    \end{threeparttable}
\end{table}












\begin{table}[h]
    \begin{threeparttable}
        \textbf{Table \ref{tbl-sakai-grades-percent}}\par\medskip\par\medskip
        \caption{Grade results need to be treated as numbers for the purpose of this analysis, this table shows all different value types and the appropriate treatment for each. Because of the volume of data, it was not checked how many of these symbols apply to undergraduate students specifically, so these cases were handled generically}
        \label{tbl-sakai-grades-percent}
        \begin{tabularx}{\textwidth}{>{\hsize=0.6\hsize}X>{\hsize=1.3\hsize}X>{\hsize=1.1\hsize}X}
            \toprule
            \mC{c}{Symbol} & \mC{c}{Meaning}          & \mC{c}{Handling Logic}                     \\
            \midrule
            49A            & Absent for supplementary & Grade used                                 \\
            49S            & Supplementary pending    & Grade used                                 \\
            50C            & ?                        & Grade used                                 \\
            78             & Grade                    & Grade used                                 \\
            AB             & Absent (fail)            & 30\% Grade used\tnote{\textsuperscript{1}} \\
            ATT            & ?                        & N/A                                        \\
            DE             & Deferred                 & N/A                                        \\
            DPR            & Duly performed refused   & 20\% Grade used\tnote{\textsuperscript{2}} \\
            F              & Fail                     & 40\% Grade used\tnote{\textsuperscript{3}} \\
            GIP            & Thesis only              & N/A                                        \\
            INC            & Incomplete (fail)        & 20\% Grade used\tnote{\textsuperscript{2}} \\
            LOA            & Leave of absence         & N/A                                        \\
            OS             & Outstanding              & N/A                                        \\
            OSS            & Outstanding              & N/A                                        \\
            PA             & Pass (thesis)            & N/A                                        \\
            SAT            & Thesis only              & N/A                                        \\
            UF             & Unclassified Fail        & 30\% Grade used\tnote{\textsuperscript{1}} \\
            UNS            & Thesis only              & N/A                                        \\
            UP             & Unclassified pass        & 50\% Grade used\tnote{\textsuperscript{4}} \\
            \bottomrule
        \end{tabularx}
        \scriptsize
        \begin{tablenotes}
            \item[\textsuperscript{1}]30\% was applied on the estimate that failing in this case was slightly 'worse' than a regular fail.
            \item[\textsuperscript{2}]20\% was applied on the estimate that these students wouldn't necessarily have completed the coursework.
            \item[\textsuperscript{3}]40\% was applied on the estimate that this symbol would apply to students who participated in the course.
            \item[\textsuperscript{4}]50\% was applied on the estimate that these students passed without distinction or by concession.
        \end{tablenotes}
    \end{threeparttable}
\end{table}