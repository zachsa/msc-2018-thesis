\subsection{Data Analysis}
As mentioned previously, CouchDB views are optimized when using built-in reduce functions, with custom reduce functions performing most poorly on Windows machines. Since this project was completed on the Windows OS, the analysis on how best to aggregate the different entities (grades, demographics and events) was confined to using just the \textit{\_stats} built in reduce function. Of the three available built-in reduce functions, \textit{\_count}, \textit{\_sum}, and \textit{\_stats}, the \textit{\_stats} function provides the same output of the \textit{\_count} and \textit{\_sum} functions plus additional output. Using custom reduce functions would greatly increase the number of possible methods of joining entities since output of map functions would not be constrained to match the contracts of the \textit{\_stats} functions. Such an approach shouldn't be discounted considering that on platforms other than Windows, reduce function calculation (whether custom or built-in) represents a small percentage of computer resources used in view calculations overall (see appendix \ref{slack-1-nov}). And in any case, a system that utilizes CouchDB is likely to be based on a cluster of Linux machines rather than a single Windows machine. However, looking at utilizing the built-in reduce functions for aggregating data across many entities is worthy of an investigation in and of itself and is the subject of this project.

To initiate the analysis, a Map function is required with the constraint that output should adhere to the input requirements of the \textit{\_stats} function; that values are lists of numbers, and that lists should be of the same length, and that each index of the list corresponds with defined output. Analysis of the data will occur in two phases:

\begin{enumerate}
    \item Outlining the process of joining on Grade and Demographic data
    \item Introducing the 3rd entity (Sakai usage) as an additional component to the existing join
\end{enumerate}

This limitation is not a problem in the domain of EDM where analyses are based around numerical indicators (grades), but this may not always be the case in other problem domains. For the purposes of this project the \textit{Map} function will be explored in terms of implementing JOINS through use of CouchDB's compound-keys feature. That is, that the key component in Map output may in itself be a tuple.

