\subsection{Data Analysis}



Since the work of this project is being completed on a Windows machine, and to limit the scope of this MSc, only the 'built-in' \textit{\_stats} reduce function is used. This function takes an array of numbers as the \textit{values} parameter of the reduce-contract and calculates numerical statistics per index of emitted values. For example, a map function that emits

\begin{minted}{javascript}
{"somKey": [1,1,0]}
{"somKey": [3,1,3]}
\end{minted}

shows a reduce output (when using the \_stats reduce function) of:

\begin{minted}{javascript}
{"somKey": [
    {"sum":4,"count":2,"min":1,"max":3,"sumsqr":10}
    {"sum":2,"count":2,"min":1,"max":1,"sumsqr":2}
    {"sum":3,"count":2,"min":0,"max":3,"sumsqr":9}
], /* ... */}
\end{minted}

using the \textit{\_stats} function (and other built-in functions) all values must be numerical by nature, so such an approach won't work if joined datasets need to include strings. This limitation is not a problem in the domain of EDM where analyses are based around numerical indicators (grades), but this may not always be the case in other problem domains. For the purposes of this project the \textit{Map} function will be explored in terms of implementing JOINS through use of CouchDB's compound-keys feature. That is, that the key component in Map output may in itself be a tuple. Groupings of such keys can be configured to take into account a varying number of the indexes in tuples (or all) - with each value at any given index treated as a string.

