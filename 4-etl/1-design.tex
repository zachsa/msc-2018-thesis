\section{Design}
Conceptualizing a single ETL process as an entity of type \textit{Task}, that is, the ``extraction, transformation and loading of data from a source to destination'', provides a focal point on which the nETL software can be architected. Task instances are created by a constructor that takes a configuration object (loaded from a JSON file) as an object. Each task-instance is referenced by a singleton object created on app instantiation - the \mintinline{text}{taskManager}.

Starting the long-running nETL process comprises instantiating the \mintinline{text}{taskManager} singleton. This object provides a CLI (command line interface) to facilitate user interactions. Via the CLI, users can interact with \mintinline{text}{taskManager} to register different ETL components, start/stop tasks, configure application options such as log output path, etc.

The relationship between the \mintinline{text}{taskManager} singleton, \mintinline{text}{task} instances, and components is shown in Figure \ref{fig-nETL}. ETL components comprise modules that adhere to the Module interface and that accept a TaskConfig object as a paramater on instantiation. As such, modules are paired with configuration objects as shown in the coloured blocks. An ETL task comprises the following steps:

\begin{itemize}
    \item Loading all component modules required for a specific task into the nETL runtime
    \item Loading a configuration object that directs how each module should perform into the nETL runtime
    \item The runtime engine reads the configuration object, and runs the components as directed
\end{itemize}

\begin{figure}[H]
    \centering
    \begin{mdframed}
        \centering
        \includegraphics[scale=0.39]{./resources/figures/netl.png}
    \end{mdframed}
    \caption[nETL Architecture]{\textbf{Figure \ref{fig-nETL}: nETL architecture}}
    \label{fig-nETL}
\end{figure}