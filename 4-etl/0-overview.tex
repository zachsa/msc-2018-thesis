Working with data exports requires an ETL (Extraction, Transformation, and Loading) process, in which information stored in the CSV files is loaded into computer memory, transformed into JSON-structured objects, and then inserted into CouchDB. Once persisted, data interactions in terms of producing analyses becomes a flexible and agile process with room for creative insight since the cumbersome work of extracting and transforming data into a suitable format for querying does not have to to repeated.

Due to the ever-increasing complexity of the code being used for ETL, the codebase was formalized and structured as a component-based ETL engine that performed ETL tasks based on independent and external configuration objects. These tasks comprise sequential piping of output from one component to input of another component. By providing input/output data contracts, components can be strung together in any order, allowing for versatile and configurable ETL pipelines. By implementing the components as user-configurable, external processes to the ETL engine, specific logic required to process specific information is greatly decoupled from the logic required to extract, transform and load information as a series of high level tasks. The resultant software to achieve this task is implemented in JavaScript (node.js), and is called nETL (node.js ETL).