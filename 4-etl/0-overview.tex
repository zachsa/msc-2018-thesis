Working with data exports requires an ETL (Extraction, Transformation, and Loading) process, in which information stored in the CSV files is loaded into computer memory, transformed into JSON-structured objects, and then inserted into CouchDB. Once persisted, the data is available for exploration by retrieval from the data store itself, or from customized representation of the underlying data store via CouchDB MapReduce views (indices). By treating the data cleaning process as distinct from the analysis phase, data exploration becomes a flexible and agile process with room for creative insight since the cumbersome work of extracting and transforming data into a suitable format for querying does not have to to repeated.

Within the context of RDBMSs, there are a large variety of tools available to assist in ETL processing of CSV source data into databases - for example \textit{Open Studio for Data Integration} \cite{talend}, Microsoft's \textit{SSIS} \cite{ssis} and many, many more options including a host of cloud-based tools such as \textit{zapier} \cite{zapier} that allow for data-exchange between a plethora of different platforms - for example, inserting data into CouchDB from a Google Sheets spreadsheet. However, these tools are not available for CouchDB for the most part and do bespoke solutions are required. Initially, scripts developed for this project comprised ETL logic for specific source files and transformations specific to file contents. But it quickly became unmaintainable and a configurable ETL tool was developed instead.

Similarly to configurable ETL tools mentioned above, \textit{node.js ETL} (nETL) is designed in terms of \textit{Tasks} that are configured in terms of a pipeline of components. These tasks comprise sequential piping of output from one component to input of another component. Components are built adhering to a specified contract in terms of input/output, and as such can be strung together in any order allowing for versatile and configurable ETL pipelines. nETL consists of a central ETL engine that loads components from external files as specified in a task's configuration. Such separation of logic decouples management of tasks from using components and greatly decouples the logic of ETL tasks from the more specific logic of how extractions, transformations and loading logic is applied to a data source. The ETL engine as well as components is implemented in JavaScript (\textit{ECMAScript 2017\textsuperscript{®}}) \cite{ecmascript2017} and designed to run in the context of the \textit{node.js} runtime environment \cite{nodejs}. \textit{node.js} emphasizes asynchronous IO, making it a good fit for handling ETL tasks in which IO (CouchDB is accessed exclusively via network requests) accounts for the greatest amount of computational overhead.