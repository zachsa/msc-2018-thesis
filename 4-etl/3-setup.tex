\section{Setup}
Running nETL requires an installation of node.js V8.9.0 +, which should include an installation of npm \cite{npm}. After cloning the nETL repository from Github to a local drive, dependencies should be restored using the npm CLI tool. Then the nETL app can be started from a terminal. Once the CLI is running, typing anything into the terminal and pressing enter outputs help where further direction can be obtained.

In conjunction with setting up nETL, a CouchDB server needs to be configured. This is easy to do on Windows machines - simply download the executable from apache.org and use the installer. Once installed the server should be run in single node configuration, binded to the 127.0.0.1 address. This allows access to the CouchDB UI via the browser at the address: \url{http://127.0.0.1:5984/\_utils}, where first an admin user should be created. Working with databases via the CouchDB interface (called Fauxton) is straightforward.

Database creation involves only the single step of specifying a name and (optionally) security roles. CouchDB database configuration should be specified as part of creation - though this is only available when databases creation is specified via the HTTP interface and not the Fauxton GUI (the GUI doesn't allow for case-by case configration, but does allow for global configuration changes - although this is not recommended \cite{fauxton}); examples of configurable settings are sharding (q) and replica (n) count. For a single node setup, q=8 and n=1, meaning that a database has 8 shards and only 1 replica of each shard. This is the configuration used in this project. There is no point in storing more than one copy of a single shard on a single server, which is why n=1. For CouchDBs operation in cluster mode the default setup is q=8 and n=3. For clusters with a large number of nodes it might make sense to increase the value of the q parameter.

\subsection{CouchDB Design Documents}
Design documents are simply JSON strings, in which JavaScript functions are defined as strings. Working directly in JSON is unpleasant, however, and as such a open source tool, \textit{couchapp}, written in Python is used for authoring and installing Design Documents to CouchDB databases \cite{pythonCouchapp}. This tool maps a directory structure to a JSON document; in other words, directory names become keys, and directory contents become values associated with these keys. Directories can be nested in the same way that JSON allows nesting objects; nested directories translate to nested keys in the final JSON document.

Using this tool, map and list functions can be authored as JavaScript files and don't have to be manually serialized to JSON. Working with map and list functions as JavaScript files allows for code completion, syntax highlighting and numerous other IDE-specific benefits. Isolated function files are also possible to unit test.