\subsection{Related work}
Falling under the umbrella category of \textit{Educational Data Mining} (EDM), much work has been done with the intent of modeling student performance as dependent on certain markers such as attendance, assignment and test grades, high school marks, demographic data, etc. Different means of model generation have been discussed by the EDM community such as predictive analysis via decision tree generation as recently done by Honors-level students at UCT \cite{Balestra2017,casper2017} and several other researchers \cite{Qasem20016,Dimitris,zebun2005,Mierle:2005} with varying results. Many other models have been applied within the field of EDM as discussed in a review of EDM up to 2009 by (\cite{bakerEdMiningSummary}).

Although these studies discuss analysis-frameworks at length, very little work has been done on the underlying data stores that form a basis of such analysis. As such, it appears that the work done so far looks to research feasible models in terms of accuracy and implementation rather than feasible means of implementing such mining techniques. Where \textit{Extraction, Transformation, and Loading} (ETL) processes have been discussed with regards to data preparation, SQL is used but without much discussion as to the implementation of the underlying relational data stores (\cite{Balestra2017,casper2017,Mierle:2005}). No attempts have been made to use newer and less conventional NoSQL stores such as CouchDB despite that benefit that such data stores provide in terms of the unique features that these newer data stores offer (\textit{schema-less} entites, easier scaling, lower costs of implementation and licensing, etc.).

\subsubsection*{NoSQL Overview}
Fundamental to the concept of data storage is the implementation of \textit{ACID} constraints - \textit{A}tomicity (transactions either fail or succeed), \textit{C}onsistency (a database's state should always be consistent regardless of transactions), \textit{I}solation (transactions are self-contained) and \textit{D}urability (data is persisted reliably). Such constraints are well understood in RDBMSs and are fundamental to RDBMS usage; that they are often not translatable to NoSQL solutions means that data-handling via a NoSQL data store needs to be approached from a fundamentally different perspective.

Such idioms as \textit{BASE} - \textit{B}asically \textit{A}vailable (data stores are optimized to be available at the expense of consistency), \textit{S}oft state (allowance that state is inconsistent even across a single transaction) \textit{E}ventual consistency (data stores update to become consistent) are provide alternative frameworks to the \textit{ACID} metaphors and are solutions to the trade-offs required as per the \textit{CAP} theorem \cite{GANESHCHANDRA201513}: i.e. that a database represents a trade-off between 3 properties: \textit{consistency}, \textit{availability} and \textit{partition tolerance} with only 2 of the 3 properties fully achievable in a single system. As mentioned by the authors in \cite{GANESHCHANDRA201513}, modern (NoSQL) databases are focused on \textit{available} systems, where data layers are always available to users regardless of consistency (because the alternative is that user interactions fail). Generally (as in the case with CouchDB), availability is achieved via eager replication with some level of consistency applied via the idea of a quorum. CouchDB, when configured to work in sharded clusters with multiple copies of each shards (on different nodes), allows for configuring successful read/writes only when document representation is agreed upon by a set number of nodes. Because of the potential for inconsistency, NoSQL seeks to provide a 'relaxed' viewing model - i.e. \textit{Soft state} views where data representation is not tied to the underlying entities (an entity can be updated whilst being viewed unaware of such changes). That is, sacrificing of \textit{availability} at the expense of \textit{consistency} as per the \textit{cap} theorem; data conflicts, where entities are updated separately and independently of each other are often acceptable in NoSQL databases - particularly in an offline-first approach to data-handling.

Despite moving away from the relational model as provided by RDBMSs, the concept of 'entities' is usually still highly relevant in many NoSQL databases; these database can, as such, be grouped into two categories:

\begin{enumerate}
    \item \textit{aggregate orientated} stores that model data similarly to the relational model but with isolated entity boundaries (\cite{fowlerAggregate}) and
    \item \textit{aggregate ignorant} stores where the concept of entities is fundamentally different (e.g. a graph database such as \textit{Neo4J} where the entities are edges and nodes)
\end{enumerate}

By far, most databases operate within a domain where data is for the most part entity-driven. The family of \textit{aggregate orientated} NoSQL data stores include \textit{key/value} stores such such as Amazon's \textit{Dynamo} database, column based stores such as \textit{Cassandra}, \textit{HBase} and document stores such as \textit{CouchDB}, \textit{Mongo}, etc. As \cite{GANESHCHANDRA201513} points out there are hundreds of NoSQL data stores and a comprehensive categorization of such products is not sensible. Although NoSQL databases are said to be \textit{schema-less}, \cite{ATZENI2016} points out that this is not the case: instead NoSQL allows for inconsistent schema representation across different entity instances due to the nature of aggregations. Such flexibility is at the heart of document stores such as CouchDB and Mongo where loose-schema modeling is one of the properties that makes such technologies suitable for large systems that generate data from inconsistent sources (i.e. a constant 'survey' entity with each instance having different questions). As aggregates, instance-specific \textit{ACID} constraints CAN be implemented by \textit{aggregate oriented} data stores. CouchDB, for example, guarantees atomicity at a document level and optionally when inserting many documents at a time (using the \textit{bulk\_docs} endpoint \cite{bulkDocs}). Despite this, CouchDB doesn't support the idea of multistep transactions in the same was as many RDBMSs do. "Multi-step transactional atomicity" is a key feature for many RDBMSs including \textit{MySQL}, \textit{SQL Server}, etc. and overcoming this limitation is something that is required in order to implement NoSQL databases in traditional RDBMS environments. This is possible, as shown by \cite{Rashmi2017} for CouchDB specifically and NoSQL in general \cite{LOTFY2016133} via implementing \textit{ACID}/transactional properties as bespoke middleware externally to these DBSs. But it would seem this is not an ideal way of replacing RDBMS products due to the complexity and the potential for error that bespoke software, in such an important aspect of business logic as data storage, introduces.

Prior to CouchDB 2.0, middleware was also used for implementing sharding in CouchDB \cite{CORBELLINI20171}. However due to the nature of the CouchDB software itself, indexing such systems would have resulted in additional complications since the indexing engine of CouchDB prior to 2.0 did not allow dispersed calculations. In terms of classifying CouchDB within the NoSQL solutions, it is undoubtedly a document store but with certain characteristics of a \textit{key:value} data store due to the fine grained control users have over index creation on the nature of how those indexes are queried. In such cases, CouchDB mimics \textit{key:vale} database characteristics such as \textit{Redis}, \textit{MemCached}, \textit{Dynamo}, etc. \cite{MAKRIS201694,CORBELLINI20171,GANESHCHANDRA201513}.

The concept of data warehousing is also starting to undergo a shift towards a NoSQL mentality \cite{BICEVSKA2017104}. These authors point out that while very well suited to fact-table representation of structured data, there is still the requirement for handling and storing less structured data as well. Some advantages of turning to unstructured data warehousing technologies include, similarly to the move from structured to less-structured databases in general, easier and more agile development, easier storage of more types of data and as a result, potential for more meaningful and comprehensive data analysis. As \cite{BICEVSKA2017104} mentions, in this regard, NoSQL databases are likely to provide the most value when used in conjunction with traditional RDBMSs.

\subsubsection*{Querying via MapReduce}
In response to dealing with huge amounts of data on a daily basis, authors at Google (Jeffrey Dean and Sanjay Ghemawat) outlined a programming model that abstracted complications associated with distributed computing such as how to parallelize processing, data distribution, fault tolerance, load balancing and execution time \cite{Dean:2008}. This model, known as \textit{MapReduce}, provides programmers a conceptually-simple interface for specifying dispersed data computations succinctly and hides implementation details. The framework relies on an astoundingly simple programming model described by \cite{Dean:2008} as a computation that takes a set of input \textit{key:value} pairs and produces a set of output \textit{key:value} pairs via the following 3 steps:

\begin{enumerate}
    \item A \textit{mapping} stage in which distributed \textit{key:value} pairs are produced from input data as described by a user-defined \textit{map} function
    \item A \textit{grouping} stage where distributed \textit{key:value} output from the mapping stage is collected to common \textit{keys} - i.e. \textit{key:[value, value, value]} datasets
    \item And a \textit{reduce} stage where \textit{values} per key \textit{key} are processed as described by a user-defined \textit{reduce} function
\end{enumerate}

Due to the distributed and isolated nature of \textit{map} and \textit{reduce} tasks, \textit{MapReduce} as an idea is greatly fault tolerant (fault tolerance is implemented via reexecution), which has in turn resulted in the "New Software Stack" as mentioned by \cite{mining2011} - large scale computing clusters built on commodity (cheap) hardware and software that computes in parallel. The "New Software Stack" represents processing ever-greater amounts of data at ever cheaper rates and has spurred information explosion across all manor of software applications.

With the development of the \textit{Hadoop} framework as an open-source alternative to Google's proprietary file system and MapReduce framework, data computations within a MapReduce context have become mainstream. As \cite{chandar2010} discusses in his MSc thesis "Join Algorithms using Map/Reduce" made available by the University of Edinburgh, many companies now utilize this idea including Yahoo, Facebook, Amazon and many others (The Apache Foundation maintains a list of companies that use the Hadoop framework \cite{hadoopPower:2017}).

With increasing update within a data-analysis context, it is fair to say that many of the algorithms required on a day-to-day basis in common data-querying tasks can be implemented via the MapReduce framework including \textit{relational-algebra} operations such as \textit{selection}, \textit{projection} (selection of a subset of attributes from a tuple), \textit{union}, \textit{intersection}, \textit{difference}, \textit{joins} (non-equi joins cannot be implemented via MapReduce), \textit{grouping} and \textit{aggregation} \cite{mining2011}.

As mentioned by \cite{chandar2010} both \textit{Two-Way} and \textit{Multi-Way} joins can be implemented via the MapReduce framework in general, though this is dependent on specific implementations of MapReduce. \textit{Two-Way} joins can be achieved via MapReduce using \textit{Reduce-Side Join}, \textit{Map-side Join}, and \textit{Broadcast Join} algorithms. As the subject of his thesis, \cite{chandar2010} outlines and measures performance for \textit{Multi-Way} joins using \textit{Map-Side Join}, \textit{Reduce-Side One-Shot Join},\textit{Reduce-Side Cascade Join} algorithms. The author found that \textit{Multi-Way} joins were feasible using MapReduce, and that implementation complications of a dispersed system were hidden (as expected) by the Hadoop framework. In other words, \cite{chandar2010} found that relational operations are feasable within dispersed systems running the MapReduce framework.