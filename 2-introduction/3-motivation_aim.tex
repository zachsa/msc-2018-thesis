\subsection{Motivation \& Aim}
The data provided for this thesis deals with students usage of the \textit{Sakai} platform implemented at the University of Cape Town in conjunction with achieved grades. The trend of educational institutions has been to increasingly implement and rely on such learning management software, and UCT is no exception. It is unknown, however, to what extent these software solutions are used by students within their learning process in a traditional face-to-face environment is not known. A key feature of \textit{Sakai} is the ability to host and deliver course content via an online platform, taking advantage of functionality that such a platform can provide - in this case with reference to tracking of platform and content usage. With reference specifically to the Sakai platform, it is possible to follow which students accessed which resources and the timing of such events.

The striking at the University of Cape Town provides a unique opportunity to analyze this data as a comparison between classroom learning augmented by online tools vs primary reliance on online tools over an extended period by the same learners. Effectively a blind-control was inadvertently created by circumstance that allows assessment of a single group of students in two isolated environments (primarily face-to-face learning vs primarily distant learning) in terms of their resultant grades. Due to the disparate nature of the datasets involved (grade entities from different courses, events, and potentially other datasets), data from the Sakai platform is NOT well structured from an analysis point of view, with as many schema's as there are grade books. Although the data collected by such a platform is structured and well suited to a transactional system, any analysis of data done across the platform would have to take into account a multitude of schemas.

The Couchbase project \cite{couchbaseWhitePaper} mentions that in recent times, much of that data being produced on a day-to-day basis is semi-structured or unstructured due to the diversity of the kind of devices and users that are using data-collecting devices. RDBMSs seem ungainly in this scenario with their strictly defined data models making handling unstructured data cumbersome and expensive in terms of implementation time, and complex in terms of architecture and design. Additionally, systems like Oracle, DB2, SQL Server, MySQL and others scale more easily vertically than they do horizontally \cite{couchbaseWhitePaper}, which is comparatively limiting and expensive. Unlike RDMSs, NoSQL databases with their unstructured data models allow for easy expansion beyond single servers to many, many servers fairly easily and allow for a more agile approach to data modeling since data does not have to be statically modeled, and a model can changed very easily as a system evolves. Such is the motivation for rendering a data-analysis on student grade/event (semistructured) data with a NoSQL database (in this case CouchDB). In general, there has been substantial update of NoSQL data solutions such as CouchDB, Couchbase, Mongo, Cassandra and other solutions at the expense of established RDBMSs.

CouchDB is a scalable JSON storage that allows for database sharding (clustering features were added to CouchDB 2.0 released in 2016 with the merge of IBM's Cloudant code \cite{couchdb2.0}) across multiple commodity servers very easily. Theoretically, CouchDB as a data store is suitable for storing an unlimited amount of unstructured data at affordable infrastructure costs. It is also substantially cheaper to license than many RDBMSs since it is open source and available for free. This project looks to assess CouchDB in terms of being a viable alternative to working with data where SQL operations (JOINS in particular) are a common business requirement.

As a side note, CouchDB also supports an HTTP (RESTful) API that allows for interacting with the data-layer from any HTTP(S) clients such as browsers. Combined with CouchDB's handling of file data (attachments), and a couple other features (show functions, list functions, URL rewrites and a vHost engine to name a few) CouchDB can double as a web server that serves HTML - what the CouchDB community has traditionally referred to as "CouchApps". These tools allow for designing bespoke database-management tools very quickly and are definitely a CouchDB draw-card for many as seen in this open email exchange \cite{googleCon2017} by Apache Foundation members. However, this same exchange shows (unfortunately) that the "CouchApp" feature of CouchDB is unlikely to survive future releases due to lack of interest in contributing to this feature.

