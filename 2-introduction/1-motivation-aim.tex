\subsection{Motivation \& Aim}
The Couchbase project \cite{couchbaseWhitePaper} mentions that in recent times, much of that data being produced on a day-to-day basis is semi-structured or unstructured due to the diversity of the kind of devices and users that are using data-collecting devices. RDBMSs seem ungainly in this scenario with their strictly defined data models making handling unstructured data cumbersome and expensive in terms of implementation time, and complex in terms of architecture and design. Additionally, systems like Oracle, DB2, SQL Server, MySQL and others scale more easily vertically than they do horizontally \cite{couchbaseWhitePaper}, which is comparatively limiting and expensive. Unlike RDMSs, NoSQL databases with their unstructured data models allow for easy expansion beyond single servers to many, many servers fairly easily and allow for a more agile approach to data modeling since data does not have to be statically modeled, and a model can changed very easily as a system evolves. Such is the motivation for rendering a data-analysis on student grade/event (semistructured) data with a NoSQL database (in this case CouchDB). In general, there has been substantial uptake of NoSQL data solutions such as CouchDB, Couchbase, Mongo, Cassandra and other solutions at the expense of established RDBMSs.

CouchDB is a scalable JSON storage that allows for database sharding (clustering features were added to CouchDB 2.0 released in 2016 with the merge of IBM's Cloudant code \cite{couchdb2.0}) across multiple commodity servers very easily. Theoretically, CouchDB as a data store is suitable for storing an unlimited amount of unstructured data at affordable infrastructure costs. It is also substantially cheaper to license than many RDBMSs since it is open source and available for free. This project looks to assess CouchDB in terms of being a viable alternative to working with data where SQL operations (JOINS in particular) are a common business requirement.

CouchDB supports additional features that make it suitable for usage in a modern, ultra-connected world; an HTTP API. Unlike other databases with inconstant and undocumented protocols for communication between the database server and clients, CouchDB's HTTP interface allows for easier data accessibility. Both in the variety of ways data can be accessed (directly within a browser for example) as well as the ease of accessing data (via URLs either written by hand or via server/browser \textit{JavaScript}). Such features make CouchDB a suitable tool moving into the information-orientated society of the future where an 'agile' approach to data storage becomes the norm within an ever-more connected (in terms databases) world. The HTTP interface also allows for the idea of 'CouchApps', where HTML is served directly from documents stored in the database to browsers. This was originally a CouchDB draw-card for many as seen in this open email exchange \cite{googleCon2017} by Apache Foundation members. However, this same exchange shows (unfortunately) that the "CouchApp" feature of CouchDB is unlikely to survive future releases due to lack of interest in contributing to this feature.

In short, CouchDB is innovate as a database and allows for innovative system as a result. It would be useful for a better idea on the feasibility of it's usage for different use cases. This project aims to look at feasibility of querying data in CouchDB compared to similar queries in a SQL environment.