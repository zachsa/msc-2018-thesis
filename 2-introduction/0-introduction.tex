Insights into student grades, class/online participation, material engagement, demographic information and more allows for data-driven feedback on different approaches to learning and teaching. As such, exploring the plethora of data that learning management software such as UCT's Sakai platform collects has the potential to greatly improve the educational experience through data-mining. Intrinsic to this process is the concept of data-storage and retrieval - a topic that is becoming ever more important as the amount of data collected increases exponentially. As alternatives to traditional relational-orientated databases are become preferred software for housing large data stores in many cases, a mentality shift from data retrieval via the SQL (structured queried language) standard is required. Although many new NoSQL databases due implement a version of the SQL standard for querying, many do not. An alternative paradigm, and the subject of this project is \textit{MapReduce}, a framework for data querying that allows for infinitely dispersed data storage and processing, and by association, infinite possibilities in the world of data-mining.

\subsection{Project Significance}
An investigation of CouchDB's \textit{MapReduce} implementation in terms of how it handles querying of relational entities (and how) is the first step in assessing the databases suitability as a store for large amounts of data for the purposes of mining in the context of educational management systems. CouchDB is a new technology that embodies much of the NoSQL trend; a schema-less data model, MapReduce queries, open-source code-base and suitability for distribution over commodity hardware. This analysis will provide information for consideration when designing data-mining architectures of the future - be that in support of educational data-mining, general database usage, availability, etc. If CouchDB were a suitable technology for implementation within large-scale data architectures, that nature of a highly-available data store with mobile, browser and server implementations raises some interesting possibilities in terms of data access within the South African context - where data is expensive and there are still areas where university students have slow internet access.