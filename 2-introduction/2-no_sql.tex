\subsection{NoSQL Overview}
Fundamental to the concept of data storage is the implementation of \textit{ACID} constraints - \textit{A}tomicity (transactions either fail or succeed), \textit{C}oncistency (a database's state should always be consistent regardless of transactions), \textit{I}solation (transactions are self-contained) and \textit{D}urability (data is persisted reliably). Such constraints are well understood in RDBMSs and are fundamental to RDBMS usage; that they are often not translatable to NoSQL solutions means that data-handling via a NoSQL data store needs to be approached from a fundamentally different perspective.

Such idioms as \textit{BASE} - \textit{B}asically \textit{A}vailable (data stores are optimized to be available at the expense of consistency), \textit{S}oft state (allowance that state is inconsistent even across a single transaction) \textit{E}ventual consistency (data stores update to become consistent) are provide alternative frameworks to the \textit{ACID} metaphors and are solutions to the trade-offs required as per the \textit{CAP} theorem \cite{GANESHCHANDRA201513}: i.e. that a database represents a trade-off between 3 properties: \textit{consistency}, \textit{availability} and \textit{partition tolerance} with only 2 of the 3 properties fully achievable in a single system. As mentioned by the authors in \cite{GANESHCHANDRA201513}, modern (NoSQL) databases are focused on \textit{available} systems, where data layers are always available to users regardless of consistency (because the alternative is that user interactions fail). Generally (as in the case with CouchDB), availability is achieved via eager replication with some level of consistency applied via the idea of a quorum. CouchDB, when configured to work in sharded clusters with multiple copies of each shards (on different nodes), allows for configuring successful read/writes only when document representation is agreed upon by a set number of nodes. Because of the potential for inconsistency, NoSQL seeks to provide a 'relaxed' viewing model - i.e. \textit{Soft state} views where data representation is not tied to the underlying entities (an entity can be updated whilst being viewed unaware of such changes). That is, sacrificing of \textit{availability} at the expense of \textit{consistency} as per the \textit{cap} theorem; data conflicts, where entities are updated separately and independently of each other are often acceptable in NoSQL databases - particularly in an offline-first approach to data-handling.

Despite moving a way from the relational model as provided by RDBMSs, the concept of 'entities' is usually still highly relevant in many NoSQL databases; these database can, as such, be grouped into two categories:

\begin{enumerate}
    \item \textit{aggregate orientated} stores that model data similarly to the relational model but with isolated entity boundaries (\cite{fowlerAggregate}) and
    \item \textit{aggregate ignorant} stores where the concept of entities is fundamentally different (e.g. a graph database such as \textit{Neo4J} where the entities are edges and nodes)
\end{enumerate}

By far, most databases operate within a domain where data is for the most part entity-driven. The family of \textit{aggregate orientated} NoSQL data stores include \textit{key/value} stores such such as Amazon's \textit{Dynamo} database, column based stores such as \textit{Cassandra}, \textit{HBase} and document stores such as \textit{CouchDB}, \textit{Mongo}, etc. As \cite{GANESHCHANDRA201513} points out there are hundreds of NoSQL data stores and a comprehensive categorization of such products is not sensible. Although NoSQL databases are said to be \textit{schemaless}, \cite{ATZENI2016} points out that this is not the case: instead NoSQL allows for inconsistent schema representation across different entity instances due to the nature of aggregations. Such flexibility is at the heart of document stores such as CouchDB and Mongo where loose-schema modeling is one of the properties that makes such technologies suitable for large systems that generate data from inconsistent sources (i.e. a constant 'survey' entity with each instance having different questions). As aggregates, instance-specific \textit{ACID} constraints CAN be implemented by \textit{aggregate oriented} data stores. CouchDB, for example, guarantees atomicity at a document level and optionally when inserting many documents at a time (using the \textit{bulk\_docs} endpoint \cite{bulkDocs}). Despite this, CouchDB doesn't support the idea of multistep transactions in the same was as many RDBMSs do. "Multi-step transactional atomicity" is a key feature for many RDBMSs including \textit{MySQL}, \textit{SQL Server}, etc. and overcoming this limitation is something that is required in order to implement NoSQL databases in traditional RDBMS environments. This is possible, as shown by \cite{Rashmi2017} for CouchDB specifically and NoSQL in general \cite{LOTFY2016133} via implementing \textit{ACID}/transactional properties as bespoke middleware externally to these DBSs. But it would seem this is not an ideal way of replacing RDBMS products due to the complexity and the potential for error that bespoke software, in such an important aspect of business logic as data storage, introduces.

Prior to CouchDB 2.0, middleware was also used for implementing sharding in CouchDB \cite{CORBELLINI20171}. However due to the nature of the CouchDB software itself, indexing such systems would have resulted in additional complications since the indexing engine of CouchDB prior to 2.0 did not allow dispersed calculations. In terms of classifying CouchDB within the NoSQL solutions, it is undoubtedly a document store but with certain characteristics of a \textit{key:value} data store due to the fine grained control users have over index creation on the nature of how those indexes are queried. In such cases, CouchDB mimics \textit{key:vale} database characteristics such as \textit{Redis}, \textit{MemCached}, \textit{Dynamo}, etc. \cite{MAKRIS201694,CORBELLINI20171,GANESHCHANDRA201513}.

The concept of data warehousing is also starting to undergo a shift towards a NoSQL mentality \cite{BICEVSKA2017104}. These authors point out that while very well suited to fact-table representation of structured data, there is still the requirement for handling and storing less structured data as well. Some advantages of turning to unstructured data warehousing technologies include, similarly to the move from structured to less-structured databases in general, easier and more agile development, easier storage of more types of data and as a result, potential for more meaningful and comprehensive data analysis. As \cite{BICEVSKA2017104} mentions, in this regard, NoSQL databases are likely to provide the most value when used in conjunction with traditional RDBMSs.