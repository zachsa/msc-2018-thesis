Insight into the relationships between student grades, class/online participation, material engagement, demographic information and more allows for data-driven feedback on different approaches to learning and teaching. As such, data mining the plethora of information that learning management software such as UCT's Sakai platform collects has the potential to greatly improve the educational experience. Intrinsic to the process of working with these datasets are systems that support data-storage and retrieval. Logical frameworks that guide the design of such systems are becoming ever more important as the amount and rate of data collected increases exponentially.

Concurrently to increasing implementation and usage of data mining, alternatives to traditional relational-orientated databases are becoming preferred software for housing large data stores in many cases. But swapping out relational-storage for newer alternatives involves a mentality shift at many levels within the software stack; this is most noticeably evident in the data-retrieval layer with the shift in mindset from using SQL (structured query language) to query data stores, to other paradigms that are less familiar to most data-professionals (software engineers, software architects, database administrators, etc). Although many new NoSQL databases implement a version of the SQL standard for querying (which eases the learning curve for new technologies somewhat), many new databases do not. The SQL language (or more specifically, the data-query-language component of SQL) is based on relational classification (sets) and is not (easily or cleanly) implemented in fault-tolerant, dispersed systems such as CouchDB. Systems like Oracle, DB2, SQL Server, MySQL and others scale more easily vertically than they do horizontally \cite{couchbaseWhitePaper}, which is comparatively limiting and expensive. Unlike RDMSs, NoSQL (non-relational) databases allow for easy expansion beyond single servers to many, many servers fairly easily.

An alternative paradigm to SQL is MapReduce, a logical framework for data querying that allows for easily processing infinitely dispersed datasets. This technology, already used by several software giants \cite{chandar2010} is being adopted as part of the new technology stack along with the continuing trend of `information explosion`. As distributed computing power becomes more and more obtainable with the proliferation of cloud providers such as Digital Ocean, Hetzner, Amazon, Google, etc., it is worth investigating how technologies that make use of dispersed processing (such as via the MapReduce paradigm) can be incorporated into everyday business processes.

Also relevant to the shift from relational to non-relational databases is the increasing diversity of data being collected digitally. As mentioned by the Couchbase project \cite{couchbaseWhitePaper}, and from general experience in the work place in recent times, much of the data being produced on a day-to-day basis is semi-structured or unstructured (for example text-documents). And increasing technological gains such as represented by the proliferation of IOT (internet of things) devices creates the requirement of digital housing of ever more varied data. RDBMSs seem ungainly in this scenario with their strictly defined data models making handling semi-structured and unstructured data cumbersome. Appropriate systems are expensive in terms of time to implement, and complex in terms of architecting and usage. Storing data without having to first define rigid models allows for a more agile approach to data modeling, since subsequent changes to unstructured data models as a system evolves are straightforward and knock-on effects of code-changes isolated are much more isolated. Data models that don't rely on structured data storage can also easily be configured to evolve dynamically, though that is beyond the scope of this project.

\section{Project Significance}
CouchDB is a new technology that embodies much of the NoSQL trend; a schema-less data model, data projection, selection, and aggregation via MapReduce, an open-source code-base and suitability for distribution over commodity hardware. This analysis will provide information for consideration when designing data storage and mining architectures of the future - be that in support of educational data-mining or completely unrelated business domains.

As an example of CouchDB implementation, this project further enables development of CouchDB-based applications. With a focus on highly available data and an API implemented across multiple types of devices (servers, browsers, tablets, mobile phones, etc), CouchDB suitable as the foundation of 'offline-first' applications that can be used in relatively disconnected locations. For example, CouchDB formed part of the effort in containing the 2013 - 2016 Ebola virus outbreak \cite{ebola2017} by providing a means of digital data collection in areas with unreliable internet. Similarly there is a lot of scope in the South African context where data is still very expensive and internet access is sporadic throughout much of the country.

In terms of this case study, insight into the effectiveness of the University of Cape Town's student benchmarks for first time students is discussed as well as the correlation between usage of learning management software (Sakai) and course grades.

\section{Motivation \& Aim}
Theoretically, CouchDB as a data store is suitable for storing an unlimited amount of unstructured data across distributed clusters of commodity servers since it is a scalable JSON storage that allows for database sharding (clustering features were added to CouchDB 2.0 released in 2016 with the merge of IBM's Cloudant code \cite{couchdb2.0}). It is also substantially cheaper to license than many RDBMSs since it is open source and available for free. CouchDB also provides a novel way of interacting with a data store: an HTTP API that is effectively an interface that allows users to interact directly with b+tree structures from any client that supports HTTP. Such features make CouchDB a suitable tool moving into the information-orientated society of the future where an 'agile' approach to data storage becomes the norm within an ever-more connected world where more and more unstructured data needs to be processed.

In short, CouchDB is innovate as a database and allows for innovative systems as a result. Case studies involving CouchDB are necessary to develop an understanding of all the different use-cases that such novel software represents.

\section{Project Overview}
This report covers the following topics in the indicated order:

\begin{enumerate}
    \item An overview of CouchDB including the steps taken to set up the CouchDB software
    \item A discussion of CouchDBs mechanism for querying data (MapReduce as implemented by CouchDB)
    \item Development of nETL - bespoke software written to facilitate loading and adaption of CSV data to a CouchDB JSON store
    \item An iterative approach to querying the student data using nETL and CouchDB
    \item A discussion of the datasets created using CouchDB and correlations
\end{enumerate}

