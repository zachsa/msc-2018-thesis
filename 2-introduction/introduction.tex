Insights into student grades, class/online participation, material engagement, demographic information and more allows for data-driven feedback on different approaches to learning and teaching. As such, exploring the plethora of data that learning management software such as UCT's Sakai platform collects has the potential to greatly improve the educational experience through data-mining. Intrinsic to this process is the concept of data-storage and retrieval - a topic that is becoming ever more important as the amount of data collected increases exponentially. This project looks at the correlation between Sakai usage and achieved grades via contrasting two distinct methods of querying the data; that is, via CouchDB's MapReduce paradigm on the one hand vs using T-SQL (\textit{SQL Server}).

Concurrently with increased interest in data mining in general, alternatives to traditional relational-orientated databases are become preferred software for housing large data stores in many cases; this involves a mentality shift from data retrieval via the SQL (structured queried language) standard. Although many new NoSQL databases implement a version of the SQL standard for querying, many do not. An alternative paradigm is \textit{MapReduce}, a framework for data querying that allows for infinitely dispersed data storage and processing, and by association, infinite possibilities in the world of data-mining. This technology is being adopted as part of the new technology stack as information explosion continues, and is worth investigation with the aim of incorporating MapReduce into the data-mining process.

\section{Project Significance}
An investigation of CouchDB's \textit{MapReduce} implementation in terms of how it handles querying of relational entities (and how) is the first step in assessing the databases suitability as a store for large amounts of data for the purposes of mining in the context of educational management systems. CouchDB is a new technology that embodies much of the NoSQL trend; a schema-less data model, MapReduce queries, open-source code-base and suitability for distribution over commodity hardware. This analysis will provide information for consideration when designing data-mining architectures of the future - be that in support of educational data-mining, general database usage, availability, etc. If CouchDB were a suitable technology for implementation within large-scale data architectures, that nature of a highly-available data store with mobile, browser and server implementations raises some interesting possibilities in terms of data access within the South African context - where data is expensive and there are still areas where university students have slow internet access. In addition, this project will produce a joined dataset of course results associated with Sakai usage and a precursory analysis of such data.

\section{Project Overview}
This project discusses the work-flow of using CouchDB database as an analysis tool. Loosely the following topics are covered:

\begin{enumerate}
    \item Development of \textit{nETL} - bespoke software written to facilitate loading and adaption of CSV data to a CouchDB JSON store
    \item An overview of CouchDB including the steps taken to set up the CouchDB software
    \item A discussion of CouchDBs mechanism for querying data
    \item An iterative approach to querying the student data using CouchDB's MapReduce implementation
    \item A discussion of the datasets created using CouchDB and correlations
\end{enumerate}

\section{Motivation \& Aim}
The Couchbase project \cite{couchbaseWhitePaper} mentions that in recent times, much of that data being produced on a day-to-day basis is semi-structured or unstructured due to the diversity of the kind of devices and users that are using data-collecting devices. RDBMSs seem ungainly in this scenario with their strictly defined data models making handling unstructured data cumbersome and expensive in terms of implementation time, and complex in terms of architecture and design. Additionally, systems like Oracle, DB2, SQL Server, MySQL and others scale more easily vertically than they do horizontally \cite{couchbaseWhitePaper}, which is comparatively limiting and expensive. Unlike RDMSs, NoSQL databases with their unstructured data models allow for easy expansion beyond single servers to many, many servers fairly easily and allow for a more agile approach to data modeling since data does not have to be statically modeled, and a model can changed very easily as a system evolves. Such is the motivation for rendering a data-analysis on student grade/event (semistructured) data with a NoSQL database (in this case CouchDB). In general, there has been substantial uptake of NoSQL data solutions such as CouchDB, Couchbase, Mongo, Cassandra and other solutions at the expense of established RDBMSs.

CouchDB is a scalable JSON storage that allows for database sharding (clustering features were added to CouchDB 2.0 released in 2016 with the merge of IBM's Cloudant code \cite{couchdb2.0}) across multiple commodity servers very easily. Theoretically, CouchDB as a data store is suitable for storing an unlimited amount of unstructured data at affordable infrastructure costs. It is also substantially cheaper to license than many RDBMSs since it is open source and available for free. This project looks to assess CouchDB in terms of being a viable alternative to working with data where SQL operations (JOINS in particular) are a common business requirement.

CouchDB supports additional features that make it suitable for usage in a modern, ultra-connected world; an HTTP API. Unlike other databases with inconstant and undocumented protocols for communication between the database server and clients, CouchDB's HTTP interface allows for easier data accessibility. Both in the variety of ways data can be accessed (directly within a browser for example) as well as the ease of accessing data (via URLs either written by hand or via server/browser \textit{JavaScript}). Such features make CouchDB a suitable tool moving into the information-orientated society of the future where an 'agile' approach to data storage becomes the norm within an ever-more connected (in terms databases) world. The HTTP interface also allows for the idea of 'CouchApps', where HTML is served directly from documents stored in the database to browsers. This was originally a CouchDB draw-card for many as seen in this open email exchange \cite{googleCon2017} by Apache Foundation members. However, this same exchange shows (unfortunately) that the "CouchApp" feature of CouchDB is unlikely to survive future releases due to lack of interest in contributing to this feature.

In short, CouchDB is innovate as a database and allows for innovative system as a result. It would be useful for a better idea on the feasibility of it's usage for different use cases. This project aims to look at feasibility of querying data in CouchDB compared to similar queries in a SQL environment.