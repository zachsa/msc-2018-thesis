Insights into student grades, class/online participation, material engagement, demographic information and more allows for data-driven feedback on different approaches to learning and teaching. As such, exploring the plethora of data that learning management software such as UCT's Sakai platform collects has the potential to greatly improve the educational experience through data-mining. Intrinsic to this process is the concept of data-storage and retrieval - a topic that is becoming ever more important as the amount of data collected increases exponentially.

Concurrently with increased interest in data mining in general, alternatives to traditional relational-orientated databases are become preferred software for housing large data stores in many cases; this involves a mentality shift from data retrieval via the SQL (structured queried language) standard. Although many new NoSQL databases implement a version of the SQL standard for querying, many do not. An alternative paradigm is \textit{MapReduce}, a framework for data querying that allows for infinitely dispersed data storage and processing, and by association, infinite possibilities in the world of data-mining. This technology, already used by several software giants \cite{chandar2010} is being adopted as part of the new technology stack as information explosion continues. As distributed computing power becomes more and more obtainable with the proliferation of cloud providers such as Digital Ocean, Hetzner, Amazon, Google, etc. etc. it is worth investigating how technologies that make use of dispersed processing (such as the MapReduce paradigm) can be incorporated into everyday business processes.

\section{Project Significance}
CouchDB is a new technology that embodies much of the NoSQL trend; a schema-less data model, MapReduce queries, open-source code-base and suitability for distribution over commodity hardware. This analysis will provide information for consideration when designing data storage and mining architectures of the future - be that in support of educational data-mining or completely unrelated business domains.

CouchDB, with it's focus on highly available data stores and compatible implementations available for mobile devices, browsers, and servers, also raises some interesting possibilities for applications within relatively disconnected locations. For example, CouchDB is at the forefront of the 'mobile-first' movement (it formed part of the effort in containing the Ebola virus outbreak \cite{ebola2017}) due to the ease with which you can interact with it using the HTTP interface (i.e. using JavaScript in a browser). There is vast potential for application development in the South African context where data is still very expensive and internet access is sporadic even at some educational institutions.

In terms of this case study, insight into the effectiveness of the University of Cape Town's student benchmarks for first time students is discussed as well as the correlation between usage of learning management software (Sakai) and course grades.

\section{Project Overview}
This report covers the following topics in the indicated order:

\begin{enumerate}
    \item Development of \textit{nETL} - bespoke software written to facilitate loading and adaption of CSV data to a CouchDB JSON store
    \item An overview of CouchDB including the steps taken to set up the CouchDB software
    \item A discussion of CouchDBs mechanism for querying data
    \item An iterative approach to querying the student data using CouchDB's MapReduce implementation
    \item A discussion of the datasets created using CouchDB and correlations
\end{enumerate}

\section{Motivation \& Aim}
As mentioned by the  Couchbase project \cite{couchbaseWhitePaper}, and from general experience in the work place in recent times, much of that data being produced on a day-to-day basis is semi-structured or unstructured due to the diversity of the kind of devices and users that are using data-collecting devices. RDBMSs seem ungainly in this scenario with their strictly defined data models making handling unstructured data cumbersome and expensive in terms of implementation time, and complex in terms of architecture and design. Additionally, systems like Oracle, DB2, SQL Server, MySQL and others scale more easily vertically than they do horizontally \cite{couchbaseWhitePaper}, which is comparatively limiting and expensive. Unlike RDMSs, NoSQL databases with their unstructured data models allow for easy expansion beyond single servers to many, many servers fairly easily and allow for a more agile approach to data modeling since data does not have to be statically modeled, and a model can changed very easily as a system evolves. In general, there has been substantial uptake of NoSQL data solutions such as CouchDB, Couchbase, Mongo, Cassandra and other solutions at the expense of established RDBMSs. This project looks to asses CouchDB as a means of aggregating across semi-structured data via an EDM (educational data mining) case study.

CouchDB is a scalable JSON storage that allows for database sharding (clustering features were added to CouchDB 2.0 released in 2016 with the merge of IBM's Cloudant code \cite{couchdb2.0}) across multiple commodity servers very easily. Theoretically, CouchDB as a data store is suitable for storing an unlimited amount of unstructured data at affordable infrastructure costs. It is also substantially cheaper to license than many RDBMSs since it is open source and available for free. CouchDB is novel in that it's HTTP API effectively provides an easy means of interacting with b+tree structures directly from any client that supports the HTTP protocol. Such features make CouchDB a suitable tool moving into the information-orientated society of the future where an 'agile' approach to data storage becomes the norm within an ever-more connected world where more and more unstructured data needs to be processed.

In short, CouchDB is innovate as a database and allows for innovative system as a result. Case studies involving CouchDB are necessary to develop an understanding of all the different use-cases that such novel software represents.