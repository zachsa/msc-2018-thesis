\subsection{Results}
The \textit{nETL} application used to process the Sakai-event and student-grade data shows runtimes, CPU usage and memory usage as per \ref{netl-runtimes}. Network latency in this case can be ignored since CouchDB was bound to the \textit{127.0.0.1} loopback address on the same machine the \textit{nETL} was run. Although for the purposes of this project a 'pause' and 'resume' feature that would have allowed \textit{nETL} to be interrupted gracefully was implemented, it was found that such a feature was not required. The software ran to completion without fail on all it's numerous runs.

CouchDB's indexing engine is fault-tolerant and was able to handle machine restarts gracefully (which was required when calculated on locally due to the long-running time of some of the index calculations). \ref{couch-indexing} shows CouchDB's MapReduce indexing time dependent on cluster configuration. As seen in the table, increasing the cluster size reduces the time taken to index a database as per the described \textit{MapReduce} function decreases. This is as expected since dispersing documents across shards should be random enough that documents get distributed across shards uniformly (see \ref{appendix:slack5}), meaning that increasing nodes in a cluster reduces the amount of work a node must do when calculating \textit{map} output.

Although analysis of the student data is not the focus of this project, it's worth mentioning that on average there doesn't appear to be any correlation between grades achieved and Sakai learning management software usage as per \ref{sakai-grades-and-events}. This is likely because there problem domain is over-simplified and a more complicated model needs to be created to determine any meaningful results. For example, it is not enough to compare semester 1 grade average vs semester 2 grade average for 2016; there could be a systematic difference in course difficulty between first and second semester. Certain courses make greater use of the Sakai platform than other courses; this needs to be factored in. Even when normalizing the grades by a 'non-strike' year, it is necessary to take varying student quality into account using the student-background demographic data. In any case, for this test-result, data was retrieved from CouchDB via the \textit{List} function described previously as a CSV, and analyzed via Microsoft Excel.

\begin{table}[]
    \centering
    \caption{nETL task runtimes}
    \label{netl-runtimes}
    \begin{tabular}{cllll}
        \multicolumn{1}{l}{Run} & Task         & Runtime (sec) & Max memory usage (mb)* & Lines Processed (loaded)** \\ \hline
        \multirow{3}{*}{1}      & Events       & 4253          & \textless 200          & 44420508 (13184082)        \\
                                & Grades       & 73.9          & \textless 200          & 180893 (106301)            \\
                                & Demographics & 2             & \textless 200          & 12219 (9874)               \\
        \multirow{3}{*}{2}      & Events       & 7125          & \textless 200          & 44420508 (13184082)        \\
                                & Grades       & 107.5         & \textless 200          & 180893 (106301)            \\
                                & Demographics & 2             & \textless 200          & 12219 (9874)               \\
        \multirow{3}{*}{3}      & Events       & 4992          & \textless 200          & 44420508 (13184082)        \\
                                & Grades       & 85.6          & \textless 200          & 180893 (106301)            \\
                                & Demographics & 2.3           & \textless 200          & 12219 (9874)               \\ \cline{2-5}
    \end{tabular}
\end{table}

\begin{table}[]
    \centering
    \caption{CouchDB MapReduce index calculation time. For all cases, \textit{n} = 1 (shard copies), with variable \textit{q} (number of) shards and variables cluster size.}
    \label{couch-indexing}
    \begin{tabular}{cccc}
                       & Single node & 3 node cluster & 6 node cluster \\ \hline
        \textit{q} = 1 & 1           & 1              & 1              \\
        \textit{q} = 3 & 1           & 1              & 1              \\
        \textit{q} = 8 & 1           & 1              & 1              \\ \hline
    \end{tabular}
\end{table}

\begin{table}[]
    \centering
    \caption{Undergraduate Sakai usage vs grade results}
    \label{sakai-grades-and-events}
    \begin{tabular}{llll}
        Semester 1 Events           & Semester 2 Events           & Semester 1 Grade Avg       & Semester 2 Grade Avg       \\
        \multicolumn{1}{c}{5368126} & \multicolumn{1}{c}{4826540} & \multicolumn{1}{c}{67.9\%} & \multicolumn{1}{c}{74.2\%}
    \end{tabular}
\end{table}