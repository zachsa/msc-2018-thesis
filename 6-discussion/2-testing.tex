\section{Testing}
To check the accuracy of the statistacal measurements in the context of the source data, unit tests were used for testing of the nETL application components as well as the Map and List functions. In addition, a small sample of the data was processed equivalently to the analysis procedures as described above for manually testing the the accuracy of the output CSV (and ensure that the unit tests themselves are working as expected). Unit tests are written using the open source JavaScript \textit{Mocha} testing framework \cite{mochaTest}.

\subsection{nETL Unit Tests}
The basic premise of the ETL process is that the lines are extracted from CSVs and loaded into CouchDB reliably. Assertions are used to ensure that each nETL module (the extraction module, the transformation modules, and the loading module) perform as expected. No integration tests are performed except by manually checking that the test data is loaded into CouchDB in the anticipated format - which is indeed the case.

Tests relating to CSV-line extraction assert that all lines from the CSV are extracted iteratively and not all loaded into memory at once, and that all lines are extracted from CSVs.

The process of converting text lines to objects requires extensive test coverage due to the nature since many of the fields contain ASCII characters used for demarcating value separation in CSVs (for example, the comma). Tests ensure that CSVs are treated according to the RFC 4180 specification; qualifiers are handled correctly, columns line up correctly with the headers, values are handled correctly, and lines are correctly transformed into JavaScript objects. In terms of filtering lines, unit tests ensure that objects may be filtered for individual values for up to multiple attributes, that objects can be filtered any number of values on any number of attributes, and that filtering is done on an all-or-nothing-basis (objects either meet all filter requirements or are returned as ``null''). Unit tests assert that creation and whitelisting of attributes works as expected,

Tests that demonstrate correctness of nETL's CouchDB-loading module consist of a single unit test that inserts 3 documents into CouchDB using the the \_bulk\_docs API via a single API call, ensuring that the database exists, that permissions are correct, that network connectivity is available, etc.

\subsection{Manual Map \& List Function Tests}
Manually testing each of the processes described in Chapter \ref{chapter-analysis} was done using small dummy datasets with just 5 IDs. Based on this datset, the MapReduce output as shown in Figure \ref{fig-test-map-output} (i.e. the CouchDB view output) is as expected:

\begin{itemize}
  \item Correlation between Grades and admissions benchmarks
        \begin{itemize}
          \item Document output is ordered key[0], then[1], then key[2] - For every student output of benchmark data is followed by that student's grade data
          \item Multiple results from the same course are output in order, ordered by course registration date.
          \item Value output for the Grade documents is a single number, and output for the Benchmark document is an array of 8 numbers (that correspond to the CSV input)
          \item No documents appear that should be filtered out
          \item Output contains the correct number of documents
        \end{itemize}
  \item Correlation between Sakai presence events and $\Delta ClassRank$
        \begin{itemize}
          \item Event counts are correct, and the output format is correct for semester 1 and semester 2 for each student
          \item Document ordering is correct for each student (benchmarks output, followed by events output, followed by \textit{grades} output)
          \item Grade documents from years other than 2016 are not included in the output
        \end{itemize}
\end{itemize}

List function output of these views is shown in Figure \ref{fig-test-list-output} and shows that data across both entity types (for the 2-way join) and all three entity types (for the 3-way join) are joined correctly. One particular case worth testing is that when MapReduce output includes benchmarks and event counts, but no grade documents for a student, the List output does not include that student. This case often occurs when a student's took CSC1015F in a year other than 2016, and that students event data includes usage for courses other than CSC1015F in 2016.

\newpage
\begin{figure}[H]
    \centering
    \begin{mdframed}[rightline=false,leftline=false]
        \begin{minted}{text}
/* Two-way Join */
{"total_rows":9,"offset":0,"rows":[
    {"id":"<uuid>","key":[1,0,0],"value":[84,94,88,0,0,80,76,89]},
    {"id":"<uuid>","key":[1,"CSC1015F",2016],"value":70},
    {"id":"<uuid>","key":[2,0,0],"value":[76,76,78,0,0,75,58,61]},
    {"id":"<uuid>","key":[2,"CSC1015F",2016],"value":55},
    {"id":"<uuid>","key":[3,0,0],"value":[75,84,78,0,0,0,0,0]},
    {"id":"<uuid>","key":[3,"CSC1015F",2015],"value":77},
    {"id":"<uuid>","key":[4,0,0],"value":[82,94,85,0,0,73,71,86]},
    {"id":"<uuid>","key":[4,"CSC1015F",2015],"value":39},
    {"id":"<uuid>","key":[4,"CSC1015F",2016],"value":54}
]}

/* Three-way Join */
{"rows":[
    {"key":[1,0,0],"value":[84,94,88,0,0,80,76,89]},
    {"key":[1,0,2016],"value":[1,1]},
    {"key":[1,"CSC1015F",2016],"value":70},
    {"key":[2,0,0],"value":[76,76,78,0,0,75,58,61]},
    {"key":[2,0,2016],"value":[2,1]},
    {"key":[2,"CSC1015F",2016],"value":55},
    {"key":[3,0,0],"value":[75,84,78,0,0,0,0,0]},
    {"key":[3,0,2016],"value":[4,0]},
    {"key":[4,0,0],"value":[82,94,85,0,0,73,71,86]},
    {"key":[4,0,2016],"value":[0,1]},
    {"key":[4,"CSC1015F",2016],"value":54}
]}
        \end{minted}
    \end{mdframed}
    \caption[CouchDB map output]{\textbf{Figure \ref{fig-test-map-output}: CouchDB map output.}. Two way join API: \texttt{http://localhost:5984/test/\_design/two-way-join/\_view/2-way-join.csv?include\_docs=false}. Three-way join API \texttt{http://localhost:5984/test/\_design/three-way-join/\_view/3-way-join.csv?group=true\&reduce=true}. (include\_docs is not valid when reduce=true)}
    \label{fig-test-map-output}
\end{figure}

\begin{sidewaysfigure}
    \centering
    \begin{mdframed}[rightline=false,leftline=false,topline=false]
        \centering
        \begin{BVerbatim}
/* 2-way-join */
+------+----+----------+--------+----------+----------+----------+--------------+--------------+--------+--------+---------+
| Year | ID |  Course  | Course | Gr12 Eng | Gr12 Sci | Gr12 Mth | Gr12 Mth Lit | Gr12 Mth Adv | NBT AL | NBT QL | NBT Mth |
+------+----+----------+--------+----------+----------+----------+--------------+--------------+--------+--------+---------+
| 2016 |  1 | CSC1015F |     70 |       84 |       94 |       88 |            0 |            0 |     80 |     76 |      89 |
| 2016 |  2 | CSC1015F |     55 |       76 |       76 |       78 |            0 |            0 |     75 |     58 |      61 |
| 2015 |  3 | CSC1015F |     77 |       75 |       84 |       78 |            0 |            0 |      0 |      0 |       0 |
| 2015 |  4 | CSC1015F |     39 |       82 |       94 |       85 |            0 |            0 |     73 |     71 |      86 |
| 2016 |  4 | CSC1015F |     54 |       82 |       94 |       85 |            0 |            0 |     73 |     71 |      86 |
+------+----+----------+--------+----------+----------+----------+--------------+--------------+--------+--------+---------+

/* 3-way-join */
+------+----+----------+-------+----------+----------+----------+--------------+--------------+--------+--------+---------+-------+-------+
| Year | ID |  Course  | Grade | Gr12 Eng | Gr12 Sci | Gr12 Mth | Gr12 Mth Lit | Gr12 Mth Adv | NBT AL | NBT QL | NBT Mth | Ev S1 | Ev S2 |
+------+----+----------+-------+----------+----------+----------+--------------+--------------+--------+--------+---------+-------+-------+
| 2016 |  1 | CSC1015F |    70 |       84 |       94 |       88 |            0 |            0 |     80 |     76 |      89 |     1 |     1 |
| 2016 |  2 | CSC1015F |    55 |       76 |       76 |       78 |            0 |            0 |     75 |     58 |      61 |     2 |     1 |
| 2016 |  4 | CSC1015F |    54 |       82 |       94 |       85 |            0 |            0 |     73 |     71 |      86 |     0 |     1 |
+------+----+----------+-------+----------+----------+----------+--------------+--------------+--------+--------+---------+-------+-------+
        \end{BVerbatim}
    \end{mdframed}
    \caption[CouchDB List output]{\textbf{Figure \ref{fig-test-list-output}: List output of CouchDB test data.} The two-way join output is achieved via the URI \texttt{http://localhost:5984/test/\_design/two-way-join/\_list/2-way-join/2-way-join.csv}. The three-way join output is achieved via the URI \texttt{http://localhost:5984/test/\_design/three-way-join/\_list/3-way-join/3-way-join.csv?group=true\&reduce=true}}
    \label{fig-test-list-output}
\end{sidewaysfigure}


\subsection{Map \& List Function Unit Tests}
Unit testing the Map and List function code is a more involved, complicated process than testing the nETL components since the Map and List functions as structured for usage by the python-couchapp tool don't conform to valid JavaScript syntax and additionally invoke other functions provided by the CouchDB runtime environment. Unit testing these functions required loading List/Map function code as strings from their respective files and inserting the strings into the testing runtime environment via runtime evaluation (using the \mintinline{javascript}{eval(`var mapFn = ${mapFnStr}; var lstFn = ${lstFnStr};`)}). Function stubs for the \mintinline{javascript}{emit()}, \mintinline{javascript}{provides()}, \mintinline{javascript}{getRow()}, and \mintinline{javascript}{send()} CouchDB execution environment functions were authored - including a complicated generator-subroutine to mimic the contract of the \mintinline{javascript}{getRow()} function. The same sample CSVs as used for the manual Map/List function tests is used for unit testing - the documents as loaded into CouchDB via nETL are used as the source data. The benefit of unit testing in conjunction with manual testing is that vastly more assertions can be made in a much shorter time. For the purposes of this project, the same assertions are used as for the manual testing of these functions. These tests assert that:

\begin{itemize}
  \item todo: writeup the assertions used and complete the 3 way join code
\end{itemize}